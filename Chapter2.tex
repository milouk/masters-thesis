\chapter{\textlatin{Docker Background}}
\label{dockerBackground}

Σε αυτή την ενότητα θα συζητήσουμε για τις τεχνολογίες
\textlatin{containerization} (Ενότητα 3.1) καθώς και για το
\textlatin{Docker} (Ενότητα 3.2).


\section{Λογισμικό \textlatin{containerization}}

Το λογισμικό \textlatin{containerization} απομονώνει τις διεργασίες που
εκτελούνται σε έναν \textlatin{host} υπολογιστή μεταξύ τους. Μια διεργασία σε
ένα κοντέινερ "βλέπει" διαφορετικά το του host συστήμα από διεργασίες εκτός του
κοντέινερ. Μια διαδικασία μέσα σε ένα κοντέινερ έχει πρόσβαση σε διαφορετικά
αρχεία, διεπαφές δικτύου και χρήστες από τις διαδικασίες εκτός του κοντέινερ.
Οι διεργασίες μέσα στο κοντέινερ μπορούν να δουν μόνο άλλες διεργασίες μέσα
στο ίδιο κοντέινερ.


\begin{figure}[ht]
    \centering
    \begin{subfigure}[t]{.45\textwidth}
        \centering
        \begin{tikzpicture}[x=0.75pt,y=0.75pt,yscale=-1,xscale=1]
            % Process A Rectangle
            \draw (0,25) -- (95,25) -- (95,55) -- (0,55) -- cycle ;
            \draw (47.5,40) node {Διαδικασία Α};

            % Process B Rectangle
            \draw (105,25) -- (200,25) -- (200,55) -- (105,55) -- cycle ;
            \draw (152.5,40) node {Διαδικασία Β};

            % Host Rectangle
            \draw (0,65) -- (200,65) -- (200,95) -- (0,95) -- cycle ;
            \draw (100, 80) node {\textlatin{Linux Host}};
        \end{tikzpicture}
        \caption{Δυο διαδικασίες.}\label{subfig:processes}
    \end{subfigure}
    \begin{subfigure}[t]{.45\textwidth}
        \centering
        \begin{tikzpicture}[x=0.75pt,y=0.75pt,yscale=-1,xscale=1]
            % Process A Rectangle
            \draw (0,25) -- (90,25) -- (90,55) -- (0,55) -- cycle ;
            \draw (45,40) node {Διαδικασία Α};

            % Process B container Rectangle
            \draw (100,0) -- (200,0) -- (200,60) -- (100,60) -- cycle ;
            \draw (152.5,15) node {\textlatin{Container}};

            %% Process B Rectangle
            \draw (105,25) -- (195,25) -- (195,55) -- (105,55) -- cycle ;
            \draw (150,40) node {Διαδικασία Β};

            % Host Rectangle
            \draw (0,65) -- (200,65) -- (200,95) -- (0,95) -- cycle ;
            \draw (100, 80) node {\textlatin{Linux Host}};
        \end{tikzpicture}
        \caption{Μια διαδικασία μέσα σε ένα \textlatin{container}.}\label{subfig:container}
    \end{subfigure}
    \caption{Τρέχοντας δύο διαδικασίες εντός και εκτός \textlatin{container}.}\label{fig:with-without-container}
\end{figure}

Στο σχήμα 3.1 παρατηρούμε δύο σενάρια. Στο 3.1a παρατηρούμε τον συνηθισμένο τρόπο για να τρέξει μια διαδικάσια.
Το λειτουργικό σύστημα ξεκινά διαδικασίες που μπορούν να επικοινωνήσουν με
άλλες διαδικασίες. Η οπτική τους στο σύστημα αρχείων είναι η ίδια.
Στο σχήμα 3.1b μία από τις διεργασίες εκτελείται μέσα σε ένα δοχείο.
Αυτές οι διαδικασίες δεν μπορούν να επικοινωνούν μεταξύ τους. Εάν η
διαδικασία Α έχει πρόσβαση στα αρχεία στο  \texttt{\textlatin{/tmp}}, έχει πρόσβαση σε διαφορετικό
τμήμα του συστήματος αρχείων από ό,τι όταν η διαδικασία Β έχει πρόσβαση στα
αρχεία στη τοποθεσία \texttt{\textlatin{/tmp}} 
\footnote{Πρόσβαση στα αρχεία πρέπει να δοθεί ρητά όπως θα συζητήσουμε και παρακάτω στην ενότητα 3.2.3}.
Η διαδικασία Β δεν μπορεί καν να δει ότι η διαδικασία Α υπάρχει.
Η διαδικασία Α και η διαδικασία Β βλέπουν ένα τόσο διαφορετικό μέρος του
\textlatin{host} συστήματος που η διαδικασία Β φαίνεται σαν να τρέχει σε ένα
εντελώς διαφορετικό σύστημα.

\subsection{Πλεονεκτήματα \textlatin{Containerization}}

Τα \textlatin{containers} μπορούν να γίνουν εύκολα επεκτάσιμα πακέτα
που ονομάζονται εικόνες (\textlatin{images}).
Αυτές οι εικόνες περιέχουν μόνο τα απαραίτητα αρχεία για την εκτέλεση
συγκεκριμένου λογισμικού. Άλλα αρχεία, βιβλιοθήκες και \textlatin{binary}
αρχεία είναι κοινόχρηστα μεταξύ του λειτουργικού συστήματος του
\textlatin{host} υπολογιστή (το σύστημα πάνω στο οποίο τρέχει το
\textlatin{container}). Αυτό επιτρέπει στους προγραμματιστές
να δημιουργήσουν ελαφριές διανομές λογισμικού που περιέχουν μόνο
τις απαραίτητες εξαρτήσεις.


Αυτές οι εικόνες μπορούν να δημιουργηθούν για να προσομοιώσουν ένα σύστημα
αρχείων (\textlatin{file system})
μίας διαφορετικής διανομής \textlatin{Linux}. Για παράδειγμα, εάν μια
εφαρμογή έχει αναπτυχθεί ειδικά για το λειτουργικό \textlatin{centOS} και
δεν εκτελείται στο \textlatin{Ubuntu}, τότε είναι δυνατή η
δημιουργία μιας εικόνας που περιέχει όλα τα απαραίτητα αρχεία που αφορούν το
\textlatin{CentOS} και άλλες εξαρτήσεις. Αυτό Η εικόνα μπορεί στη συνέχεια να εκτελεστεί
σε έναν κεντρικό υπολογιστή που εκτελεί το \textlatin{Ubuntu}. Στην εφαρμογή που
εκτελείται μέσα σε ένα κοντέινερ που εκτελεί μια παρουσία της εικόνας, το
λειτουργικό σύστημα είναι η \textlatin{CentOS}.

Τα \textlatin{containers} καθιστούν επίσης δυνατή την εκτέλεση πολλαπλών
εκδόσεων του ίδιου λογισμικό σε έναν υπολογιστή.
Κάθε \textlatin{container} μπορεί να
περιέχει μια συγκεκριμένη έκδοση και όλα τα τα \textlatin{containers} να τρέχουν
στον ίδιο κεντρικό υπολογιστή. Επειδή τα δοχεία είναι απομονωμένα μεταξύ τους,
οι ασυμβίβαστες εξαρτήσεις τους δεν δημιουργούν πρόβλημα. \\


Για παράδειγμα, αν θέλουμε να εκτελέσουμε το λογισμικό του
\textlatin{Wordpress} \footnote{Ένα δημοφιλές σύστημα διαχείρησης περιεχομένου
για να χτίζει κανείς ιστοσελίδες}, δεν χρειάζεται να εγκαταστήσουμε όλες τις
εξαρτήσεις του \textlatin{wordpress}. Το μόνο που χρειάζεται είναι να
κατεβάσουμε την εικόνα που δημιούργησαν οι προγραμματιστές του
\textlatin{Wordpress}. Η εικόνα περιέχει όλες τα προεγκατεστημένες εξαρτήσεις.

Αν θέλουμε να δοκιμάσουμε μια νεότερη έκδοση του \textlatin{Wordpress} στον ίδιο
κεντρικό υπολογιστή, πρέπει μόνο να "σηκώσουμε" ένα νέο \textlatin{container}
στον ίδιο κεντρικό υπολογιστή. Οι ασύμβατες εξαρτήσεις των δύο οντοτήτων του
\textlatin{Wordpress} δεν αποτελούν πρόβλημα, επειδή βλέπουν διαφορετικά
τμήματα του συστήματος αρχείων και δεν βλέπουν καν το ένα τις διαδικασίες του
άλλου.

Η απλότητα που φέρνει το \textlatin{containerization}, την  καθιστά εξαιρετικά
δημοφιλή στην ανάπτυξη λογισμικού, τη συντήρηση και την ανάπτυξη του.


\subsection{\textlatin{Virtualization}}

Το \textlatin{virtualization} είναι μια παλαιότερη, παρόμοια τεχνική με την
απομόνωση λογισμικού. Στο \textlatin{virtualization}, ένα ολόκληρο σύστημα
προσομοιώνεται πάνω στον κεντρικό υπολογιστή. Αυτό το νέο εικονικό 
μηχάνημα ονομάζεται επισκέπτης (\textlatin{guest}). Ο \textlatin{guest} και ο
\textlatin{host} δεν μοιράζονται κανένα πόρο του συστήματος. Αυτό έχει κάποια
πλεονεκτήματα, για παράδειγμα, επιτρέπει την εκτέλεση
ενός εντελώς διαφορετικού λειτουργικού συστήματος \textlatin{guest} 
(π.χ. μπορούμε να τρέξουμε λειτουργικό \textlatin{Windows} σε
\textlatin{Linux host}).

Το λογισμικό που διαχειρίζεται την εικονική μηχανή ονομάζεται \textlatin{hypervisor}.
Ο \textlatin{hypervisor} μπορεί να εκτελεστεί πάνω σε ένα λειτουργικό σύστημα ή
να εκτελεστεί απευθείας στο υλικό απευθείας(που ονομάζεται 
\textlatin{bare-metal hypervisor}).


\begin{figure}[ht]
    \centering
    \begin{subfigure}[t]{.45\textwidth}
        \centering
        \begin{tikzpicture}[x=0.75pt,y=0.75pt,yscale=-1,xscale=1]
            % VM Process A Rectangle
            \draw (0,0) -- (100,0) -- (100,100) -- (0,100) -- cycle ;
            \draw (50,10) node {{\tiny Εικονική Μηχανή}};

            % OS Process A Rectangle
            \draw (5,20) -- (95,20) -- (95,95) -- (5,95) -- cycle ;
            \draw (50,30) node {\small Λειτουργικό};

            % Process A Rectangle
            \draw (10,40) -- (90,40) -- (90,70) -- (10,70) -- cycle ;
            \draw (50,55) node {\tiny Διαδικασία Α};

            % VM Process B Rectangle
            \draw (105,0) -- (205,0) -- (205,100) -- (105,100) -- cycle ;
            \draw (155,10) node {{\tiny Εικονική Μηχανή}};

            % OS Process B Rectangle
            \draw (110,20) -- (200,20) -- (200,95) -- (110,95) -- cycle ;
            \draw (155,30) node {\small Λειτουργικό};

            % Process B Rectangle
            \draw (115,40) -- (195,40) -- (195,70) -- (115,70) -- cycle ;
            \draw (155,55) node {\tiny Διαδικασία Β};

            % Hypervisor
            \draw (0,105) -- (205,105) -- (205,120) -- (0,120) -- cycle ;
            \draw (102.5,112.5) node {{\small \textlatin{Hypervisor}}};

            % Linux Host
            \draw (0,125) -- (205,125) -- (205,155) -- (0,155) -- cycle ;
            \draw (102.5,140) node {\textlatin{Linux Host}};
        \end{tikzpicture}
        \caption{Εικονικές Μηχανές}
    \end{subfigure}
    \begin{subfigure}[t]{.45\textwidth}
        \centering
         \begin{tikzpicture}[x=0.75pt,y=0.75pt,yscale=-1,xscale=1]
            % Container Process A Rectangle
            \draw (0,0) -- (97.5,0) -- (97.5,50) -- (0,50) -- cycle ;
            \draw (50,10) node {\textlatin{Container}};

            % Process A Rectangle
            \draw (5,20) -- (95,20) -- (95,45) -- (5,45) -- cycle ;
            \draw (50,32.5) node {\small Διαδικασία Α};

            % Container Process B Rectangle
            \draw (102.5,0) -- (200,0) -- (200,50) -- (102.5,50) -- cycle ;
            \draw (150,10) node {\textlatin{Container}};

            % Process B Rectangle
            \draw (105,20) -- (195,20) -- (195,45) -- (105,45) -- cycle ;
            \draw (150,32.5) node {\small Διαδικασία Β};

            % Host Rectangle
            \draw (0,55) -- (200,55) -- (200,85) -- (0,85) -- cycle ;
            \draw (100, 70) node {\textlatin{Linux Host}};
        \end{tikzpicture}
        \caption{\textlatin{Containers}}
    \end{subfigure}
    \caption{Σύγκριση εικονικών μηχανών και κοντέινερς}
\end{figure}


Επειδή το λογισμικό \textlatin{containerization} μοιράζεται πολλούς πόρους με
τον κεντρικό υπολογιστή,
είναι πολύ πιο γρήγορο και πιο ευέλικτο από το \textlatin{virtualization}.Tο
\textlatin{virtualization} χρειάζεται να ξεκινήσει ένα εντελώς νέο λειτουργικό
σύστημα, ενώ το το λογισμικό \textlatin{containerization} το μόνο που
χρειάζεται είναι η ξεκινήσει είναι μια μόνο διαδικασία.

\subsection{Ο αντίκτυπος των \textlatin{Containers} στην ασφάλεια}

Ένα \textlatin{container} απομονώνει το λογισμικό από τον κεντρικό υπολογιστή,
αλλά δεν το αλλάζει. Αυτό σημαίνει ότι τα τρωτά σημεία στο λογισμικό δεν
επηρεάζονται από την εκτέλεση αυτού του λογισμικού μέσα σε περιβάλλον ενός
\textlatin{container}.
Ωστόσο, ο αντίκτυπος αυτών των τρωτών σημείων είναι περιορισμένος, επειδή η
ευπάθεια υπάρχει σε ένα απομονωμένο περιβάλλον.

Εάν, για παράδειγμα, μπορεί κάποιος να εκτελέσει απομακρυσμένα  κώδικα
(\textlatin{RCE})
\footnote{Απομακρυνσμένη εκτέλεση κώδικα είναι μια ευπάθεια στην οποία ο
επιτηθέμενος καταφέρνει να εκτελέσει απομακρυνσμένα δικό του κώδικα σε μια
ευπαθή μηχανή} καθώς τρέχουμε \textlatin{Wordpress}, η εκτέλεση του Wordpress
σε ένα \textlatin{container} δεν διορθώνει την ευπάθεια. Ένας εισβολέας
εξακολουθεί να μπορεί να το εκμεταλλευτεί. Αλλά ο επιτιθέμενος είναι
πολύ λιγότερο πιθανό να αποκτήσει πρόσβαση στο κεντρικό σύστημα, καθώς το
λογισμικό που εκμεταλλέτηκε είναι απομονωνμένο από το σύστημα υποδοχής λόγω του
\textlatin{containerization}.

Επειδή ένα \textlatin{container} χρησιμοποιεί τον ίδιο πυρήνα και πόρους με
τον κεντρικό υπολογιστή, ένα
\texttt{root exploit} (δηλ. μια εκμετάλλευση που επιτρέπει στους μη
προνομιούχους χρήστες να αυξήσουν τα προνόμιά τους) μπορεί να είναι εξίσου
αποτελεσματική στο εσωτερικό όσο και έξω από το κοντέινερ, επειδή
ο στόχος (π.χ. ο πυρήνας) είναι ο ίδιος. Η ευπάθεια με κωδικό
\textlatin{CVE–2016–5195 (Dirty Cow)}
\footnote{\textlatin{https://dirtycow.ninja/}} είναι ένα καλό παράδειγμα εκμετάλλευσης που επιτρέπει
να διαφύγει κανείς από το \textlatin{container} και να αποκτήσει πρόσβαση στο
\textlatin{host}, επειδή επιτίθεται στον πυρήνα του οικοδεσπότη.
\cite{Dirty-Cow-Escape}


\section{\textlatin{Docker}}

Η έννοια του \textlatin{containerization} υπάρχει εδώ και πολύ καιρό
\footnote{\textlatin{https://docs.freebsd.org/44doc/papers/jail/jail-9.html}}.
Τα τελευταία χρόνια όμως έχει αποκτήσει σημαντική φήμη ως ένας δημοφιλής τρόπος
πακεταρίσματος, διαμοιρασμού και εκτέλεσης λογισμικού. Αυτό οφείλεται κυρίως
στο \textlatin{Docker}. \cite{Docker-Popular} \\

Το \textlatin{Docker} κυκλοφόρησε το 2013 και δεν προσφέρει μόνο έναν τρόπο
για να κάνει κανείς \textlatin{containerize} λογισμικό, αλλά και έναν τρόπο
διανομής των \textlatin{container}. Αυτό δίνει τη δυνατότητα στους δημιουργούς
λογισμικού
(δηλαδή προγραμματιστές και οργανισμοί) για τη δημιουργία και τη διανομή πακέτων
που δεν έχουν εξαρτήσεις. Αν θέλουμε να τρέξουμε μια συγκεκριμένο εφαρμογή,
χρειάζεται μόνο να κατεβάσουμε το πακέτο που οι προγραμματιστές της
εφαρμογής έχουν δημιουργήσει. Αυτό επιτρέπει πολύ ταχύτερη ανάπτυξη και
\textlatin{deployment}, επειδή δεν υπάρχουν πλέον εξαρτήσεις και η εγκατάσταση
λογισμικού δεν αποτελεί πια μια ανησυχία.


\subsection{Βασικές Έννοιες}

Το \texttt{\textlatin{Docker}} αποτελείται από μερικές έννοιες: δαίμονες,
εικόνες, \textlatin{containers} και \texttt{\textlatin{Dockerfiles}}. Αυτά θα
καλυφθούν στις επόμενες ενότητες.

\subsubsection{Δαίμονες}

Ο δαίμονας είναι μια υπηρεσία (ένα πρόγραμμα με δικαιώματα που εκτελείται στο
παρασκήνιο) που εκτελείται (ως \textlatin{root} 
\footnote{Γίνονται προσπάθειες για μια πειραματική έκδοση χωρίς την ανάγκη
για \textlatin{root}.}
\footnote{\textlatin{https://github.com/docker/engine/blob/master/docs/rootless.md}})
στον κεντρικό υπολογιστή. Διαχειρίζεται
όλα τα πράγματα που σχετίζονται με το \textlatin{Docker} σε αυτό το μηχάνημα.
Για παράδειγμα, εάν ένας χρήστης χρειάζεται να επανεκκινήσει ένα 
\textlatin{container}, τότε ο δαίμονας του \textlatin{Docker} είναι η διαδικασία
που επανεκκινεί το \textlatin{container}. Καλό είναι να σημειωθεί αυτό, επειδή
όλα όσα σχετίζονται με το \textlatin{Docker} τα χειρίζεται ο δαίμονας και
το \textlatin{Docker} έχει πρόσβαση σε όλους τους πόρους του κεντρικού
υπολογιστή (επειδή εκτελείται ως \textlatin{root}), η δυνατότητα χρήσης του
\textlatin{Docker} ισοδυναμεί με την πρόσβαση \textlatin{root} στον κεντρικό
υπολογιστή
\footnote{\textlatin{https://docs.docker.com/engine/security/security/}}.

\subsubsection{Εικόνες (\textlatin{Docker Images})}

Μια εικόνα \textlatin{Docker} είναι μια δομή συσκευασμένων καταλόγων 
(\textlatin{Packaged directory structure}). Είναι
ένα σύνολο στρωμάτων. Καθε επίπεδο προσθέτει, αλλάζει ή αφαιρεί συγκεκριμένα
αρχεία ή καταλόγους στην εικόνα. Το πρώτο επίπεδο (που ονομάζεται εικόνα βάσης)
είναι είτε μια υπάρχουσα εικόνα είτε τίποτα (αναφέρεται ως \textlatin{scratch}).
Κάθε στρώμα πάνω από αυτό είναι μια αλλαγή στο πρηγούμενο στρώμα.

\subsubsection{\textlatin{Containers}}

Ένα \textlatin{container} είναι ένα \textlatin{instance} μιας εικόνας
\textlatin{Docker} που βρίσκεται σε εκτέλεση. Αν τρέχουμε λογισμικό
συσκευασμένο ως εικόνα \textlatin{Docker}, δημιουργούμε ένα
\textlatin{container} με βάση αυτήν την εικόνα. Αν Θέλουμε να εκτελέσουμε
δύο \textlatin{instances} της ίδιας εικόνας \textlatin{Docker}, μπορούμε να
δημιουργήσουμε δύο \textlatin{containers}.

\subsubsection{\texttt{\textlatin{Dockerfiles}}}

Ένα \texttt{\textlatin{Dockerfile}} περιγράφει από ποια επίπεδα αποτελείται
μια εικόνα \textlatin{Docker}. Περιγράφει τα βήματα για τη δημιουργία της
εικόνας. Ας δούμε ένα βασικό παράδειγμα: \\

\texttt{\textlatin{FROM alpine:latest}}

\texttt{\textlatin{RUN apt-get update}}

\texttt{\textlatin{COPY . /app}}

\texttt{\textlatin{RUN pip install -r requirements.txt}}

\texttt{\textlatin{EXPOSE 8000}}

Οι εντολές αυτές θα πουν στο \textlatin{Docker Engine} πως να δημιουργήσει μια
νέα εικόνα \textlatin{Docker}.

Ενδεικτικά:

\begin{enumerate}
 
   \item H εντολή \texttt{\textlatin{FROM}} δείχνει στο \textlatin{Docker} πάνω
    σε τι να βασιστεί η εικόνα. Αντί να ξεκινήσει απο μια άδεια εικόνα λέμε
    στο \textlatin{Docker} να βασιστεί σε μια προϋπάρχουσα εικόνα (σε αυτή
    τη περίπτωση την \textlatin{Alpine Linux}).

    \item H εντολή \texttt{\textlatin{RUN}} δηλώνει ότι θέλουμε να τρέξουμε μια
    \textlatin{unix} εντολή όπως θα κάναμε τοπικά στο τερματικό μας. Στην
    προκειμένη περίπτωση αναβαθμίζουμε όλες τα πακέτα της βασικής μας
    εικόνας.

    \item H εντολή \texttt{\textlatin{COPY}} αντιγράφει το \texttt{.} δηλαδή το
    \textlatin{root} που βρίσκεται το \textlatin{Dockerfile} μας στη μονοπάτι
    \texttt{\textlatin{/app}} του \textlatin{container} (αν δεν υπάρχει το
    δημιουργεί). 

    \item Η εντολή \texttt{\textlatin{EXPOSE}} ουσιαστικά ανοίγει την πόρτα
    8000 από το εσωτερικό του \textlatin{container}.

\end{enumerate}

Υπάρχουν πολλές ακόμα εντολές που μπορούν να χρησιμοποιηθούν μέσα σε ένα
\textlatin{Dockerfile} 
\footnote{\textlatin{https://docs.docker.com/engine/reference/builder/}}.


\subsubsection{Εσωτερικά δομικά στοιχεία}

Ένα \textlatin{Docker Container} είναι στην πραγματικότητα ένας συνδυασμός
πολλαπλών χαρακτηριστικών εντός του πυρήνα \textlatin{Linux}. Κυρίως 
\texttt{\textlatin{cgroups}}, \texttt{\textlatin{namespaces}} και 
\texttt{\textlatin{OverlayFS}}. \\

Οι ομάδες ελέγχου (\textlatin{cgroups}) είναι ένας τρόπος περιορισμού των
πόρων (π.χ. \textlatin{CPU} και χρήση \textlatin{RAM}) σε (ομάδες) διεργασιών
και παρακολούθηση αυτών των διεργασιών.

Τα \texttt{\textlatin{namespaces}} είναι ένας τρόπος απομόνωσης πόρων από
διεργασίες. Για παράδειγμα, εάν προσθέσουμε μια διεργασία σε ένα
\textlatin{process namespace}, μπορεί να δει μόνο τις διεργασίες
σε αυτόν το \textlatin{namespace}. Αυτό επιτρέπει στις διαδικασίες να είναι
απομωνομένες μεταξύ τους. Το \textlatin{Linux} υποστηρίζει τους ακόλουθους
τύπους χώρων ονομάτων. \\

• \textbf{\textlatin{Cgroup}}: Για απομόνωση διεργασιών από ιεραρχίες
\textlatin{cgroup}.

• \textbf{\textlatin{IPC}}: Απομονώνει την επικοινωνία μεταξύ διεργασιών.
Αυτό, για παράδειγμα, απομονώνει περιοχές κοινής μνήμης.

• \textbf{\textlatin{Network}}: Απομονώνει τη στοίβα δικτύου (π.χ. διευθύνσεις
\textlatin{IP}, διεπαφές, \textlatin{routes} και πόρτες).

• \textbf{\textlatin{Mount}}: Απομονώνει τα \textlatin{mount points}. Όταν
δημιουργείται ένα νέο \textlatin{mount namespaces}, τα υπάρχοντα
\textlatin{mount points} αντιγράφονται από το τρέχον \textlatin{namespaces}.
Τα νέα \textlatin{mount points} δεν διαδίδονται. \\

Ένα\textlatin{mount namespace} είναι παρόμοιο με ένα
\textlatin{\texttt{chroot} jail}. Μια \textlatin{\texttt{chroot} jail}
αλλάζει τον \textlatin{root} κατάλογο για μια συγκεκριμένη διαδικασία.
Αυτή η διαδικασία δεν μπορεί να έχει πρόσβαση αρχεία εκτός της νέας ρίζας.

• \textbf{\textlatin{PID}}: Απομονώνει διεργασίες από το να βλέπουν
αναγνωριστικά διεργασιών σε άλλα \textlatin{namespaces}.
Οι διεργασίες σε διαφορετικά \textlatin{namespaces} μπορούν να έχουν το ίδιο
\textlatin{PID}.

• \textbf{\textlatin{User}}: Απομονώνει τους χρήστες και τις ομάδες.

• \textbf{\textlatin{UTS}}: Απομονώνει τα ονόματα κεντρικού υπολογιστή και
τομέα. \\

Όταν ο δαίμονας \textlatin{Docker} δημιουργεί ένα νέο κοντέινερ, δημιουργεί ένα
νέο
χώρο ονομάτων (\textlatin{namespaces}) κάθε τύπου για τη διεργασία που
εκτελείται στο κοντέινερ. Με αυτό το τρόπο το κοντέινερ δεν μπορεί να δει καμία
από τις διεργασίες, τις διεπαφές δικτύου και τα \textlatin{mount points} του
κεντρικού υπολογιστή (από προεπιλογή μπορεί
να επικοινωνήσει με άλλο \textlatin{Docker} κοντέινερ, επειδή είναι συνδεδεμένο
με το εσωτερικό δίκτυο \textlatin{Docker}). Στο κοντέινερ φαίνεται ότι στην
πραγματικότητα εκτελεί μια εντελώς ξεχωριστό λειτουργικό σύστημα. \\

Το \texttt{\textlatin{OverlayFS}} είναι ένα σύστημα αρχείων 
(\textlatin{union mount}) που επιτρέπει το συνδυασμό πολλών
καταλόγων και τους παρουσιάζει σαν να είναι ένας. Αυτό χρησιμοποιείται για να
δείξει το πολλαπλά επίπεδα σε μια εικόνα Docker ως ένας ενιαίος ριζικός
κατάλογος.

\subsection{Διατήρηση δεδομένων}

Χωρίς πρόσθετη διαμόρφωση, ένα κοντέινερ \textlatin{Docker} δεν έχει μόνιμη
αποθήκευση. Η αποθήκευσή του διατηρείται όταν σταματά το δοχείο, αλλά
όχι όταν αφαιρείται το κοντέινερ. Είναι δυνατή η προσάρτηση ενός καταλόγου
στον \textlatin{host} μέσα σε ένα κοντέινερ \textlatin{Docker}. Αυτό επιτρέπει
στο κοντέινερ να έχει πρόσβαση σε αρχεία του κεντρικό υπολογιστή και να
αποθηκεύει σε αυτόν τον προσαρτημένο κατάλογο.


\texttt{\textlatin{(host)\$ echo test > /tmp/test}}

\texttt{\textlatin{(host)\$ docker run -it --rm -v /tmp:/host-tmp ubuntu:latest}}

\texttt{\textlatin{bash}}

\texttt{\textlatin{(cont)\# cat /host-tmp/test}}

\texttt{\textlatin{test}}

\texttt{\textlatin{(cont)\# cat /tmp/test}}

\texttt{\textlatin{cat: /tmp/test: No such file or directory}} \\

ο κατάλογος \textbf{\texttt{\textlatin{/tmp}}} του \textlatin{host} προσαρτάται
στο κοντέινερ ως \textlatin{/host-tmp}. Μπορούμε να δούμε ότι ένα αρχείο που
δημιουργείται στον κεντρικό υπολογιστή είναι αναγνώσιμο από το κοντέινερ.
Βλέπουμε επίσης ότι το κοντέινερ έχει το δικό του
\textbf{\texttt{\textlatin{/tmp}}} κατάλογο, ο οποίος δεν έχει καμία σχέση με το
\textlatin{/host-tmp}.

\subsection{Δικτύωση}

Όταν δημιουργείται ένα κοντέινερ \textlatin{Docker}, ο δαίμονας
\textlatin{Docker} δημιουργεί ένα \textlatin{network sandbox} για αυτό το
κοντέινερ και (από προεπιλογή) το συνδέει σε ένα εσωτερικό δίκτυο. Αυτό δίνει
στους πόρους δικτύωσης κοντέινερ (π.χ. διεύθυνση \textlatin{IPv4}
\footnote{Υποστήριξη \textlatin{IPV6} δεν είναι ενεργοποίημένη από προεπιλογή},
διαδρομές και καταχωρήσεις \textlatin{DNS}) που είναι ξεχωριστά από τον
κεντρικό υπολογιστή. Όλα η εισερχόμενη και η εξερχόμενη κίνηση στο κοντέινερ
δρομολογείται μέσω μιας διεπαφής (από προεπιλογή) το οποίο γεφυρώνεται 
\footnote{Μια διεαπαφή τύπου \textlatin{bridge} είναι μια διεπαφή η οποία
συνδέει το δίκτυο μιας διεπαφής στο δίκτυο μιας άλλης διεπαφής} 
σε μια διεπαφή στον κεντρικό υπολογιστή. δηλ. \textlatin{host}.

Η εισερχόμενη κίνηση (που δεν αποτελεί μέρος μιας υπάρχουσας σύνδεσης) είναι
δυνατή μέσα από δρομολόγηση της κυκλοφορίας για συγκεκριμένες πόρτες από τον
κεντρικό υπολογιστή στο κοντέινερ. Ο προσδιορισμός του ποιες θύρες στον κεντρικό
υπολογιστή δρομολογούνται σε ποιες θύρες στο κοντέινερ γίνεται κατά την 
δημιουργία ενός κοντέινερ. Αν, για παράδειγμα, θέλουμε να "εκθέσουμε" τη θύρα
8000 στην εικόνα \textlatin{Docker} που δημιουργήθηκε από τις εντολές στην
παράγραφο \texttt{\textlatin{Dockerfiles}} μπορούμε είτε να το κάνουμε όπως
δείξαμε πριν είτε πιο εύκολα να εκτελέσουμε τις εξής εντολές. \\

\texttt{\textlatin{(host)\$ docker build -t test .}}

\texttt{\textlatin{docker run --rm -p 8000:8000 --name=test container test}} \\


Η πρώτη εντολή δημιουργεί μια εικόνα \textlatin{Docker} χρησιμοποιώντας το
αρχείο \textlatin{Docker} και στη συνέχεια δημιουργούμε (και ξεκινάμε) ένα
κοντέινερ από αυτήν την εικόνα. «Δημοσιεύουμε» την πόρτα
8000 του κεντρικού υπολογιστή στη θύρα 8000 του κοντέινερ. Αυτό σημαίνει ότι,
ενώ το κοντέινερ εκτελείται, όλη η κίνηση από τη θύρα 8000 στον κεντρικό
υπολογιστή δρομολογείται στη θύρα 8000 μέσα στο κοντέινερ.

Από προεπιλογή, όλα τα κοντέινερ \textlatin{Docker} προστίθενται στο ίδιο
εσωτερικό δίκτυο. Αυτό σημαίνει ότι (από προεπιλογή) όλα τα κοντέινερ
\textlatin{Docker} μπορούν να επικοινωνήσουν μεταξύ τους μέσω του δικτύου.
Αυτό διαφέρει από την απομόνωση την οποία χρησιμοποιεί το \textlatin{Docker}
για τα \textlatin{namespaces} του. Σε άλλα \textlatin{namespaces}, το
\textlatin{Docker} απομονώνει κοντέινερς από τον οικοδεσπότη και από άλλα
κοντέινερ. Αυτή η διαφορά στο σχεδιασμό μπορεί να οδηγήσει σε επικίνδυνες
εσφαλμένες διαμορφώσεις (\textlatin{misconfigurations}), επειδή οι
προγραμματιστές μπορεί να πιστεύουν ότι τα \textlatin{Docker} κοντέινερς
είναι εντελώς απομονωμένα μεταξύ τους (συμπεριλαμβανομένου και του δικτύου).

\subsection{Μηχανισμοί Προστασίας}

Για να μειωθούν σημαντικά οι κίνδυνοι που θέτουν τα (μελλοντικά) τρωτά σημεία
σε ένα σύστημα με \textlatin{Docker}, υπάρχουν πολλαπλοί μηχανισμοί προστασίας
ενσωματωμένοι στο \textlatin{Docker} και στον ίδιο τον πυρήνα \textlatin{Linux}.
Σε αυτή την ενότητα, θα δούμε τους πιο γνωστούς και τα πιο σημαντικούς
μηχανισμούς προστασίας.

Θα πρέπει να σημειωθεί ότι επειδή αυτοί οι μηχανισμοί προσθέτουν πολυπλοκότητα
και χαρακτηριστικά, ορισμένες ευπάθειες επικεντρώνονται αποκλειστικά στην
παράκαμψη ενός ή περισσότερων μηχανισμών προστασίας. Για παράδειγμα το
\textlatin{CVE–2019–5021} το οποίο θα μελετήσουμε σε επόμενο κεφάλαιο.

\subsubsection{Δυνατότητες}

Για να επιτρέπεται ή να μην επιτρέπεται σε μια διαδικασία να χρησιμοποιεί μια
συγκεκριμένη λειτουργικότητα η οποία απαιτεί επαυξημένα δικαιώματα,
ο πυρήνας του \textlatin{Linux} έχει ένα χαρακτηριστικό που ονομάζεται
"δυνατότητες". Μια δυνατότητα είναι ένας λεπτομερής τρόπος παροχής ορισμένων
προνομίων σε διαδικασίες. Μια δυνατότητα επιτρέπει σε μια διαδικασία να
εκτελέσει μια προνομιακή ενέργεια χωρίς να δώσουμε στη διαδικασία πλήρη
δικαιώματα ρίζας (\textlatin{root privileges}). Για παράδειγμα, αν θέλουμε μια
διεργασία να μπορεί να δημιουργήσει τα δικά της πακέτα δικτύου, της δίνουμε
μόνο τη δυνατότητα \textlatin{CAP\_NET\_RAW}.

Από προεπιλογή, κάθε κοντέινερ \textlatin{Docker} ξεκινά μόνο με τις ελάχιστες
απαραίτητες δυνατότητες. Οι προεπιλεγμένες δυνατότητες βρίσκονται στο κώδικα
\textlatin{Docker}
\footnote{\textlatin{https://github.com/moby/moby/blob/master/oci/caps/defaults.go}}
Μπορούμε να προσθέσουμε ή αφαιρέσουμε δυνατοτήτες κατά το χρόνο εκτέλεσης
χρησιμοποιώντας το ορίσματα \texttt{\textlatin{--cap-add}} και
\texttt{\textlatin{--cap-drop}} \cite{More-Secure-Non-Root-Container}.

\subsubsection{\textlatin{Secure Computing Mode}}

Η λειτουργία Ασφαλούς Υπολογισμού (\textlatin{seccomp}), όπως και οι
δυνατότητες, είναι ένας ενσωματωμένος τρόπος να περιορίσει την προνομιακή
λειτουργικότητα που επιτρέπεται να χρησιμοποιεί μια διεργασία. Οπου
oι δυνατότητες περιορίζουν τη λειτουργικότητα (όπως η ανάγνωση προνομιακών
αρχείων), η λειτουργία Ασφαλούς Υπολογισμού περιορίζει συγκεκριμένες κλήσεις
συστήματος (\textlatin{syscalls}). Αυτό επιτρέπει λεπτομερή έλεγχο ασφαλείας.
Αυτό επιτυγχάνεται μέσα από \textlatin{whitelists} (που ονομάζονται προφίλ)
των \textlatin{syscalls}. Για την ρύθμιση ενός αυστηρού αλλά λειτουργικού
προφίλ \textlatin{seccomp} απαταίτειται βαθιά γνώση ποιών \textlatin{syscalls}
γίνονται από το κάθε πρόγραμμα.

Το προεπιλεγμένο προφίλ \textlatin{seccomp} που λαμβάνουν οι διαδικασίες στα
κοντέινερ \textlatin{Docker} βρίσκεται στον πηγαίο κώδικα
\footnote{\textlatin{https://github.com/moby/moby/blob/master/profiles/seccomp/default.json}}.
Για να χρησιμοποιήσουμε ένα δικό μας προφίλ \textlatin{seccomp} το μπορεί να
χρησιμοποιηθεί το \texttt{\textlatin{--security-opt}} \textlatin{seccomp}.

\subsubsection{\textlatin{Application Armor}}

Το \textlatin{Application Armor (AppArmor)} είναι ένα \textlatin{module} του
πυρήνα που επιτρέπει περιορισμούς αρχείων και πόρων συστήματος για
συγκεκριμένες εφαρμογές. Το Docker προσθέτει ένα προεπιλεγμένο προφίλ
\textlatin{AppArmor} σε κάθε κοντέινερ. Αυτό είναι ένα
προφίλ που δημιουργείται κατά το χρόνο εκτέλεσης με βάση ένα πρότυπο
\footnote{\textlatin{https://github.com/moby/moby/blob/master/profiles/apparmor/template.go}}.

Είναι επίσης δυνατή η δημιουργία προσαρμοσμένων προφίλ \textlatin{AppArmor}, 
χρησιμοποιώντας κάποιο εργααλείο όπως το bane
\footnote{\textlatin{https://github.com/genuinetools/bane]}}.

\subsubsection{\textlatin{SElinux}}

Το \textlatin{Security-Enhanced Linux (SELinux)} είναι ένα σύνολο αλλαγών στον
πυρήνα του \textlatin{Linux} που υποστηρίζουν έλεγχο πρόσβασης σε όλο το
σύστημα για αρχεία και πόρους συστήματος. Διατίθεται από προεπιλογή σε
ορισμένες διανομές \textlatin{Linux} (π.χ. \textlatin{Red Hat Linux} διανομές).

Το \textlatin{Docker} δεν ενεργοποιεί την υποστήριξη \textlatin{SELinux} από
προεπιλογή, αλλά παρέχει μια πολιτική \textlatin{SELinux}
\footnote{\textlatin{https://www.mankier.com/8/docker\_selinux}}.

\subsubsection{Όχι \textlatin{root} χρήστες σε κοντέινερ}

Εκτός από τους μηχανισμούς προστασίας στον κεντρικό υπολογιστή, υπάρχουν και 
μηχανισμοί προστασίας σε επίπεδο εικόνας \textlatin{Docker}. Ο πιο σημαντικός
μηχανισμός προστασίας που μπορούν να εφαρμόσουν οι δημιουργοί εικόνων
\textlatin{Docker} δεν εκτελούν διεργασίες μέσα στο κοντέινερ ως
\texttt{\textlatin{root}}.

Από προεπιλογή, οι διεργασίες στα κοντέινερ \textlatin{Docker} εκτελούνται ως
\texttt{\textlatin{root}}, επειδή η διαδικασία είναι απομονωμένη από τον
κεντρικό υπολογιστή. Ωστόσο, όπως θα δούμε, υπάρχουν πολλοί τρόποι διαφυγής
από κοντέινερ. Οι περισσότεροι από αυτούς τους τρόπους απαιτούν δικαιώματα
\texttt{\textlatin{root}} (μέσα στο κοντέινερ). Για αυτό το λόγο συνιστάται η
εκτέλεση διεργασιών σε κοντέινερ χωρίς χρήση \texttt{\textlatin{root}}. Αν
κάποιος καταφέρει να παραβιάσει το κοντέινερ δεν θα καταφέρει να διαφύγει από
αυτό διότι δεν έχει δικαιώματα \texttt{\textlatin{root}}.

\subsection{\texttt{\textlatin{docker-compose}}}

Το \texttt{\textlatin{docker-compose}} είναι ένα πρόγραμμα \textlatin{wrapper}
(πρόγραμμα που απλοποιεί τη χρήση ενός άλλου προγράμματος) γύρω από το
\textlatin{Docker} που μπορεί να χρησιμοποιηθεί για τον καθορισμό
\textlatin{configuration} κοντέινερ \textlatin{Docker} σε αρχεία
\textlatin{YAML} \footnote{\textlatin{https://yaml.org/]}}. Αυτά τα αρχεία
καταργούν την ανάγκη εκτέλεσης εντολών \textlatin{Docker} με τα σωστά ορίσματα
στη σωστή σειρά. Εμείς μόνο πρέπει να καθορίσουμε τα απαραίτητα ορίσματα μία
φορά στο \texttt{\textlatin{docker-compose.yaml}} αρχείο.

Παρακάτω βλέπουμε ένα παράδειγμα αρχείου
\texttt{\textlatin{docker-compose.yaml}}. Σε παραγωγικό περιβάλλον τα
\textlatin{Docker} κοντέινερς πρέπει να έχουν πολλές ρυθμίσεις (π.χ.
μεταβλητές περιβάλλοντος, εκτεθειμένες θύρες και εξαρτήσεις από άλλα κοντέινερ).
Ο καθορισμός των πάντων σε ένα μόνο αρχείο απλοποιεί κατά πολύ τα πράγματα.


\texttt{\textlatin{version: "3.8"}}

\texttt{\textlatin{services:}}

\texttt{\textlatin{postgres:}}

\texttt{\textlatin{image: "postgres:latest"}}

\texttt{\textlatin{restart: "always"}}

\texttt{\textlatin{environment:}}

\texttt{\textlatin{cat: /tmp/test: No such file or directory}} \\

Παρόμοια λειτουργικότητα είναι επίσης ενσωματωμένη στο
\textlatin{Docker Engine}, που ονομάζεται \textlatin{Docker Stack}.
Χρησιμοποιεί επίσης το \texttt{\textlatin{docker-compose.yaml}}. Ορισμένες
λειτουργίες που υποστηρίζονται από το \texttt{\textlatin{docker-compose}} δεν
υποστηρίζονται από το \textlatin{Docker Stack} και αντίστροφα.


\subsection{\textlatin{Docker Hub}}

Οι εικόνες \textlatin{Docker} διανέμονται μέσω μητρώων (\textlatin{registries}).
Το μητρώο είναι ένας διακομιστής (που μπορεί να φιλοξενήσει ο οποιοσδήποτε),
που αποθηκεύει εικόνες Docker. Όταν ένας \textlatin{client} δεν έχει μια εικόνα
\textlatin{Docker} που χρειάζεται, μπορεί να επικοινωνήσει με ένα μητρώο για να
κάνει λήψη αυτής της εικόνας. Επειδή τα μητρώα είναι ένας εύκολος τρόπος
διανομής oι εικόνες \textlatin{Docker}, είναι ένα ενδιαφέρον
\textlatin{attack vector}.

Το πιο δημοφιλές (και προεπιλεγμένο) μητρώο είναι το \textlatin{Docker Hub}
\footnote{\textlatin{https://hub.docker.com/}}, το οποίο τρέχει
από την ίδια την εταιρεία \textlatin{Docker}. Οποιοσδήποτε μπορεί να
δημιουργήσει έναν λογαριασμό \textlatin{Docker Hub} και να αρχίσει να
δημιουργεί και να δημοσιεύει εικόνες που μπορεί να κατεβάσει οποιοσδήποτε.
