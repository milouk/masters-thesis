\chapter{Βασικές Έννοιες}
\label{basicConcepts}

Σε όλη αυτή τη διατριβή, θα δούμε πολλά παραδείγματα χρησιμοποιώντας
\textlatin{Unix Shell} εντολές. Θα αναφερθούμε επίσης στην επιστήμη της
πληροφορικής (που σχετίζεται με την ασφάλεια). Αυτό το κεφάλαιο θα παρουσιάσει
τη σημειογραφία και τις έννοιες που χρησιμοποιούνται.

\section{Εντολές \textlatin{Unix / Shell}}

Οι ακόλουθες συμβάσεις χρησιμοποιούνται για την αναπαράσταση των διαφορετικών
πλαισίων στα οποία οι διάφορες εντολές εκτελούνται. \\

• Εάν μια εντολή εκτελείται απευθείας σε ένα κεντρικό σύστημα, έχει το πρόθεμα
"(\textlatin{host})".

• Εάν μια εντολή εκτελείται μέσα σε ένα κοντέινερ, έχει το πρόθεμα
"(\textlatin{cont})".

• Εάν μια εντολή εκτελείται από μη προνομιούχο χρήστη, έχει το πρόθεμα από
"\$".

• Εάν μια εντολή εκτελείται από προνομιούχο χρήστη (δηλαδή 
\texttt{\textlatin{root}}), έχει το πρόθεμα με "\#".

• Η μεγάλη ή άσχετη έξοδος εντολών αντικαθίσταται από το “. . . ".

• Προκειμένου να βελτιωθεί η αναγνωσιμότητα, οι εντολές που εμφανίζονται
χρησιμοποιούν συντομευμένα ορίσματα εντολών (όπου είναι δυνατόν) και τιμές
ορίσματος σε εισαγωγικά.

\subsection{Ευπάθειες}

Το σύστημα \textlatin{Common Vulnerabilities and Exposures (CVE)} είναι μια
λίστα με δημόσια γνωστά σφάλματα που σχετίζονται με την ασφάλεια.

Σε κάθε ευπάθεια που εντοπίζεται δίνεται ένα αναγνωριστικό \textlatin{CVE}, το
οποίο φαίνεται όπως το \textlatin{CVE–2019–0000}. Ο πρώτος αριθμός
αντιπροσωπεύει το έτος κατά το οποίο το διαπιστώνεται ευπάθεια. Ο δεύτερος
αριθμός είναι ένας αυθαίρετος αριθμός τουλάχιστον τεσσάρων ψηφίων. Στην πράξη ο
αυθαίρετος αριθμός εφαρμόζεται ως μετρητής (π.χ. το πρώτο \textlatin{CVE} ενός
έτους παίρνει \textlatin{CVE–YYYY–0001} και το δεύτερο παίρνει
\textlatin{CVE–YYYY–0002)}.

Το σύστημα συντηρείται από την εταιρεία \textlatin{MITER Corporation}
\footnote{\textlatin{https://cve.mitre.org/}}. Οργανισμοί
που επιτρέπεται να δίνουν νέα αναγνωριστικά \textlatin{CVE} ονομάζονται
\textlatin{CVE Numbering Authorities (CNA} για συντομία). Μπορούμε να
διαβάσουμε και να αναζητήσουμε την πλήρη λίστα στον ιστότοπο της
\textlatin{MITRE}, στην Εθνική Βάση Δεδομένων Ευπάθειας των Ηνωμένων Πολιτειών
\footnote{\textlatin{https://nvd.nist.gov/}} \textlatin{(NVD)} και άλλους
ιστότοπους όπως το \textlatin{CVEDetails}
\footnote{\textlatin{https://www.cvedetails.com/}}.

Η σοβαρότητα (επίπτωση και πιθανότητα εκμετάλλευσης) ενός \textlatin{CVE}
καθορίζεται από τη βαθμολογία \textlatin{Common Vulnerability Scoring System
(CVSS}). Οι βαθμολογίες \textlatin{CVSS} κάθε \textlatin{CVE} μπορούν να
βρεθούν στο \textlatin{National Vulnerability Database}.


\subsection{Δοκιμές διείσδυσης}

Η δοκιμή διείσδυσης (\textlatin{pentesting}) είναι μια προσομοιωμένη επίθεση
για τη δοκιμή του ασφάλεια συστημάτων και εφαρμογών. Ο στόχος μιας δοκιμής
διείσδυσης είναι να βρούμε τα αδύνατα σημεία ενός συστήματος για να
μπορέσουμε να τα διορθώσουμε και να τα ασφαλίσουμε.

Εταιρείες, όπως η \textlatin{Secura}, εκτελούν δοκιμές διείσδυσης για πελάτες.
Το αποτέλεσμα μιας τέτοιας δοκιμής διείσδυσης είναι μια έκθεση που περιγράφει
λεπτομερώς τις αδυναμίες του συστήματος και των εφαρμογών του πελάτη. Αυτό
δίνει στον πελάτη πληροφορίες για το πώς πρέπει να κινηθεί προκειμένου να
ασφαλίσει τα συστήματά του και τις αδυναμίες που μπορεί να στοχεύσει ένας
εισβολέας. 

Μια τυπική δοκιμή διείσδυσης εκτελείται σε φάσεις (που ονομάζεται
\textlatin{kill chain}): \\

1. Αναγνώριση: Εδώ συλλέγουμε δεδομένα σχετικά με το σύστημα ή την
εφαρμογή-στόχο. Αυτά μπορούν να συγκεντρωθούν ενεργά (δηλαδή με αλληλεπίδραση
με τον στόχο) ή παθητικά (δηλαδή χωρίς αλληλεπίδραση με τον στόχο).

2. Εκμετάλλευση: Τα συγκεντρωμένα δεδομένα χρησιμοποιούνται για τον εντοπισμό
αδύναμων σημείων και τρωτών σημείων. Αυτά δέχονται επίθεση και εκμεταλλεύονται
για να κερδίσουν (μη προνομιούχα) πρόσβαση.

3. Μετά την εκμετάλλευση: Μετά από επιτυχή εκμετάλλευση και απόκτηση ενός
πατήματος, προσπαθούμε να πετύχουμε ένα πιο μόνιμο πάτημα.

4. Διήθηση: Από τη στιγμή που έχει εδραιωθεί μια βάση, μπορούμε να ανακτήσουμε
ευαίσθητα δεδομένα από το σύστημα.

5. Καθαρισμός: Μόλις η επίθεση ολοκληρωθεί με επιτυχία, όλα τα ίχνη της επίθεσης
πρέπει να αφαιρεθούν.

Υπάρχουν πολλά είδη αξιολογήσεων. Οι περισσότερες δοκιμές διαφέρουν ως προς το
ποιες πληροφορίες σχετικά με το σύστημα λαμβάνει ο αξιολογητής από τον διαχειριστή του συστήματος ή
ιδιοκτήτη πριν από την έναρξη της αξιολόγησης ή τι είδους συστήματα ή εφαρμογές
δοκιμάζονται. Ακολουθούν ορισμένες κοινές εκτιμήσεις που εταιρείες όπως η
\textlatin{Secura}, εκτελεί:

• \textlatin{Black Box Application / Infrastructure Test}: Ο αξιολογητής δεν
έχει λάβει οποιεσδήποτε πληροφορίες σχετικά με το σύστημα που βρίσκονται στο
πεδίο αξιολόγησης.

• \textlatin{Grey Box Application / Infrastructure Test:} Ο αξιολογητής
λαμβάνει ορισμένες πληροφορίες (π.χ. \textlatin{credentials}) σχετικά με τα
συστήματα στο πεδίο αξιολόγησης.

• \textlatin{Crystal Box Application / Infrastructure Test:} Ο αξιολογητής έχει
γνώση όλων των  διαθέσιμων πληροφοριών σχετικά με το σύστημα και τις εσωτερικές
λειτουργίες του. Επιπλέον, οι αρχιτέκτονες του συστήματος μπορούν να ερωτηθούν.
Οι αξιολογήσεις τύπου \textlatin{Crystal Box} πολλές φορές ονομάζονται και
\textlatin{White Box.}

• \textlatin{Design Review}: Μια αξιολόγηση όπου η αρχιτεκτονική, τεκμηρίωση
και οι ρυθμίσεις παραμέτρων όλων των συστημάτων σε ένα περιβάλλον
επανεξετάζονται. Δεν πραγματοποιούνται πραγματικές δοκιμές κατά τη διάρκεια
μιας αναθεώρησης σχεδιασμού.

• \textlatin{Internal Penetration Test:} Αξιολόγηση του εσωτερικού δικτύου ενός
\textlatin{client}. Τις περισσότερες φορές, η αξιολόγηση έχει έναν ξεκάθαρο
στόχο (π.χ. εύρεση ορισμένων ευαίσθητων πληροφοριών).

• \textlatin{Social Engineering}: Αξιολόγηση της ασφάλειας των ατόμων που
αλληλεπιδρούν με ένα σύστημα (π.χ. υπάλληλοι μιας εταιρείας). Για παράδειγμα,
στέλνοντας μηνύματα ηλεκτρονικού ψαρέματος ή προσπαθώντας να αποκτήσπυμε φυσική
πρόσβαση σε ένα κτίριο υποδυόμενοι έναν υπάλληλο.

• \textlatin{Code Reviews}: Έλεγχος του πηγαίου κώδικα μιας εφαρμογής.
