\chapter{Γενικά Συμπεράσματα}
\label{conclusions}

Τα κοντέινερ βοηθούν τους ανθρώπους να δημιουργούν πιο ασφαλή περιβάλλοντα,
επειδή απομονώνουν το λογισμικό. Ωστόσο, η χρήση κοντέινερ αυξάνει επίσης την
επιφάνεια επίθεσης καθώς και τους κινδύνους, επειδή το λογισμικό
\textlatin{containerization} προσθέτει επίσης επιπλέον επίπεδα αφαίρεσης και
πολυπλοκότητας. Αυτό δημιουργεί προκλήσεις τόσο για τους επιτιθέμενους όσο και
για τους αμυνόμενους ενός συστήματος \textlatin{Docker}. Θα εξετάσουμε τα
ευρήματα αυτής της έρευνας και από τις δύο πλευρές.

\section{Συμπεράσματα απο την πλευρά του επιτιθέμενου}

Κατά την εκτέλεση μιας δοκιμής διείσδυσης, είναι σημαντικό να γνωρίζουμε τα
παρακάτω σημεία. \\

• \textbf{Να γνωρίζουμε τα μοντέλα επιτιθέμενου που περιγράφονται στο κεφάλαιο
3.}

Όπως είδαμε στο κεφάλαιο 3, υπάρχουν δύο μοντέλα επιτιθέμενων: διαφυγές
κοντέινερ που επικεντρώνονται στην απόδραση από την απομόνωση ενός κοντέινερ και
επιθέσεις \textlatin{Docker Daemon} που επικεντρώνονται στη χρήση μιας
εγκατάστασης \textlatin{Docker} σε ένα \textlatin{host} για να αποκτήσει
κανείς πρόσβαση σε προνομιακά δεδομένα. Είναι σημαντικό να γνωρίζουμε κατά τη
διάρκεια μιας δοκιμής διείσδυσης ποια από τις 2 επιθέσεις είναι σχετική.

• \textbf{Οι εσφαλμένες διαμορφώσεις είναι πιο ενδιαφέρουσες από ό,τι
σφάλματα λογισμικού που σχετίζονται με την ασφάλεια.}

Εξετάσαμε πολλά τρωτά σημεία στο κεφάλαιο 4. Εξετάσαμε και εσφαλμένες ρυθμίσεις
αλλά και σφάλματα. Τόσο οι λανθασμένες ρυθμίσεις όσο και τα σφάλματα
αποτελούν κίνδυνο. Ωστόσο, οι εσφαλμένες διαμορφώσεις είναι πιο ενδιαφέρουσες
για έναν επιτιθέμενο, γιατί είναι πιο δύσκολο να διορθωθούν. Τα σφάλματα
λογισμικού επιδιορθώνονται εύκολα χρησιμοποιώντας την πιο πρόσφατη έκδοση του
\textlatin{Docker}, ενώ οι εσφαλμένες διαμορφώσεις απαιτούν αλλαγή του τρόπου
χρήσης του \textlatin{Docker}.

• \textbf{Μην βασιζόμαστε αποκλειστικά σε λίστες οδηγιών.}

Οι λίστες οδηγιών (π.χ. το \textlatin{CIS Docker Benchmark}) είναι μια καλή αρχή
για τον εντοπισμό πιθανών ευάλωτων τμημάτων ενός συστήματος. Ωστόσο, όπως
είδαμε με το \textlatin{CIS Docker Benchmark}, δεν είναι εξαντλητικές οι λίστες
οδηγιών.

• \textbf{Μην βασιζόμαστε αποκλειστικά σε εργαλεία για την αυτοματοποίηση των
αξιολογήσεων ασφαλείας.}

Τα εργαλεία (π.χ. που εξετάσαμε στην ενότητα 5.2) βοηθούν στην αυτοματοποίηση
της δοκιμής διείσδυσης. Είναι χρήσιμα γιατί εξοικονομούν χρόνο και εξετάζουν
συστηματικά τα συστήματα στόχου. Ωστόσο, υπολείπονται όσον αφορά τον εντοπισμό
πιο περίπλοκων τρωτών σημείων και την κάλυψη μεγαλύτερων τμημάτων ενός
συστήματος. Τα περισσότερα εργαλεία έχουν σχεδιαστεί για να σαρώνουν ή να
εκμεταλλεύονται μόνο μία συγκεκριμένη ευπάθεια. Μια λεπτομερής, μη αυτόματη
αξιολόγηση θα πάρει περισσότερο χρόνο, αλλά θα αποκαλύψει επίσης περισσότερα
τρωτά σημεία και αδύναμα σημεία.

• \textbf{Χρησιμοποιούμε τη λίστα ελέγχου που περιλμβάνεται στο κεφάλαιο 6.}

Η λίστα ελέγχου στο κεφάλαιο 6 παρέχει σε έναν επιτιθέμενο τρία σετ με
ενδιαφέρουσες ερωτήσεις για την αναγνώριση και την εκμετάλλευση ενός
συστήματος που χρησιμοποιεί \textlatin{Docker}. Η πρώτη λίστα έχει σκοπό να
ελέγξει εάν ένας εισβολέας τρέχει μέσα σε ένα κοντέινερ ή σε έναν κεντρικό
υπολογιστή. Η δεύτερη και η τρίτη λίστα προορίζονται για τη συλλογή δεδομένων
και τον εντοπισμό τρωτών σημείων μέσα σε κοντέινερ και σε \textlatin{hosts},
αντίστοιχα.


\section{Συμπεράσματα απο την πλευρά του αμυνόμενου}

Αν και αυτή η έρευνα εστιάζει σε μια προοπτική επίθεσης κατά του
\textlatin{Docker}, μπορεί να χρησιμοποιηθεί για τη οχύρωση και την ασφάλιση
ενός συστήματος που χρησιμοποιεί \textlatin{Docker}. Κατά το σχεδιασμό ή τη
συντήρηση ενός συστήματος που χρησιμοποιεί \textlatin{Docker}, είναι σημαντικό να
λάβουμε υπόψη τα ακόλουθα σημεία. \\

• \textbf{Η χρήση του \textlatin{Docker} προσθέτει ένα επίπεδο απομόνωσης στο
λογισμικό μας.}

Το \textlatin{Docker}, όπως όλα τα λογισμικά κοντέινερ, προσθέτει ένα στρώμα
απομόνωσης. Αυτό προσθέτει ασφάλεια, επειδή το λογισμικό είναι απομονωμένο από
το κεντρικό σύστημα.

Ωστόσο, αυτό προσθέτει επίσης ένα στρώμα αφαίρεσης στο σύστημα. Αντί να τρέχει
ένα λογισμικό απευθείας σε έναν κεντρικό υπολογιστή, εκτελείται μέσα σε ένα
κοντέινερ σε ένα κεντρικό υπολογιστή. Αυτό το στρώμα αφαίρεσης αυξάνει την
επιφάνεια επίθεσης του συστήματος.

• \textbf{Να χρησιμοποιούμε πάντα την πιο πρόσφατη έκδοση του
\textlatin{Docker.}}

Όπως είδαμε στο κεφάλαιο 4, υπάρχουν πολλά τρωτά σημεία που αποτελούν κίνδυνο
σε συστήματα που χρησιμοποιούν \textlatin{Docker}. Είναι δυνατό να μειωθεί
σημαντικά ο κίνδυνος ενός από τους τύπους ευπάθειας που εξετάσαμε. Τα σφάλματα
λογισμικού είναι εύκολο να διορθωθούν (από τον χρήστη) χρησιμοποιώντας πάντα την
πιο πρόσφατη έκδοση του \textlatin{Docker}.

• \textbf{Η λίστα ελέγχου στο κεφάλαιο 6 θα μας βοηθήσει να δούμε ένα σύστημα
όπως ένας επιτιθέμενος.}

Αν γνωρίζουμε πώς βλέπει το σύστημά μας ένας εισβολέας, μπορούμε πιο εύκολα να
προσδιορίσουμε τα μέρη που αυτός θα στόχευε. Η λίστα ελέγχου των ερωτήσεων στο
κεφάλαιο 6 θα μας βοηθήσει να δούμε ένα σύστημα από την πλευρά του επιτιθέμενου.

