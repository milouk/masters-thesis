\section{Εισαγωγή}
\label{introduction}


Στόχος της παρούσας έρευνας είναι να αποτελέσει έναν οδηγό και να παράσχει
μια μεθοδολογία σε \textlatin{penetration testers}, την οποία θα χρησιμοποιούν
όποτε καλούνται να αξιολογήσουν την ασφάλεια ενός συστήματος που χρησιμοποιεί
\textlatin{Docker}. Με άλλα λόγια, η εργασία αυτή αποσκοπεί στο να βοηθήσει τους
αξιολογητές να συντάξουν αναφορές και καλύτερες προτάσεις στους
εργοδότες τους.\mbox{} \\

Αρχικά, θα αναφέρουμε κάποιες γενικές
πληροφορίες σχετικά με το λογισμικό \textlatin{containerization} και το
\textlatin{Docker} (ενότητα 1), καθώς και τις απαραίτητες έννοιες (ενότητα 2).
Στη συνέχεια, θα εξετάσουμε λεπτομερέστερα
τα μοντέλα επιθέσεων (ενότητα 3) που πρέπει να έχουμε υπόψιν μας όταν
αναφερόμαστε σε \textlatin{containers}.
Στην ενότητα 4 θα αναφερθούμε σε θέματα ευπαθειών, δηλαδή τόσο στις εσφαλμένες
ρυθμίσεις παραμέτρων όσο και στα σχετικα με την ασφάλεια
σφάλματα λογισμικού, που υπάρχουν στο \textlatin{Docker}. Θα τα καταγραφούν
με γνώμονα έναν οδηγό βέλτιστων πρακτικών που
χρησιμοποιείται από εταιρείες, και ονομάζεται
\textlatin{CIS Docker Benchmark}. 
Θα συζητήσουμε πώς μπορούν τα τρωτά σημεία να εντοπιστούν και να προσδιοριστούν
κατά τη διάρκεια μια αξιολόγησης ασφαλείας (ενότητα 5). Σημαντικός στόχος της
έρευνας αυτής είναι να συμβάλει στη καταγραφή μιας σειράς βημάτων, την οποία
οι ερευνητές ασφαλείας πρέπει να συμβουλέυονται πριν κάνουν μια αξιολόγηση σε
ένα σύστημα που χρησιμοποιεί \textlatin{Docker}. Τέλος, θα εξετάσουμε το πεδίο
εφαρμογής, αλλά και άλλες ενδιαφέρουσες ιδέες για την επέκταση αυτής της
έρευνας (ενότητα 7) και θα εξάγουμε συμπεράσματα που αφορύνν τόσο την πλευρά του
επιτιθέμενου όσο και του αμυνόμενου  (ενότητα 8).\mbox{} \\

Θα επικεντρωθούμε στο \textlatin{Linux}, επειδή το \textlatin{Docker} έχει αναπτυχθεί για
\textlatin{Linux} (αν και υπάρχουν και εκδόσεις του \textlatin{Docker} για συστήματα πέρα από \textlatin{Linux} \footnote{Το \textlatin{Docker} σε συστήμαta μη-\textlatin{Linux} τρέχει μέσα σε εικονικές μηχανές \textlatin{Linux}}).
Σε όλη αυτή τη διατριβή θα μελετήσουμε πρακτικά παραδείγματα, οπότε μια καλή
κατανόηση του \textlatin{Linux} είναι χρήσιμη.
