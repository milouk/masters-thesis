\chapter{Μελλοντική Έρευνα}
\label{futureWork}

Αυτή η διατριβή εξετάζει τρόπους για να κάνει κανείς δοκιμές διείσδυσης σε
συστήματα \textlatin{Docker}. Κατά τη διάρκεια της έρευνας και της συγγραφής,
βρήκαμε μερικά ενδιαφέροντα θέματα που πέρα από το πεδίο εφαρμογής αυτής της
διατριβής.

\section{Λογισμικό Ενορχήστρωσης}

Στη σύγχρονη ανάπτυξη λογισμικού, το \textlatin{containerization} είναι μόνο
μέρος του παζλ. Οι μεγάλες εταιρείες διαχειρίζονται πολλά διαφορετικά λογισμικά
και κάθε \textlatin{instance} πρέπει να υποστηρίζει πολλές συνδέσεις και πολλή
υπολογιστική ισχύ. Αυτό σημαίνει ότι για πολλές εφαρμογές, απαιτούνται πολλά
κοντέινερ. Για τη διαχείριση όλων αυτών κοντέινερ υπάρχει λογισμικό
ενορχήστρωσης. Τα πιο γνωστά λογισμικά είναι το Kubernetes 
\footnote{\textlatin{https://kubernetes.io/}} και το Docker
Swarm \footnote{\textlatin{https://docs.docker.com/engine/swarm/}}.

Θα ήταν ενδιαφέρον να συνεχίσουμε αυτή την έρευνα εξετάζοντας πώς
θα μπορούσε κανείς να εκτελέσει δοκιμές διείσδυσης σε λογισμικό ενορχήστρωσης
και πώς το λογισμικό ενορχήστρωσης επηρεάζει την ασφάλεια των συστημάτων. Αυτό
θα μπορούσε να επεκταθεί σε συγκεκριμένα τη χρήση του \textlatin{Docker} σε
παρόχους υπολογιστικού νέφους.

\section{\textlatin{Docker} σε \textlatin{Non-Linux} λειτουργικά συστήματα}

Αυτή η πτυχιακή εργασία εξετάζει το \textlatin{Docker} στο \textlatin{Linux},
επειδή το \textlatin{Docker} χρησιμοποιεί χαρακτηριστικά που υπάρχουν μόνο στον
πυρήνα του \textlatin{Linux}. Ωστόσο, είναι επίσης δυνατή η εκτέλεση του
\textlatin{Docker} σε λειτουργικά συστήματα που δεν είναι \textlatin{Linux}
(π.χ. \textlatin{Windows} και \textlatin{MacOS}). Τρέχοντας μια εικονική μηχανή
\textlatin{Linux} που τρέχει το \textlatin{Docker}.

Αυτή η εικονική μηχανή είναι ένα επιπλέον στρώμα αφαίρεσης που είναι και από
μόνο του ένα \textlatin{attack surface} και προσθέτει περισσότερο κίνδυνο.

Ορισμένα από τα τρωτά σημεία και τις εσφαλμένες διαμορφώσεις που περιγράφονται
σε αυτή η διατριβή μπορεί επίσης να είναι σχετικές με λειτουργικά συστήματα που
δεν είναι \textlatin{Linux}.


Υπάρχουν επίσης τρωτά σημεία που σχετίζονται με συγκεκριμένα λειτουργικά
συστήματα. Για παράδειγμα, τα \textlatin{CVE–2019–15752} και
\textlatin{CVE–2018–15514} αφορούν μόνο τα \textlatin{Windows}.

Θα ήταν ενδιαφέρον (και σχετικό με τις δοκιμές διείσδυσης) να συνεχίσουμε
αυτή η έρευνα εξετάζοντας συγκεκριμένα το \textlatin{Docker} σε λειτουργικά
συστήματα εκτός \textlatin{Linux}.

\section{Σύγκριση \textlatin{Virtualization} και \textlatin{Containerization}}

Αυτή η διατριβή εξετάζει την ασφάλεια του \textlatin{Docker}. Όπως αναφέρθηκε
στο παρασκήνιο, το \textlatin{Virtualization} είναι ένας άλλος τρόπος για να
επιτευχθεί η απομόνωση. Πολλά έχουν γραφτεί για τη σύγκριση
\textlatin{Virtualization} και \textlatin{Containerization}
\cite{Virtualization-vs-Containerization-to-Support-PaaS}
\cite{Hypervisor-vs-Lightweight-Virtualization}
\cite{Updated-Performance-Comparison-Virtual-Machines-Containers}. Ωστόσο, θα
ήταν ενδιαφέρον να συγκρίνουμε συγκεκριμένα την απομόνωση και την ασφάλεια που
το \textlatin{Virtualization} προσφέρει στην απομόνωση και την ασφάλεια σε σχέση
με αυτά που προσφέρει το \textlatin{Containerization}.


\section{\textlatin{Docker Man-in-the-Middle}}

Στην ενότητα 4.1.7 εξετάσαμε την εκτέλεση μιας επίθεσης
\textlatin{man-in-the-middle} χρησιμοποιώντας \textlatin{ARP Spoofing}. Θα ήταν
ενδιαφέρον να δούμε πιο σύνθετες επιθέσεις \textlatin{man-in-the-middle}. Για
παράδειγμα, καταγραφή όλης της κίνησης από και προς έναν διακομιστή ιστού η
οποία εκτελείται σε \textlatin{Docker} ή τροποποιώντας την κίνηση.

\section{\textlatin{Ένα \textlatin{Docker Specific} εργαλεία διείσδυσης}}

Στην ενότητα 5.2 συζητάμε πολλά εργαλεία που αυτοματοποιούν μέρος των
αξιολογήσεων ασφαλείας. Ωστόσο, βλέπουμε ότι ορισμένα εργαλεία δεν είναι
ενδιαφέροντα από την πλευρά του επιτιθέμενου και τα περισσότερα εργαλεία
επικεντρώνονται σε συγκεκριμένα τρωτά σημεία. Θα ήταν ενδιαφέρον να αναπτυχθεί
ένα νέο εργαλείο ή να επεκταθεί ένα υπάρχον εργαλείο που εστιάζει στο
πλήρες φάσμα εκμεταλλεύσεων και τρωτών σημείων ενός ή περισσότερων μοντέλων
επιτθέμενων (κεφάλαιο 3) και όχι μόνο για συγκεκριμένα τρωτά σημεία. Ένα καλό
σημείο εκκίνησης για αυτό θα ήταν με την αυτοματοποίηση της απάντησης στις
ερωτήσεις που τέθηκαν στο κεφάλαιο 6.
