\section{Περίληψη}
\label{Abstract}

Οι \textlatin{penetration testers} συναντούν πολλά διαφορετικά συστήματα
κατά τη διάρκεια των αξιολογήσεων που διενεργούν. Πολλά από αυτά τα συστήματα
συχνά χρησιμοποιούν την τεχνολογία \textlatin{Docker},
λόγω της δημοτικότητας του τα τελευταία χρόνια. Η έρευνα αυτή
εξετάζει το \textlatin{Docker} από την άποψη της ασφάλειας και εξετάζει πώς
ένας \textlatin{penetration tester} θα πρέπει να αξιολογεί την ασφάλεια των
συστημάτων που χρησιμοποιούν \textlatin{Docker}.\mbox{} \\

Παρουσιάζουμε δύο μοντέλα επιτιθέμενου: αποδράσεις \textlatin{docker container}
και επιθέσεις \textlatin{Docker daemon}. Αυτά τα μοντέλα επιθέσεων είναι
γενικεύσεις επιθέσεων από συγκεκριμένη προοπτική. Οι αποδράσεις
\textlatin{docker container} αφορούν ένα σενάριο κατά το οποίο, ο
εισβολέας εκμεταλλεύεται μια διαδικασία που εκτελείται μέσα
σε ένα \textlatin{container}. Συζητάμε επίσης τις επιθέσεις
\textlatin{Docker Daemon}, ένα μοντέλο επίθεσης όπου ο εισβολέας εκμεταλλεύεται
μια διαδικασία που εκτελείται σε έναν κεντρικό υπολογιστή που έχει
εγκατεστημένο το \textlatin{Docker}.\mbox{} \\

Σε αυτή την έρευνα θα εξετάσουμε γνωστά τρωτά σημεία στο \textlatin{Docker}.
Συγκεκριμένα, εξετάζουμε εσφαλμένες ρυθμίσεις παραμέτρων 
(\textlatin{misconfiguations}) και σφάλματα λογισμικού που
σχετίζονται με την ασφάλεια. Παρέχουμε πρακτικές και παραδείγματα για το πώς να
εκμεταλλευτεί κανείς τις εσφαλμένες ρυθμίσεις παραμέτρων και τι
αντίκτυπος θα μπορούσε να προκύψει. Θεωρούμε ότι οι λανθασμένες ρυθμίσεις
είναι πιο ενδιαφέρουσες από τα σφάλματα λογισμικού, επειδή τα σφάλματα
λογισμικού είναι πολύ πιο εύκολο να διορθωθούν.

Χαρτογραφούμε αυτά τα τρωτά σημεία στο σχετικό σημείο αναφοράς του
\textlatin{CIS Docker} (ένας οδηγός καλής πρακτικής σχετικά με την χρήση του
\textlatin{Docker}). Παρατηρούμε πως δεν καλύπτονται όλες οι λανθασμένες
ρυθμίσεις από το \textlatin{CIS Docker Benchmark}.\mbox{} \\

Επιπλέον, περιγράφουμε τον τρόπο αναγνώρισης του σχετικού μοντέλου
επιτιθέμενου κατά τη διάρκεια του \textlatin{penetration testing}.
Μετά από αυτό περιγράφουμε τον τρόπο της με τον οποίο εκτελούμε την αναγνώριση
και εντοπισμό τρωτών σημείων σε συστήματα που χρησιμοποιούν
\textlatin{docker}.\mbox{} \\

Μελετάμε εργαλεία που θα μπορούσαν να αυτοματοποιήσουν τον εντοπισμό
και την εκμετάλλευση των τρωτών σημείων. Εμείς, ωστόσο, διαπιστώνουμε
ότι κανένα εργαλείο δεν αυτοματοποιεί πλήρως και αντικαθιστά τις μη
αυτόματες δηλαδή της χειροκίνητες αξιολογήσεις.\mbox{} \\

Ολοκληρώνουμε παρουσιάζοντας μια λίστα βημάτων που συνοψίζουν την έρευνα
ως ερωτήσεις που θα πρέπει να κάνει ένας \textlatin{penetration tester} σχετικά
με ένα σύστημα που χρησιμοποιεί \textlatin{Docker} κατά τη διάρκεια
μιας αξιολόγησης. Για κάθε ερώτηση, δίνεται μια απλή απάντησης  και γίνεται
αναφορά στη σχετική ενότητα της παρούσας διατριβής.
Αυτή η λίστα βημάτων βοηθά τους αξιολογητές ασφαλείας  να δοκιμάσουν την
ασφάλεια των συστημάτων που χρησιμοποιούν \textlatin{Docker}.
