\chapter{Γνωστά \textlatin{Docker CVEs}}
\label{knownDockerCves}

Επειδή το \textlatin{Docker} είναι τόσο δημοφιλές, πολλοί ερευνητές ασφαλείας
προσπαθούν βρουν και να τεκμηριώσουν διάφορες ευπάθειες. Σε αυτό το κεφάλαιο
συζητάμε τα τρωτά σημεία υψηλού αντίκτυπου που είναι χρήσιμα κατά τη διάρκεια
μιας δοκιμής διείσδυσης. Αυτά χωρίζονται σε εσφαλμένες ρυθμίσεις
(\textlatin{misconfiguration} και σφάλματα (\textlatin{bugs}).

Σφάλματα λογισμικού και εσφαλμένες ρυθμίσεις παραμέτρων μπορεί να είναι και τα
δύο προβλήματα ασφαλείας, αλλά διαφέρουν στο ποιος έκανε το λάθος.

Ένα \texttt{\textlatin{bug}} είναι ένα πρόβλημα σε ένα ίδιο το πρόγραμμα. Για
παράδειγμα, υπερχείλιση ενός \textlatin{buffer} είναι ένα σφάλμα. Το πρόβλημα
έγκειται αποκλειστικά στο ίδιο το πρόγραμμα. Για να διορθωθεί, πρέπει να
αλλάξει ο κώδικας του προγραμμάτος.

Οι εσφαλμένες ρυθμίσεις, από την άλλη πλευρά, είναι προβλήματα ασφάλειας που
προκύπτουν από την λάθος χρήση ενός προγράμματος. Το πρόγραμμα δεν έχει
ρυθμιστεί σωστά και αυτό δημιουργεί μια κατάσταση που μπορεί να είναι
εκμεταλλεύσιμη για έναν επιτιθέμενο. Μια δημόσια διαθέσιμη κονσόλα εντοπισμού
σφαλμάτων (\textlatin{debugging console}) σε έναν ιστότοπο 
\footnote{Με αυτό τον τρόπο η εταιρεία \textlatin{Patreon} χακαρίστηκε πριν
κάποια χρόνια. Βλ.\textlatin{https://labs.detectify.com/2015/10/02/how-patreon-got-hacked-publicly-exposed-werkzeug-debugger/}}
ή ένα δημοσίως αναγνώσιμο αρχείο που περιέχει κωδικούς πρόσβασης είναι
παραδείγματα εσφαλμένων ρυθμίσεων. Για να διορθώσει μια εσφαλμένη
ρύθμιση, ο χρήστης θα πρέπει να αλλάξει τη ρύθμιση του προγράμματος ή την
υποδομή του. Οι προγραμματιστές του προγράμματος μπορούν να προτείνουν μόνο
τη σωστή διαμόρφωση που πρέπει να εφαρμόσουν οι χρήστες.

Δεν αποτελούν πλήρη παραδείγματα επιθέσεων όλα τα τρωτά σημεία που καλύπτονται
σε αυτό το κεφάλαιο. Τα περισσότερα είναι χρήσιμα ως μέρος μιας επίθεσης όταν
χρησιμοποιούνται σε συνδυασμό με άλλα τρωτά σημεία. Για παράδειγμα,
παρακάμπτοντας έναν μηχανισμό προστασίας. Ωστόσο, ορισμένα σοβαρά σφάλματα
είναι επικίνδυνα ακόμη και όταν χρησιμοποιούνται από μόνα τους.
Για παράδειγμα, η κακόβουλη χρήση του \textlatin{Docker socket} που καλύπτεται
στην επόμενη ενότητα και το \textlatin{CVE-2019-16884} είναι διαφυγές κοντέινερ.

Επειδή υπάρχουν πολλοί ερευνητές ασφάλειας που αναζητούν σφάλματα στο λογισμικό
\textlatin{containerization}, η ενότητα 4.2 πιθανότατα θα καταστεί ξεπερασμένη
γρήγορα και για αυτό το λόγο δεν θα πρέπει να λαμβάνεται σοβαρά.

Όλοι οι κίνδυνοι αυτών των σφαλμάτων μπορούν να αποφευχθούν χρησιμοποιώντας
την πιο πρόσφατη έκδοση των εικόνων \textlatin{Docker} και του
\textlatin{Docker}.

Λόγω των παραπάνω λόγων, θα επικεντρωθούμε περισσότερο στις εσφαλμένες
ρυθμίσεις σε αυτό το κεφάλαιο και τα επόμενα κεφάλαια.

Στο κεφάλαιο 5 θα δούμε πώς μπορούν να εντοπιστούν αυτά τα τρωτά σημεία
κατά τη διάρκεια μιας δοκιμής διείσδυσης. Στο κεφάλαιο 6 θα συνδυάσουμε τις
πληροφορίες από αυτό το κεφάλαιο και το κεφάλαιο 5 σε μια λίστα ελέγχου.

\section{\textlatin{Misconfigurations}}

Σε αυτήν την ενότητα, θα ρίξουμε μια ματιά στις εσφαλμένες ρυθμίσεις του
\textlatin{Docker} και του αντίκτυπο που μπορεί να έχουν αυτές οι εσφαλμένες
ρυθμίσεις. Για κάθε εσφαλμένη ρύθμιση, θα μελετήσουμε πρακτικά
παραδείγματα καθώς επίσης και τον αντίκτυπο.


Οι δύο πρώτες εσφαλμένες διαμορφώσεις που θα εξετάσουμε σχετίζονται με
επιθέσεις που εκτελούνται σε έναν κεντρικό υπολογιστή και πρόκειται για
επιθέσεις \textlatin{Docker daemon}. Οι άλλες εσφαλμένες ρυθμίσεις σχετίζονται
με επιθέσεις που εκτελούνται μέσα από ένα κοντέινερ όπως διαφυγή κοντέινερ.

Αντιστοιχίζουμε κάθε εσφαλμένη διαμόρφωση στις σχετικές κατευθυντήριες γραμμές
αναφοράς του \textlatin{CIS Docker Benchmark} (εάν υπάρχουν). Θα δούμε ότι το
\textlatin{CIS Docker Benchmark} δεν καλύπτει όλες τις εσφαλμένες
ρυθμίσεις.

\subsection{\textlatin{Docker Permissions}}

Μια συνηθισμένη (και διαβόητη) εσφαλμένη διαμόρφωση είναι η παροχή πρόσβασης
στο \textlatin{Docker} σε μη προνομιούχους χρήστες, το οποίο τους επιτρέπει να
δημιουργούν, να ξεκινούν και άλλοτε να αλληλεπιδρούν με 
με κοντέινερ \textlatin{Docker} (μέσω του \textlatin{Docker Daemon}). Αυτό
είναι επικίνδυνο επειδή επιτρέπει στους μη προνομιούχους χρήστες να έχουν
πρόσβαση σε όλα τα αρχεία ως \texttt{\textlatin{root}}. Η τεκμηρίωση του
\textlatin{Docker} λέει:
\footnote{\textlatin{https://docs.docker.com/engine/security/security/}}.

\emph{Πρώτα απόλα μόνο σε έμπιστους χρήστες πρέπει να επιστρέπεται να έχουν
τον έλεγχο του \textlatin{Docker Daemon}. Αυτό είναι άμεση συνέπεια από κάποια
από τα πιο ισχυρά χαρακτηριστικά του \textlatin{Docker}. Συγκεκριμένα το
\textlatin{Docker}, επιτρέπει σε κάποιον να μοιραστεί έναν κατάλογο μεταξύ του
υπολογιστή όπου τρέχει το \textlatin{Docker} και του φιλοξενούμενου κοντέινερ
και το επιτρέπει αυτό χωρίς να περιορίσει τα δικαιώματα πρόσβασης του κοντέινερ.
Αυτό σημαίνει ότι μπορεί κανείς να ξεκινήσει από ένα κοντέινερ όπου ο κατάλογος
\textlatin{/host} είναι ο κατάλογος / στον κεντρικό υπολογιστή και το κοντέινερ
μπορεί να αλλάξει το σύστημα αρχείων του κεντρικού υπολογιστή χωρίς κανέναν
περιορισμό.}

Εν ολίγοις, επειδή ο \textlatin{Docker Daemon} εκτελείται ως
\texttt{\textlatin{root}}, εάν ένας χρήστης προσθέσει ένα κατάλογο ως
\textlatin{volume} σε ένα κοντέινερ, η πρόσβαση σε αυτό το αρχείο γίνεται ως
\texttt{\textlatin{root}}. Υπάρχουν μερικοί τρόποι πρόσβασης στο Docker για μη
προνομιούχους χρήστες. Σε αυτή την ενότητα θα τους ερευνήσουμε.

\subsubsection{\texttt{\textlatin{docker}} \textlatin{Group}}

Κάθε χρήστης στην ομάδα \texttt{\textlatin{docker}} επιτρέπεται να χρησιμοποιεί
το \textlatin{Docker}.
Αυτό επιτρέπει τη διαχείριση πρόσβασης στη χρήση του \textlatin{Docker}.
Μερικές φορές ένας διαχειριστής συστήματος δεν θέλει να κάνει σωστή διαχείριση
πρόσβασης και προσθέτει κάθε χρήστη στην ομάδα \texttt{\textlatin{docker}},
γιατί αυτό επιτρέπει την ομαλή λειτουργία των πάντων.
Αυτή η εσφαλμένη ρύθμιση, ωστόσο, επιτρέπει σε κάθε χρήστη να έχει πρόσβαση σε
κάθε αρχείο του συστήματος, όπως απεικονίζεται και παρακάτω.

Ας υποθέσουμε ότι θέλουμε ο κατακερματισμένος κωδικός πρόσβασης του
διαχειριστή σε ένα σύστημα όπου δεν έχουμε προνόμια \texttt{\textlatin{sudo}},
αλλά είμαστε μέλος της ομάδας \texttt{\textlatin{docker}}. \\
 
\texttt{\textlatin{(host)\$ sudo -v}}

\texttt{\textlatin{Sorry, user unpriv may not run sudo on host.}}

\texttt{\textlatin{(host)\$ groups | grep -o docker}}

\texttt{\textlatin{docker}}

\texttt{\textlatin{(host)\$ docker run -it --rm -v /:/host ubuntu:latest bash}}

\texttt{\textlatin{(cont)\# grep admin /host/etc/shadow}}

\texttt{\textlatin{admin:\$6\$VOSV5AVQ\$jHWxAVAUgl...:18142:0:99999:7:::}} \\


Στην παραπάνω λίστα ελέγχουμε πρώτα τα δικαιώματα μας. Δεν έχουμε
\texttt{\textlatin{sudo}} δικαιώματα, αλλά είμαστε μέλος της ομάδας
\texttt{\textlatin{docker}}. Αυτό μας επιτρέπει να δημιουργήσουμε ένα κοντέινερ
με / προσαρτημένο ως \textlatin{volume} και αποκτήσoυμε πρόσβαση σε οποιοδήποτε
αρχείο ως \texttt{\textlatin{root}}.
Αυτό περιλαμβάνει το αρχείο με κατακερματισμούς κωδικούς πρόσβασης χρήστη
(π.χ. \texttt{\textlatin{/etc/passwd}}).

Ένα πραγματικό παράδειγμα της επίδρασης των εσφαλμένων ρυθμισμένων διακαιωμάτων
\textlatin{Docker} συνέβη πριν από λίγα χρόνια με ένα από τα μαθήματα στο
\textlatin{Computing Science curriculum} (του \textlatin{Radboud}). Ένα
καθηγητής ήθελε να διδάξει μαθητές σχετικά με \textlatin{containerization} και
την ανάπτυξη σύγχρονου λογισμικού. Ο καθηγητής ζήτησε από το τμήμα πληροφορικής
να εγκαταστήσει το \textlatin{Docker} σε όλους τους φοιτητικούς υπολογιστές και
να προσθέσει όλους τους μαθητές του μαθήματος στην ομάδα
\texttt{\textlatin{docker}} (παρέχοντάς τους πλήρη δικαιώματα για να τρέξου το
\textlatin{Docker}). Αυτό έδωσε σε κάθε μαθητή το ισοδύναμο του
\texttt{\textlatin{root}} δικαιώματα σε κάθε σταθμό εργασίας. Αυτό ήταν ένα
πρόβλημα, επειδή επέτρεπε στους μαθητές να διαβάζουν ευαίσθητες πληροφορίες
(π.χ. ιδιωτικά κλειδιά και κατακερματισμένους κωδικούς πρόσβασης όλων των
χρηστών) και να κάνουν αλλαγές στο σύστημα.

Η ομάδα docker καλύπτεται από την κατευθυντήρια γραμμή 1.2.2 του
\textlatin{CIS Docker Benchmark} (Βεβαιωθείτε ότι επιτρέπεται μόνο σε
αξιόπιστους χρήστες να ελέγχουν τον \textlatin{Docker Daemon}).

\subsubsection{Δημόσια αναγνώσιμο και εγγράψιμο \textlatin{Docker Socket}}

Από προεπιλογή, μόνο ο \texttt{\textlatin{root}} και κάθε χρήστης στην ομάδα
\textlatin{docker} έχει πρόσβαση στο \textlatin{Docker}, επειδή έχουν πρόσβαση
ανάγνωσης και εγγραφής στο \textlatin{Docker Socket}. Ωστόσο, ορισμένοι
διαχειριστές ορίζουν τα δικαιώματα ανάγνωσης και εγγραφής για όλους τους χρήστες
(π.χ \texttt{666}), δίνοντας σε όλους τους χρήστες πρόσβαση στον
\textlatin{Docker Daemon}. \\

\texttt{\textlatin{(host)\$ groups | grep -o docker}}

\texttt{\textlatin{(host)\$ ls -l /var/run/docker.sock}}

\texttt{\textlatin{srw-rw-rw- 1 root docker 0 Dec 15 13:16 /var/run/docker.sock}}

\texttt{\textlatin{(host)\$ docker run -it --rm -v /:/host ubuntu:latest bash}}

\texttt{\textlatin{(cont)\# grep admin /host/etc/shadow}}

\texttt{\textlatin{admin:\$6\$VOSV5AVQ\$jHWxAVAUgl...:18142:0:99999:7:::}} \\

Στη παραπάνω λίστα, βλέπουμε ότι δεν είμαστε μέλος της ομάδας
\textlatin{Docker}, αλλά επειδή κάθε χρήστης έχει πρόσβαση ανάγνωσης και
εγγραφής (δηλαδή τα δικαιώματα "ανάγνωση" και "εγγραφή" έχουν οριστεί για
\textlatin{other}) στο \textlatin{Docker Socket} μπορούμε ακόμα να
χρησιμοποιήσουμε το \textlatin{Docker}.

Αυτό καλύπτεται από την κατευθυντήρια γραμμή 3.4 του
\textlatin{CIS Docker Benchmark} (Βεβαιωθείτε ότι τα δικαιώματα του αρχείου
\texttt{\textlatin{docker.socket}} έχουν οριστεί σε \texttt{644} ή πιο
περιοριστικά).

\subsubsection{\texttt{\textlatin{setuid}} \textlatin{Bit}}

Ένας άλλος τρόπος με τον οποίο οι διαχειριστές συστήματος ενδέχεται να
παραλείψουν τη σωστή διαχείριση πρόσβασης είναι να ορίσουν το \textlatin{bit}
\texttt{\textlatin{setuid}} στο \textlatin{docker binary}.

Το \texttt{\textlatin{setuid}} \textlatin{bit} είναι ένα \textlatin{bit} άδειας
στο \textlatin{Unix}, που επιτρέπει στους χρήστες να τρέξουν
\textlatin{binaries} ως κάτοχος (ή ομάδα) του \textlatin{binary} αντί για τον
εαυτό τους. Αυτό είναι χρήσιμο σε συγκεκριμένες περιπτώσεις. Για παράδειγμα,
οι χρήστες θα πρέπει να μπορούν να αλλάζουν τους δικούς τους κωδικούς πρόσβασης,
αλλά δεν θα πρέπει να είναι σε θέση να διαβάσουν τους κατακερματισμούς κωδικών
πρόσβασης άλλους χρήστες. Γι' αυτό το \texttt{\textlatin{passwd}}
\textlatin{binary} (το οποίο χρησιμοποιείται για την αλλαγή του κωδικού
πρόσβασης κάποιου χρήστη) έχει οριστεί το \texttt{\textlatin{setuid}}
\textlatin{bit}. Ένας χρήστης μπορεί να αλλάξει τον κωδικό πρόσβασής του,
επειδή το \texttt{\textlatin{passwd}} εκτελείται ως \texttt{\textlatin{root}}
(ο κάτοχος του \texttt{\textlatin{passwd}}) και, φυσικά, ως
\texttt{\textlatin{root}} μπορεί να διαβάζει και να γράφει στο αρχείο κωδικού
πρόσβασης. Σε αυτή την περίπτωση το \texttt{\textlatin{setuid}} \textlatin{bit}
δεν είναι θέμα ασφαλείας, επειδή το \texttt{\textlatin{passwd}} ζητά τον κωδικ
πρόσβασης του χρήστη από μόνο του και θα αλλάξει μόνο συγκεκριμένες καταχωρήσεις
στο αρχείο κωδικού πρόσβασης.

\section{Σφάλματα Λογισμικού}
\section{Γνωστά \textlatin{Docker CVEs}}

