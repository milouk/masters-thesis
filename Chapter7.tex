\chapter{Λίστα ελέγχων δοκιμών διείσδυσης \textlatin{Docker}}
\label{dockerPenTestChecklist}

Στο κεφάλαιο 4 και στο κεφάλαιο 5 εξετάσαμε τα κοινά τρωτά σημεία και πώς
να τα αναγνωρίσουμε. Σε αυτό το κεφάλαιο θα τα συνοψίσουμε σε μια λίστα ελέγχου
που αποτελείται από ερωτήσεις. Αυτές είναι ερωτήσεις που πρέπει να κάνει ένας
\textlatin{penetration tester} κατά την αξιολόγηση ενός κοντέινερ ή ενός
κεντρικού υπολογιστή. Θα δώσουμε και τα απραίτητα βήματα για το πώς να απαντήθει
η κάθε ερώτηση.

Αυτή η λίστα διατηρείται σκόπιμα σύντομη και χρησιμοποιεί μόνο εντολές
\textlatin{Unix shell} που μπορούν να εκτελεστούν χειροκίνητα, για να είναι
εύκολες και γρήγορες στη χρήση.

Στην ενότητα 5.2 εξετάσαμε τα εργαλεία που βοηθούν στην αυτοματοποίηση της
απαρίθμησης εκμετάλλευση ορισμένων τρωτών σημείων. Τα περισσότερα από αυτά
απαιτούν κάποια ρύθμιση (π.χ. εγκατάσταση βιβλιοθηκών / εξαρτήσεων) και
καλύπτουν μόνο συγκεκριμένες ευπάθειες. Αυτό αντιβαίνει ακριβώς στο σκοπό αυτής
της λίστας ελέγχου και ως εκ τούτου δεν είναι απαραίτητο να χρησιμοποιήσετε
αυτήν τη λίστα ελέγχου (με αξιοσημείωτη εξαίρεση το
\textlatin{Docker Bench for Security}).

Παρόμοια με την ενότητα 5.1, αυτό το κεφάλαιο χωρίζεται σε τρία μέρη, τα οποία
αντιστοιχούν στα μοντέλα επιτιθέμενων του κεφαλαίου 3. Η πρώτη ενότητα έχει
σκοπό να εντοπίσουμε εάν τρέχουμε μέσα σε ένα κοντέινερ (ενότητα 6.1). Αν
ξέρουμε ότι είμαστε μέσα σε ένα κοντέινερ, μπορούμε να αναζητήσουμε τρωτά σημεία
μέσα σε αυτό (βλ. ενότητα 6.2). Αν ξέρουμε ότι δεν τρέχουμε μέσα σε ένα
κοντέινερ, μπορούμε να αναζητήσουμε τρωτά σημεία στον κεντρικό υπολογιστή (βλ.
ενότητα 6.3).

\section{Είμαστε μέσα σε ένα κοντέινερ?}

Αυτές οι ερωτήσεις έχουν σκοπό να προσδιορίσουν το σχετικό μοντέλο επιτιθέμενου
(κεφάλαιο 3). Εάν η απάντηση σε οποιαδήποτε από τις παρακάτω ερωτήσεις είναι
ναι, τότε το πιο πιθανό είναι ότι είμαστε μέσα σε ένα κοντέινερ. Για λεπτομερείς
πληροφορίες, δείτε την ενότητα 5.1.1. Εάν τρέχουμε μέσα σε ένα κοντέινερ, δείτε
την ενότητα 6.2. Αν όχι, δείτε ενότητα 6.3. \\

• Υπάρχει \texttt{\textlatin{/.dockerenv}}; (βλ. ενότητα 6.1.1)

Εκτελέστε το "\texttt{\textlatin{ls /.dockerenv}}" για να δείτε αν υπάρχει
\texttt{\textlatin{/.dockerenv}}
