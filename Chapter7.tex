\chapter{Λίστα ελέγχων δοκιμών διείσδυσης \textlatin{Docker}}
\label{dockerPenTestChecklist}

Στο κεφάλαιο 4 και στο κεφάλαιο 5 εξετάσαμε τα κοινά τρωτά σημεία και πώς
να τα αναγνωρίσουμε. Σε αυτό το κεφάλαιο θα τα συνοψίσουμε σε μια λίστα ελέγχου
που αποτελείται από ερωτήσεις. Αυτές είναι ερωτήσεις που πρέπει να κάνει ένας
\textlatin{penetration tester} κατά την αξιολόγηση ενός κοντέινερ ή ενός
κεντρικού υπολογιστή. Θα δώσουμε και τα απραίτητα βήματα για το πώς να απαντήθει
η κάθε ερώτηση.

Αυτή η λίστα διατηρείται σκόπιμα σύντομη και χρησιμοποιεί μόνο εντολές
\textlatin{Unix shell} που μπορούν να εκτελεστούν χειροκίνητα, για να είναι
εύκολες και γρήγορες στη χρήση.

Στην ενότητα 5.2 εξετάσαμε τα εργαλεία που βοηθούν στην αυτοματοποίηση της
απαρίθμησης εκμετάλλευση ορισμένων τρωτών σημείων. Τα περισσότερα από αυτά
απαιτούν κάποια ρύθμιση (π.χ. εγκατάσταση βιβλιοθηκών / εξαρτήσεων) και
καλύπτουν μόνο συγκεκριμένες ευπάθειες. Αυτό αντιβαίνει ακριβώς στο σκοπό αυτής
της λίστας ελέγχου και ως εκ τούτου δεν είναι απαραίτητο να χρησιμοποιήσετε
αυτήν τη λίστα ελέγχου (με αξιοσημείωτη εξαίρεση το
\textlatin{Docker Bench for Security}).

Παρόμοια με την ενότητα 5.1, αυτό το κεφάλαιο χωρίζεται σε τρία μέρη, τα οποία
αντιστοιχούν στα μοντέλα επιτιθέμενων του κεφαλαίου 3. Η πρώτη ενότητα έχει
σκοπό να εντοπίσουμε εάν τρέχουμε μέσα σε ένα κοντέινερ (ενότητα 6.1). Αν
ξέρουμε ότι είμαστε μέσα σε ένα κοντέινερ, μπορούμε να αναζητήσουμε τρωτά σημεία
μέσα σε αυτό (βλ. ενότητα 6.2). Αν ξέρουμε ότι δεν τρέχουμε μέσα σε ένα
κοντέινερ, μπορούμε να αναζητήσουμε τρωτά σημεία στον κεντρικό υπολογιστή (βλ.
ενότητα 6.3).

\section{Είμαστε μέσα σε ένα κοντέινερ?}

Αυτές οι ερωτήσεις έχουν σκοπό να προσδιορίσουν το σχετικό μοντέλο επιτιθέμενου
(κεφάλαιο 3). Εάν η απάντηση σε οποιαδήποτε από τις παρακάτω ερωτήσεις είναι
ναι, τότε το πιο πιθανό είναι ότι είμαστε μέσα σε ένα κοντέινερ. Για λεπτομερείς
πληροφορίες, δείτε την ενότητα 5.1.1. Εάν τρέχουμε μέσα σε ένα κοντέινερ, δείτε
την ενότητα 6.2. Αν όχι, δείτε ενότητα 6.3. \\

• Υπάρχει \texttt{\textlatin{/.dockerenv}}; (βλ. ενότητα 5.1.1)

Εκτελέστε το "\texttt{\textlatin{ls /.dockerenv}}" για να δείτε αν υπάρχει
\texttt{\textlatin{/.dockerenv}} \\

• Το \texttt{\textlatin{/proc/1/cgroup}} περιέχει
"\texttt{\textlatin{/docker/}}"; (βλ. ενότητα 5.1.1)

Εκτελέστε το \texttt{\textlatin{"grep '/docker/' /proc/1/cgroup"}} για να βρείτε
όλες τις γραμμές στο \texttt{\textlatin{/proc/1/cgroup}} που περιέχουν
\texttt{\textlatin{"/docker/"}}. \\

• Υπάρχουν λιγότερες από 5 διαδικασίες; (βλ. ενότητα 5.1.1)
Εκτελέστε το \texttt{\textlatin{"ps aux"}} για να δείτε όλες τις διεργασίες.\\

• Είναι η διαδικασία με \textlatin{process id} 1 μια κοινή αρχική διεργασία;
(βλ. ενότητα 5.1.1)

Εκτελέστε το \texttt{\textlatin{"ps -p1"}} για να δείτε τη διαδικασία με το
\textlatin{process id} 1 και ελέγξτε αν είναι μια κοινή αρχική διαδικασία
(π.χ. \texttt{\textlatin{systemd}} ή \texttt{\textlatin{init}}). \\

• Δεν υπάρχουν κοινές βιβλιοθήκες και \textlatin{binary} αρχεία στο σύστημα;
(βλ. ενότητα 5.1.1)

Μπορούμε να χρησιμοποιήσουμε την εντολή \texttt{\textlatin{which}} για να βρούμε
διαθέσιμα \textlatin{binary} αρχεία. Για παράδειγμα, το
\texttt{\textlatin{"which sudo"}} θα μας πει εάν το \textlatin{binary}
\texttt{\textlatin{sudo}} είναι διαθέσιμο.

\section{Εύρεση ευπαθειών μέσα σε ένα κοντέινερ}

Οι παρακάτω ερωτήσεις και βήματα έχουν σκοπό να εντοπίσουν ενδιαφέροντα μέρη
και αδύναμα σημεία μέσα στα κοντέινερς. Για λεπτομερείς πληροφορίες, δείτε την
ενότητα 3.1 και ενότητα 5.1.2.

• Ποιος είναι ο τρέχων χρήστης; (βλ. ενότητα 5.1.2)

Εκτελέστε το \texttt{\textlatin{"id"}} για να δείτε ποιος είναι ο τρέχων χρήστης
και σε ποιες ομάδες ανήκει. \\

• Ποιοι χρήστες είναι διαθέσιμοι στο σύστημα; (βλ. ενότητα 5.1.2)

Διαβάστε το \texttt{\textlatin{/etc/passwd}} για να δείτε ποιοι χρήστες είναι
διαθέσιμοι. \\

• Ποιο είναι το λειτουργικό σύστημα του κοντέινερ; (βλ. ενότητα 5.1.2)

Διαβάστε το \texttt{\textlatin{/etc/os-release}} για να λάβετε πληροφορίες
σχετικά με το λειτουργικό σύστημα. \\

• Ποιες διαδικασίες εκτελούνται; (βλ. ενότητα 5.1.2)

Εκτελέστε το \texttt{\textlatin{"ps aux"}} για να δείτε όλες τις διεργασίες. \\

• Τι είναι το λειτουργικό στο κεντρικό υπολογιστή; (βλ. ενότητα 5.1.2)

Εκτελέστε το \texttt{\textlatin{"uname -a"}} για να λάβετε πληροφορίες σχετικά
με τον πυρήνα και το υποκείμενο λειτουργικό σύστημα κεντρικού υπολογιστή. \\

• Ποιες δυνατότητες έχουν οι διεργασίες στο κοντέινερ; (βλ. ενότητα 5.1.2)

Δείτε την τρέχουσα τιμή των δυνατοτήτων εκτελώντας το
\texttt{\textlatin{"grep CapEff /proc/self/status»}} και αποκωδικοποιήστε το με
\texttt{\textlatin{«capsh --decode=value»}}. Το \textlatin{capsh}
μπορεί να εκτελεστεί σε διαφορετικό σύστημα. \\

• Το κοντέινερ λειτουργεί σε προνομιακή λειτουργία; (βλ. ενότητα 5.1.2)

Εάν η τιμή \texttt{\textlatin{CapEff}} του προηγούμενου βήματος ισούται με
0000003ffffffffff, το κοντέινερ λειτουργεί σε προνομιακή λειτουργία και μπορούμε
να αποδράσουμε αυτό (βλ. ενότητα 4.1.3). \\

• Ποια \textlatin{volumes} είναι προσαρτημένα; (βλ. ενότητα 5.1.2)

Διαβάστε \texttt{\textlatin{/proc/mounts}} για να δείτε όλες τις προσαρτήσεις
συμπεριλαμβανομένων των \textlatin{volumes}. \\

• Υπάρχουν ευαίσθητες πληροφορίες αποθηκευμένες σε μεταβλητές περιβάλλοντος;
(βλ. ενότητα 5.1.2)

Η εντολή \texttt{\textlatin{"env"}} θα εμφανίσει όλες τις μεταβλητές
περιβάλλοντος. \\


• Είναι το \textlatin{Docker socket} τοποθετημένο μέσα στο κοντέινερ;
(βλ. ενότητα 5.1.2)

Ελέγξτε το \texttt{\textlatin{/proc/mounts}} για να δείτε εάν το
\texttt{\textlatin{docker.sock}} είναι τοποθετημένο μέσα στο κοντέινερ.
Το \texttt{\textlatin{/run/docker.sock}} είναι ένα κοινό σημείο προσάρτησης. Αν
το βρούμε, μπορούμε να ξεφύγουμε από το κοντέινερ και να αλληλεπιρδάσουμε με το
\textlatin{Docker daemon} στον κεντρικό υπολογιστή. \\


• Ποιοι κεντρικοί υπολογιστές είναι προσβάσιμοι στο δίκτυο; (βλ. ενότητα 5.1.2)

Εάν είναι δυνατόν, χρησιμοποιήστε το \textlatin{nmap} για να σαρώσετε το τοπικό
δίκτυο για προσβάσιμους κεντρικούς υπολογιστές. Η διεύθυνση \textlatin{IPv4} του
κοντέινερ βρίσκεται στο \texttt{\textlatin{/etc/host}}. \\


\section{Εύρεση ευπαθειών στο κεντρικό υπολογιστή}

Οι παρακάτω ερωτήσεις και βήματα έχουν σκοπό να εντοπίσουν ενδιαφέροντα μέρη
και αδύναμα σημεία στους κεντρικούς υπολογιστές που τρέχουν το
\textlatin{Docker}. Για λεπτομερείς πληροφορίες, βλ. ενότητα 3.2 και ενότητα
5.1.3.


• Ποια είναι η έκδοση του \textlatin{Docker}; (βλ. ενότητα 5.1.3)

Εκτελέστε το \texttt{\textlatin{"docker --version"}} για να βρείτε την έκδοση
του \textlatin{Docker}. Θα χρειασεί να ελέγξετε εάν υπάρχουν γνωστά σφάλματα που
σχετίζονται με το λογισμικό (ενότητα 4.2) σε αυτή η έκδοση του
\textlatin{Docker} (βλ. ενότητα 4.2). Μπορούμε να βρούμε σχετικά \textlatin{CVE}
στο την Εθνική Βάση Δεδομένων Ευπάθειας
\footnote{\textlatin{https://nvd.nist.gov/}}. \\

• Ποιες οδηγίες \textlatin{CIS Docker Benchmark} εφαρμόζονται λάθος ή δεν
ακολουθούνται; (βλ. ενότητα 5.2.1)

Εκτελέστε το \textlatin{Docker Bench for Security}
\footnote{\textlatin{https://github.com/docker/docker-bench-security}}
για να δείτε γρήγορα ποιες \textlatin{CIS Docker} κατευθυντήριες γραμμές
αναφορές δεν ακολουθούνται. \\


• Ποιοι χρήστες επιτρέπεται να αλληλεπιδρούν με το \textlatin{Docker socket};
(βλ. ενότητα 5.1.3)

Εκτελέστε το \texttt{\textlatin{“ls -l /var/run/docker.sock”}} για να δείτε τον
ιδιοκτήτη και την ομάδα του \texttt{\textlatin{/var/run/docker.sock}} και στο
οποίο οι χρήστες έχουν πρόσβαση ανάγνωσης και εγγραφής. Χρήστες που έχουν
δικαιώματα ανάγνωσης και εγγραφής στο \textlatin{Docker socket} επιτρέπεται να
αλληλεπιδρούν μαζί του. \\

• Ποιος ανήκει στo \texttt{\textlatin{docker group}}; (βλ. ενότητα 5.1.3)

Ελέγξτε ποιοι χρήστες ανήκουν στην ομάδα που προσδιορίστηκε στο προηγούμενο βήμα
(από προεπιλεγμένο \texttt{\textlatin{docker}}) εκτελώντας το
\texttt{\textlatin{"grep docker /etc/group"}}. \\

• Έχει οριστεί το \texttt{\textlatin{setuid bit}} στο
\textlatin{Docker client binary}; (βλ. ενότητα 5.1.3)

Ελέγξτε τα δικαιώματα (συμπεριλαμβανομένου του αν έχει οριστεί το
\texttt{\textlatin{setuid bit}}) του \textlatin{Docker Binary} εκτελώντας
\texttt{\textlatin{“ls -l \$(which docker)}}”. \\

• Ποιες εικόνες είναι διαθέσιμες; (βλ. ενότητα 5.1.3)

Δείτε τις διαθέσιμες εικόνες εκτελώντας το
\texttt{\textlatin{"docker images -a"}}. \\

• Ποια κοντέινερς είναι διαθέσιμα; (βλ. ενότητα 5.1.3)

Δείτε όλα τα κοντέινερ (σε λειτουργία ή που έχουν σταματήσει) εκτελώντας το
\texttt{\textlatin{"docker ps -a"}}. \\


• Πώς ξεκινά ο \textlatin{Docker Daemon}; (βλ. ενότητα 5.1.3)

Ελέγξτε τα αρχεία διαμόρφωσης (π.χ.
\texttt{\textlatin{/usr/lib/systemd/system/docker.service}} και
\texttt{\textlatin{/etc/docker/daemon.json}}) για πληροφορίες σχετικά με τον
τρόπο με τον οποίο ο \textlatin{Docker Daemon} έχει ξεκινήσει. \\

• Υπάρχουν αρχεία \texttt{\textlatin{docker-compose.yaml}}; (βλ. ενότητα 4.1.2
και ενότητα 5.1.3)

Βρείτε όλα τα αρχεία \texttt{\textlatin{docker-compose.yaml}} χρησιμοποιώντας το
\texttt{\textlatin{"find / -name "dockercompose.*""}}. \\

• Υπάρχουν αρχεία \texttt{\textlatin{.docker/config.json}}; (βλ. ενότητα 4.1.2
και ενότητα 5.1.3)

Διαβάστε τα αρχεία \texttt{\textlatin{config.json}} σε όλους τους καταλόγους
εκτελώντας το \texttt{\textlatin{“cat /home /*/.docker/config.json”}}. \\


• Έχουν οριστεί οι κανόνες \texttt{\textlatin{iptables}} τόσο για τον κεντρικό
υπολογιστή όσο και για τα κοντέινερ; (βλ. ενότητα 5.1.3)

Δείτε τα \texttt{\textlatin{iptables}} εκτελώντας τις εντολές
\texttt{\textlatin{"iptables -vnL"}} και
\texttt{\textlatin{"iptables -t filter -vnL”}}.


