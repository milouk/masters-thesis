\chapter{Δοκιμές Διείσδυσης \textlatin{Docker} / Βήματα}
\label{dockerPenTesting}

Στο κεφάλαιο 4 μελετήσαμε συγκεκριμένα τρωτά σημεία. Προτού μπορέσουμε να τα
εκμεταλλευτούμε όμως, πρέπει πρώτα να πραγματοποιήσουμε αναγνώριση στο
σύστημα στόχου για τη συλλογή δεδομένων. Αυτά τα δεδομένα μπορούν στη συνέχεια
να χρησιμοποιηθούν για τον εντοπισμό αδύναμων σημείων και
τρωτών σημεία. Αυτό το κεφάλαιο θα επικεντρωθεί στη συλλογή αυτών των
ενδιαφερόντων δεδομένων και τον εντοπισμό αυτών των τρωτών σημείων.
Στην ενότητα 5.1 εστιάζουμε στο πώς να το κάνουμε αυτό χειροκίνητα και για τις
δύο οπτικές του κεφαλαίου 3. Στην ενότητα 5.2 θα ερευνήσουμε τα διαθέσιμα
εργαλεία που θα μας βοηθήσουν να αυτοματοποιήσουμε μέρος των αξιολογήσεων. Στο
Κεφάλαιο 6 θα συνδυάσουμε τις πληροφορίες από το κεφάλαιο 4 και από αυτό το
κεφάλαιο σε μια λίστα ελέγχου.

\section{Χειροκίνητη αναγνώριση ευπαθειών}

Σε αυτήν την ενότητα θα συζητήσουμε πώς μπορούμε να προσδιορίσουμε με μη
αυτόματο τρόπο τα τρωτά σημεία που εξετάσαμε στο κεφάλαιο 4 μόλις έχουμε
πρόσβαση σε ένα σύστημα. Αυτή η ενότητα χωρίζεται σε τρία μέρη, που
αντιστοιχούν στα μοντέλα επιτιθέμενων του κεφαλαίου 3.

Στην ενότητα 5.1.1 εξετάζουμε τεχνικές για να προσδιορίσουμε ποιο μοντέλο
επιτιθέμενου είναι σχετικό κατά τη διάρκεια μιας αξιολόγησης. Αυτό σημαίνει ότι
θα συζητήσουμε τεχνικές για να προσδιορίζουμε αν βρισκόμαστε μέσα σε ένα
κοντέινερ ή σε έναν κεντρικό υπολογιστή.

Το δεύτερο μέρος (ενότητα 5.1.2) αντιστοιχεί αποκλειστικά στις διαφυγές
κοντέινερ (ενότητα 3.1). Παίρνουμε την οπτική μιας διαδικασίας μέσα σε ένα
δοχείο και θα δούμε πώς θα μπορούσαμε να εκτελέσουμε μια επίθεση διαφυγής
κοντέινερ.

Το τρίτο μέρος (ενότητα 5.1.3) αντιστοιχεί σε \textlatin{Docker daemon}
επιθέσεις (ενότητα 3.2). Παίρνουμε την οπτική μιας (μη προνομιούχου) διαδικασίας
σε έναν κεντρικό υπολογιστή με εγκατεστημένο το \textlatin{Docker} και θα δούμε
πώς θα μπορούσαμε 

Θα επικεντρωθούμε κυρίως στις εσφαλμένες διαμορφώσεις (ενότητα 4.1), επειδή
αν και τα σφάλματα που σχετίζονται με την ασφάλεια (ενότητα 4.2) ενδέχεται να
έχουν μεγάλο αντίκτυπο, όλα αυτά μετριάζονται με μια απλή συμβουλή:
«Διατηρήστε τα συστήματά σας ενημερωμένα». Ο έλεγχος εάν ένα σύστημα είναι
ευάλωτο σε ένα γνωστό σφάλμα είναι επίσης πολύ πιο εύκολο από το να ελέγξουμε
εάν ένα σύστημα είναι ευάλωτο λόγω εσφαλμένης διαμόρφωσης, επειδή όλα τα
σφάλματα \textlatin{Docker} εξαρτώνται από την μη ενημερωμένη έκδοση του
\textlatin{Docker}.

\subsection{Αναγνώριση του αν βρισκόμαστε σε περιβάλλον κοντέινερ}

Στις περισσότερες αξιολογήσεις ασφαλείας και δοκιμές διείσδυσης θα είναι
ξεκάθαρο τι είδους σύστημα (δηλαδή αν είμαστε μέσα σε ένα κοντέινερ ή όχι)
επιτιθέμεθα. Σε ορισμένες περιπτώσεις, ωστόσο, μπορεί να μην είναι. Ένα καλό
παράδειγμα αυτού είναι η εύρεση ενός \textlatin{RCE} σε ένα σύστημα κατά τη
δοκιμή διείσδυσης τύπου \textlatin{black box}. Αυτό μας επιτρέπει να εκτελούμε
αυθαίρετες εντολές σε ένα σύστημα για το οποίο δεν γνωρίζουμε τίποτα. Σε μια
τέτοια περίπτωση είναι σημαντικό να ξέρουμε αν είμαστε μέσα σε περιβάλλον
\textlatin{Docker} κοντέινερ ή όχι.

Σε αυτή την ενότητα, θα δούμε βήματα που μας δείχνουν αν βρισκόμαστε σε ένα
κοντέινερ \textlatin{Docker}. Αυτά τα βήματα είναι σε φθίνουσα σειρά ευκολίας
και βεβαιότητας. Εάν γνωρίζουμε ότι βρισκόμαστε μέσα σε ένα κοντέινερ, μπορούμε
να κάνουμε \textlatin{reconnaissance} δηλαδή αναγνώριση μέσα το κοντέινερ (βλ.
ενότητα 5.1.2). Αν ξέρουμε ότι δεν τρέχουμε μέσα σε ένα κοντέινερ, μπορούμε να
εκτελέσουμε αναγνώριση στον κεντρικό υπολογιστή (βλ. ενότητα 5.1.3).

\subsubsection{\texttt{\textlatin{/.dockerenv}}}

Το \texttt{\textlatin{/.dockerenv}} είναι ένα αρχείο που υπάρχει σε όλα τα
κοντέινερ \textlatin{Docker}. Χρησιμοποιήθηκε στο παρελθόν από την
\textlatin{LXC} για να φορτώσει τις μεταβλητές περιβάλλοντος στο κοντέινερ
\footnote{Η \textlatin{LXC} ήταν το \textlatin{engine} που χρησιμοποιούσε το
\textlatin{Docker} προκειμένου να δημιουργήσει κοντέινερς. Πλέον έχει
αντικατασταθεί με το \texttt{\textlatin{containerd}}}.
Προς το παρόν είναι πάντα άδειο, γιατί το \textlatin{LXC} δεν χρησιμοποιείται
πλέον. Ωστόσο, εξακολουθεί να χρησιμοποιείται (επίσημα) για να προσδιορίσει εάν
μια διεργασία εκτελείται σε ένα \textlatin{Docker} κοντέινερ
\cite{Metasploit-Linux-Gather-Container-Detection}
\cite{Removed-Dockerinit-Reference}.


\subsubsection{\textlatin{Control Group}}

Για να περιορίσει τους πόρους των κοντέινερ, το \textlatin{Docker} δημιουργεί
ομάδες ελέγχου για καθένα κοντέινερ και μια ομάδα γονικού ελέγχου
(\textlatin{parent control group}) που ονομάζεται \texttt{\textlatin{docker}}.
Εάν ξεκινήσει μια διαδικασία σε ένα κοντέινερ \textlatin{Docker}, αυτή η
διαδικασία θα πρέπει να βρίσκεται στην ομάδα ελέγχου εκείνου του κοντέινερ.
Μπορούμε να το επαληθεύσουμε αυτό κοιτάζοντας την \texttt{\textlatin{cgroup}}
της αρχικής διεργασίας \texttt{\textlatin{(/proc/1/cgroups)}}
\cite{Metasploit-Linux-Gather-Container-Detection}. \\

\texttt{\textlatin{(cont)\# cat /proc/1/cgroup}}

\texttt{\textlatin{12:hugetlb:/docker/0c7a3b8...}}

\texttt{\textlatin{11:blkio:/docker/0c7a3b8...}}

\texttt{\textlatin{...}}

Αν κοιτάξουμε έναν κεντρικό υπολογιστή, δεν βλέπουμε το ίδιο
\texttt{\textlatin{/docker/}} \textlatin{parent control group}. \\

\texttt{\textlatin{(cont)\# cat /proc/1/cgroup}}

\texttt{\textlatin{12:hugetlb:/}}

\texttt{\textlatin{11:blkio:/}}

\texttt{\textlatin{...}}

Σε ορισμένα συστήματα που χρησιμοποιούν \textlatin{Docker} (π.χ. λογισμικό
ενορχήστρωσης), το \textlatin{parent control group} έχει άλλο όνομα (π.χ.
\texttt{\textlatin{kubepod}} για \textlatin{Kubernetes}).

\subsubsection{\textlatin{Running Processes}}

Τα κοντέινερ είναι κατασκευασμένα για να εκτελούν μόνο μία διεργασία, ενώ τα
κεντρικά συστήματα (\textlatin{hosts}) εκτελούν πολλές διεργασίες. Οι
διεργασίες σε συστήματα κεντρικού υπολογιστή έχουν μία διαδικασία ρίζας
(με αναγνωριστικό διεργασίας 1 (\textlatin{process id 1})) που εκκινεί όλες τις
απαραίτητες (\textlatin{child}) διαδικασίες. Στα περισσότερα συστήματα
\textlatin{Linux} η διαδικασία είναι είτε \texttt{\textlatin{init}} είτε
\texttt{\textlatin{systemd}}. Δεν θα βλέπαμε ποτέ \texttt{\textlatin{init}} ή
\texttt{\textlatin{systemd}} σε ένα κοντέινερ, επειδή το κοντέινερ εκτελεί μόνο
μία διαδικασία και όχι πλήρη λειτουργικό σύστημα. Αυτός είναι ο λόγος για τον
οποίο ο αριθμό των διαδικασιών και της διαδικασίας με το \textlatin{pid} 1 είναι
μια καλή ένδειξη εάν τρέχουμε σε κοντέινερ.

\subsubsection{Διαθέσιμες βιβλιοθήκες και \textlatin{binaries}}

Οι εικόνες Docker κατασκευάζονται όσο το δυνατόν μικρότερες. Πολλές διαδικασίες
δεν χρειάζονται ένα πλήρως λειτουργικό σύστημα \textlatin{Linux}, χρειάζονται
μόνο ένα μέρος του. Γι' αυτό oι προγραμματιστές συχνά αφαιρούν βιβλιοθήκες και
\textlatin{binary} αρχεία που δεν χρειάζονται για την συγκεκριμένη εφαρμογή από
τις εικόνες \textlatin{Docker} τους. Αν δούμε πολλά πακέτα που λείπουν,
\textlatin{binaries} ή βιβλιοθήκες είναι μια καλή ένδειξη ότι τρέχουμε
μέσα σε ένα κοντέινερ.

Το πακέτο \texttt{\textlatin{sudo}} είναι ένα παράδειγμα αυτού. Αυτό το πακέτο
είναι ζωτικής σημασίας για πολλές διανομές \textlatin{Linux}, γιατί δίνει τη
δυνατότητα εκτέλεσης σε χρήστες που δεν είναι \texttt{\textlatin{root}}
εντολές ως \texttt{\textlatin{root}}. Ωστόσο, σε ένα κοντέινερ
\textlatin{Docker} το πακέτο \texttt{\textlatin{sudo}} δεν έχει πολύ νόημα. Εάν
μια διεργασία χρειάζεται να τρέξει κάτι ως \texttt{\textlatin{root}}, τότε
η διαδικασία πρέπει να εκτελείται ως \texttt{\textlatin{root}} στο κοντέινερ.
Γι' αυτό το \texttt{\textlatin{sudo}} συχνά δεν είναι εγκατεστημένο στις εικόνες
\textlatin{Docker}.


\subsection{Δοκιμές διείσδυσης μέσα σε ένα κοντέινερ}

Εάν έχουμε εκτέλεση κώδικα μέσα σε ένα κοντέινερ, θα πρέπει να επικεντρωθούμε
στο σενάριο διαφυγής από το κοντέινερ (βλ. ενότητα 3.1). Επειδή τρέχει ο
\textlatin{Docker daemon} ως \texttt{\textlatin{root}}, πιθανότατα θα
καταφέρουμε να αποκτήσουμε πρόσβαση \texttt{\textlatin{root}} στον κεντρικό
υπολογιστή, εάν ξεφύγουμε από το κοντέινερ. Θα ρίξουμε μια ματιά στα βήματα που
μπορούμε να κάνουμε για να αναγνωρίσουμε το λειτουργικό σύστημα του κοντέινερ, 
την εικόνα του κοντέινερ, το λειτουργικό σύστημα του κεντρικού υπολογιστή καθώς
και τα αδύναμα σημεία του κοντέινερ.

Σε πολλές εικόνες Docker έχουν αφαιρεθεί τα περιττά εργαλεία, \textlatin{binary}
αρχεία και βιβλιοθήκες για να γίνει η εικόνα μικρότερη. Ωστόσο, μπορεί να
χρειαστούμε αυτά τα εργαλεία κατά τη διάρκεια μιας δοκιμής διείσδυσης. Αν
είμαστε \texttt{\textlatin{root}} σε ένα κοντέινερ, το πιθανότερο είναι να 
μπορούμε να εγκαταστήσουμε τα απαραίτητα εργαλεία. Αν έχουμε πρόσβαση μόνο σε μη
\texttt{\textlatin{root}} χρήστη, ενδέχεται να μην είναι δυνατή η εγκατάσταση
οποιουδήποτε εργαλείου. Σε αυτή την περίπτωση, θα πρέπει να δουλέψουμε με ό,τι
έχουμε στη διάθεσή μας ή να βρούμε έναν τρόπο να μεταφέρουμε
\textlatin{statically compiled binaries} μέσα στο κοντέινερ.

\subsubsection{Αναγνώριση χρηστών}

Το πρώτο βήμα που πρέπει να κάνουμε είναι να δούμε αν είμαστε προνομιούχος
χρήστης και να αναγωρίσουμε άλλους χρήστες. Μπορούμε να δούμε τον τρέχοντα
χρήστη μας χρησιμοποιώντας το αναγνωριστικό και να δούμε όλους τους χρήστες
κοιτάζοντας το \texttt{\textlatin{/etc/passwd}}. \\

\texttt{\textlatin{(cont)\# id}}

\texttt{\textlatin{uid=0(root) gid=0(root) groups=0(root)}}

\texttt{\textlatin{(cont)\# cat /etc/passwd}}

\texttt{\textlatin{...}}

\texttt{\textlatin{test:x:1000:1000:,,,:/home/test:/bin/bash}} \\

Βλέπουμε ότι ο τρέχων χρήστης μας είναι \texttt{\textlatin{root}} (το
αναγνωριστικό χρήστη είναι 0) και ότι υπάρχουν δύο χρήστες (εκτός από τους
προεπιλεγμένους χρήστες στο \textlatin{Linux}). Από προεπιλογή, τα κοντέινερ
λειτουργούν ως \texttt{\textlatin{root}}. Αυτό είναι υπέροχο από την άποψη των
επιθετικών, γιατί μας επιτρέπει να έχουμε πλήρη πρόσβαση σε οτιδήποτε βρίσκεται
μέσα στο κοντέινερ. Ένα καλά διαμορφωμένο κοντέινερ πιθανότατα δεν εκτελείται ως
\texttt{\textlatin{root}}.

\subsubsection{Αναγνώριση του λειτουργικού συστήματος του κοντέινερ}

Το επόμενο βήμα είναι να προσδιορίσουμε το λειτουργικό σύστημα (και ίσως την
εικόνα \textlatin{Docker}) του κοντέινερ.

Όλες οι σύγχρονες διανομές \textlatin{Linux} έχουν ένα αρχείο
\texttt{\textlatin{/etc/os-release}} \footnote{Παρ'όλο που το αρχείο αυτό
πρωτοεμφανίστηκε από το \texttt{\textlatin{systemd}} τα λειτουργικά συστήματα
που ρητά δεν χρησιμοποιούν το \texttt{\textlatin{systemd}} π.χ
\textlatin{Void Linux} χρησιμοποιούν το \texttt{\textlatin{/etc/os-release}}.}
που περιέχει πληροφορίες σχετικά με το λειτουργικό σύστημα που εκτελείται. \\

\texttt{\textlatin{(host)\$ docker run -it --rm centos:latest cat /etc/os-release}}

\texttt{\textlatin{...}}

\texttt{\textlatin{PRETTY\_NAME="CentOS Linux 8 (Core)"}}

\texttt{\textlatin{...}} \\

Για να έχουμε μια καλύτερη ιδέα για το τι πρέπει να κάνει ένα κοντέινερ,
μπορούμε να δούμε τις διαδικασίες. Επειδή τα κοντέινερ θα πρέπει να έχουν μόνο
μια μοναδική εργασία (π.χ. που τρέχει μια βάση δεδομένων), θα πρέπει να έχουν
μόνο μία διεργασία που εκτελείται. \\

\texttt{\textlatin{(host)\$ docker run --rm -e MYSQL\_RANDOM\_ROOT\_PASSWORD=true --name=database mariadb:latest}}

\texttt{\textlatin{...}}

\texttt{\textlatin{(host)\$ docker exec database ps -A -o pid,cmd}}

\texttt{\textlatin{PID CMD}}

\texttt{\textlatin{   1 mysqld}}

\texttt{\textlatin{320 ps -A -o pid,cmd}} \\


Σε αυτό το παράδειγμα, βλέπουμε ότι η εικόνα \textlatin{mariadb} έχει μόνο μία
διαδικασία \textlatin{(mysqld)} \footnote{Παρατηρούμε επίσης και τη διαδικασία
που δείχνει όλες τις διαδικασίες με \textlatin{process id 320}}. Με αυτόν τον
τρόπο γνωρίζουμε ότι το κοντέινερ είναι \textlatin{MySQL Server} και είναι
πιθανώς (με βάση) την προεπιλεγμένη εικόνα \textlatin{MySQL Docker (mariadb)}.

\subsubsection{Αναγνώριση του λειτουργικού συστήματος του κεντρικού υπολογιστή}

Είναι επίσης σημαντικό να αναζητήσουμε πληροφορίες για τον κεντρικό υπολογιστή.
Αυτό μπορεί να είναι χρήσιμο για τον εντοπισμό και τη χρήση σχετικών
εκμεταλλεύσεων.

Επειδή τα κοντέινερ χρησιμοποιούν τον πυρήνα του κεντρικού υπολογιστή, μπορούμε
να χρησιμοποιήσουμε την έκδοση του πυρήνα για αναγνώριση πληροφοριών σχετικά με
τον κεντρικό υπολογιστή. Ας ρίξουμε μια ματιά στο το παρακάτω παράδειγμα που
εκτελείται σε έναν κεντρικό υπολογιστή \textlatin{Ubuntu}. \\

\texttt{\textlatin{(host)\$ docker run -it --rm alpine:latest cat /etc/os-release}}

\texttt{\textlatin{...}}

\texttt{\textlatin{PRETTY\_NAME="Alpine Linux v3.10"}}

\texttt{\textlatin{...}}

\texttt{\textlatin{(host)\$ docker run -it --rm alpine:latest uname -rv}}

\texttt{\textlatin{5.0.0-36-generic \#39\~18.04.1-Ubuntu SMP Fri Dec 12 11:09:50 UTC 2019}} \\


Τρέχουμε ένα κοντέινερ \textlatin{Alpine Linux}, το οποίο βλέπουμε όταν κοιτάμε
στο αρχείο \texttt{\textlatin{/etc/os-release}}. Ωστόσο, όταν κοιτάμε την έκδοση
του πυρήνα (χρησιμοποιώντας την εντολή \texttt{\textlatin{uname}}), βλέπουμε ότι
χρησιμοποιούμε έναν πυρήνα \textlatin{Ubuntu}. Αυτό σημαίνει ότι πιθανότατα
τρέχουμε σε έναν κεντρικό υπολογιστή \textlatin{Ubuntu}.

Βλέπουμε επίσης τον αριθμό έκδοσης του πυρήνα (σε αυτήν την περίπτωση
\textlatin{5.0.0-36-generic}). Αυτό μπορεί να χρησιμοποιηθεί για να δούμε εάν το
σύστημα είναι ευάλωτο σε εκμεταλλεύσεις πυρήνα, επειδή ορισμένα
\textlatin{exploits} του πυρήνα μπορούν να χρησιμοποιηθούν για να διαφύγουν από
το κοντέινερ.

\subsubsection{Ανάγνωση μεταβλητών περιβάλλοντος}

Οι μεταβλητές περιβάλλοντος είναι ένας τρόπος επικοινωνίας πληροφοριών σε
κοντέινερ όταν ξεκινούν. Όταν ξεκινά ένα κοντέινερ, οι μεταβλητές περιβάλλοντος
μεταβιβάζονται σε αυτό. Αυτές οι μεταβλητές συχνά περιέχουν κωδικούς πρόσβασης
και άλλες ευαίσθητες πληροφορίες.

Μπορούμε να απαριθμήσουμε όλες τις μεταβλητές περιβάλλοντος που έχουν οριστεί
μέσα σε ένα \textlatin{Docker} κοντέινερ χρησιμοποιώντας την εντολή
\texttt{\textlatin{env}} (ή κοιτάζοντας το αρχείο
\texttt{\textlatin{/proc/pid/environ}} μιας διαδικασίας).\\

\texttt{\textlatin{(host)\$ docker run --rm -e MYSQL\_ROOT\_PASSWORD=supersecret --name=database mariadb:latest}}

\texttt{\textlatin{(host)\$ docker exec -it database bash}}

\texttt{\textlatin{(cont)\# env}}

\texttt{\textlatin{...}}

\texttt{\textlatin{MYSQL\_ROOT\_PASSWORD=supersecret}}

\texttt{\textlatin{...}}\\

Πρέπει να σημειωθεί ότι δεν πρόκειται για εσφαλμένη διαμόρφωση. Η χρήση
μεταβλητών περιβάλλοντος είναι ο επιδιωκόμενος τρόπος για να μεταβιβαστούν
ευαίσθητες πληροφορίες σε έναν \textlatin{Docker} κοντέρινερ κατά το χρόνο
εκτέλεσης. Ωστόσο, κατά τη διάρκεια μιας δοκιμής διείσδυσης τύπου
\textlatin{blackbox}, οι ευαίσθητες πληροφορίες που είναι αποθηκευμένες στις
μεταβλητές περιβάλλοντος μπορεί να είναι χρήσιμες.

\subsubsection{Έλεγχος δυνατοτήτων}




\subsection{Αναγνώριση Ευπαθειών}
\subsection{Εργαλεία}
\subsection{Εύρεση \textlatin{host} / Ευπάθειες \textlatin{Container}}
\subsection{Εκμετάλλευση}
