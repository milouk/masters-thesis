\chapter{Δοκιμές Παρείσδυσης \textlatin{Docker} / Βήματα}
\label{dockerPenTesting}

Στο κεφάλαιο 4 μελετήσαμε συγκεκριμένα τρωτά σημεία. Προτού μπορέσουμε να τα
εκμεταλλευτούμε όμως, πρέπει πρώτα να πραγματοποιήσουμε αναγνώριση στο
σύστημα στόχου για τη συλλογή δεδομένων. Αυτά τα δεδομένα μπορούν στη συνέχεια
να χρησιμοποιηθούν για τον εντοπισμό αδύναμων και τρωτών σημείων. Αυτό το
κεφάλαιο θα επικεντρωθεί στη συλλογή αυτών των ενδιαφερόντων δεδομένων και τον
εντοπισμό αυτών των τρωτών σημείων. Στην ενότητα 5.1 εστιάζουμε στο πώς να το
κάνουμε αυτό χειροκίνητα και για τις δύο οπτικές του κεφαλαίου 3.Στην ενότητα
5.2 θα ερευνήσουμε τα διαθέσιμα εργαλεία που θα μας βοηθήσουν να
αυτοματοποιήσουμε μέρος των αξιολογήσεων. Στο Κεφάλαιο 6 θα συνδυάσουμε τις
πληροφορίες από το κεφάλαιο 4 και από αυτό το κεφάλαιο σε μια λίστα ελέγχου.

\section{Χειροκίνητη αναγνώριση ευπαθειών}

Σε αυτήν την ενότητα θα συζητήσουμε πώς μπορούμε να προσδιορίσουμε με μη
αυτόματο τρόπο τα τρωτά σημεία που εξετάσαμε στο κεφάλαιο 4 μόλις έχουμε
πρόσβαση σε ένα σύστημα. Αυτή η ενότητα χωρίζεται σε τρία μέρη, που
αντιστοιχούν στα μοντέλα επιτιθέμενων του κεφαλαίου 3.

Στην ενότητα 5.1.1 εξετάζουμε τεχνικές για να προσδιορίσουμε ποιο μοντέλο
επιτιθέμενου είναι σχετικό κατά τη διάρκεια μιας αξιολόγησης. Αυτό σημαίνει ότι
θα συζητήσουμε τεχνικές για να προσδιορίζουμε αν ο επιτιθέμενος βρίσκεται μέσα
σε ένα κοντέινερ ή σε έναν κεντρικό υπολογιστή.

Το δεύτερο μέρος (ενότητα 5.1.2) αντιστοιχεί αποκλειστικά στις διαφυγές
κοντέινερ (ενότητα 3.1). Παίρνουμε την οπτική μιας διαδικασίας μέσα σε ένα
κοντέινερ και θα δούμε πώς θα μπορούσαμε να εκτελέσουμε μια επίθεση διαφυγής
κοντέινερ.

Το τρίτο μέρος (ενότητα 5.1.3) αντιστοιχεί σε \textlatin{Docker daemon}
επιθέσεις (ενότητα 3.2). Παίρνουμε την οπτική μιας (μη προνομιούχου) διαδικασίας
σε έναν κεντρικό υπολογιστή με εγκατεστημένο το \textlatin{Docker} και θα δούμε
πώς θα μπορούσαμε να εκτελέσουμε επιθέσεις \textlatin{Docker Daemon}.

Θα επικεντρωθούμε κυρίως στις εσφαλμένες διαμορφώσεις (ενότητα 4.1), επειδή
αν και τα σφάλματα που σχετίζονται με την ασφάλεια (ενότητα 4.2) ενδέχεται να
έχουν μεγάλο αντίκτυπο, όλα αυτά μετριάζονται με μια απλή συμβουλή:
«Διατηρήστε τα συστήματά σας ενημερωμένα». Ο έλεγχος ως προς το εάν ένα σύστημα
είναι ευάλωτο σε ένα γνωστό σφάλμα είναι επίσης πολύ πιο εύκολος από το να
ελέγξουμε εάν ένα σύστημα είναι ευάλωτο λόγω εσφαλμένης διαμόρφωσης, επειδή όλα
τα σφάλματα \textlatin{Docker} εξαρτώνται από την μη ενημερωμένη έκδοση του
\textlatin{Docker}.

\subsection{Αναγνώριση για το αν βρισκόμαστε σε περιβάλλον κοντέινερ}

Στις περισσότερες αξιολογήσεις ασφαλείας και δοκιμές διείσδυσης θα είναι
ξεκάθαρο σε τι είδους σύστημα (δηλαδή αν είμαστε μέσα σε ένα κοντέινερ ή όχι)
επιτιθέμεθα. Σε ορισμένες περιπτώσεις, ωστόσο, μπορεί να μην είναι σαφές. Ένα
καλό παράδειγμα αυτού είναι η εύρεση ενός \textlatin{RCE} σε ένα σύστημα κατά τη
δοκιμή διείσδυσης τύπου \textlatin{black box}. Αυτό μας επιτρέπει να εκτελούμε
αυθαίρετες εντολές σε ένα σύστημα, για το οποίο δεν γνωρίζουμε τίποτα. Σε μια
τέτοια περίπτωση είναι σημαντικό να ξέρουμε αν είμαστε μέσα σε περιβάλλον
\textlatin{Docker} κοντέινερ ή όχι.

Σε αυτή την ενότητα, θα δούμε βήματα που μας υποδεικνύουν αν βρισκόμαστε σε ένα
κοντέινερ \textlatin{Docker}. Αυτά τα βήματα είναι σε φθίνουσα σειρά ευκολίας
και βεβαιότητας. Εάν γνωρίζουμε ότι βρισκόμαστε μέσα σε ένα κοντέινερ, μπορούμε
να κάνουμε \textlatin{reconnaissance} δηλαδή αναγνώριση μέσα στο κοντέινερ (βλ.
ενότητα 5.1.2). Αν ξέρουμε ότι δεν τρέχουμε μέσα σε ένα κοντέινερ, μπορούμε να
εκτελέσουμε αναγνώριση στον κεντρικό υπολογιστή (βλ. ενότητα 5.1.3).

\subsubsection{\texttt{\textlatin{/.dockerenv}}}

Το \texttt{\textlatin{/.dockerenv}} είναι ένα αρχείο που υπάρχει σε όλα τα
κοντέινερ \textlatin{Docker}. Χρησιμοποιήθηκε στο παρελθόν από την
\textlatin{LXC} για να φορτώσει τις μεταβλητές περιβάλλοντος στο κοντέινερ
\footnote{Η \textlatin{LXC} ήταν το \textlatin{engine} που χρησιμοποιούσε το
\textlatin{Docker} προκειμένου να δημιουργήσει κοντέινερς. Πλέον έχει
αντικατασταθεί με το \texttt{\textlatin{containerd}}.}.
Προς το παρόν είναι πάντα κενό, γιατί το \textlatin{LXC} δεν χρησιμοποιείται
πλέον. Ωστόσο, εξακολουθεί να χρησιμοποιείται για να προσδιορίσει εάν
μια διεργασία εκτελείται σε ένα \textlatin{Docker} κοντέινερ
\cite{Metasploit-Linux-Gather-Container-Detection}
\cite{Removed-Dockerinit-Reference}.


\subsubsection{\textlatin{Control Group}}

Για να περιορίσει τους πόρους των κοντέινερ, το \textlatin{Docker} δημιουργεί
ομάδες ελέγχου για καθένα κοντέινερ και μια ομάδα γονικού ελέγχου
(\textlatin{parent control group}) που ονομάζεται \texttt{\textlatin{docker}}.
Εάν ξεκινήσει μια διαδικασία σε ένα κοντέινερ \textlatin{Docker}, αυτή η
διαδικασία θα πρέπει να βρίσκεται στην ομάδα ελέγχου εκείνου του κοντέινερ.
Μπορούμε να το επαληθεύσουμε αυτό κοιτάζοντας την \texttt{\textlatin{cgroup}}
της αρχικής διεργασίας \texttt{\textlatin{(/proc/1/cgroups)}}
\cite{Metasploit-Linux-Gather-Container-Detection}. \\

\texttt{\textlatin{(cont)\# cat /proc/1/cgroup}}

\texttt{\textlatin{12:hugetlb:/docker/0c7a3b8...}}

\texttt{\textlatin{11:blkio:/docker/0c7a3b8...}}

\texttt{\textlatin{...}}

Αν κοιτάξουμε έναν κεντρικό υπολογιστή, δεν βλέπουμε το ίδιο
\texttt{\textlatin{/docker/}} \textlatin{parent control group}. \\

\texttt{\textlatin{(cont)\# cat /proc/1/cgroup}}

\texttt{\textlatin{12:hugetlb:/}}

\texttt{\textlatin{11:blkio:/}}

\texttt{\textlatin{...}}

Σε ορισμένα συστήματα που χρησιμοποιούν \textlatin{Docker} (π.χ. λογισμικό
ενορχήστρωσης), το \textlatin{parent control group} έχει άλλο όνομα (π.χ.
\texttt{\textlatin{kubepod}} για \textlatin{Kubernetes}).

\subsubsection{\textlatin{Running Processes}}

Τα κοντέινερ είναι κατασκευασμένα για να εκτελούν μόνο μία διεργασία, ενώ τα
κεντρικά συστήματα (\textlatin{hosts}) εκτελούν πολλές διεργασίες. Οι
διεργασίες σε συστήματα κεντρικού υπολογιστή έχουν μία διαδικασία ρίζας
με αναγνωριστικό διεργασίας 1 (\textlatin{process id 1}) που εκκινεί όλες τις
απαραίτητες (\textlatin{child}) διαδικασίες. Στα περισσότερα συστήματα
\textlatin{Linux} η διαδικασία είναι είτε \texttt{\textlatin{init}} είτε
\texttt{\textlatin{systemd}}. Δεν θα βλέπαμε ποτέ \texttt{\textlatin{init}} ή
\texttt{\textlatin{systemd}} σε ένα κοντέινερ, επειδή το κοντέινερ εκτελεί μόνο
μία διαδικασία και όχι πλήρες λειτουργικό σύστημα. Αυτός είναι ο λόγος, για τον
οποίο ο αριθμός των διαδικασιών και της διαδικασίας με το \textlatin{pid} 1
είναι μια καλή ένδειξη εάν τρέχουμε σε κοντέινερ.

\subsubsection{Διαθέσιμες βιβλιοθήκες και \textlatin{binaries}}

Οι εικόνες \textlatin{Docker} κατασκευάζονται όσο το δυνατόν μικρότερες. Πολλές
διαδικασίες δεν χρειάζονται ένα πλήρως λειτουργικό σύστημα \textlatin{Linux},
χρειάζονται μόνο ένα μέρος του. Γι' αυτό oι προγραμματιστές συχνά αφαιρούν
βιβλιοθήκες και \textlatin{binary} αρχεία που δεν χρειάζονται για την
συγκεκριμένη εφαρμογή από τις εικόνες \textlatin{Docker} τους. Αν δούμε πολλά
πακέτα που λείπουν, \textlatin{binaries} ή βιβλιοθήκες είναι μια καλή ένδειξη
ότι τρέχουμε μέσα σε ένα κοντέινερ.

Το πακέτο \texttt{\textlatin{sudo}} είναι ένα παράδειγμα αυτού. Αυτό το πακέτο
είναι ζωτικής σημασίας για πολλές διανομές \textlatin{Linux}, γιατί δίνει τη
δυνατότητα εκτέλεσης σε χρήστες που δεν είναι \texttt{\textlatin{root}} να
εκτελέσουν εντολές ως \texttt{\textlatin{root}}. Ωστόσο, σε ένα κοντέινερ
\textlatin{Docker} το πακέτο \texttt{\textlatin{sudo}} δεν έχει πολύ νόημα. Εάν
μια διεργασία χρειάζεται να τρέξει κάτι ως \texttt{\textlatin{root}}, τότε
η διαδικασία πρέπει να εκτελείται ως \texttt{\textlatin{root}} στο κοντέινερ.
Γι' αυτό το \texttt{\textlatin{sudo}} συχνά δεν είναι εγκατεστημένο στις εικόνες
\textlatin{Docker}.


\subsection{Δοκιμές παρείσδυσης μέσα σε ένα κοντέινερ}

Εάν έχουμε εκτέλεση κώδικα μέσα σε ένα κοντέινερ, θα πρέπει να επικεντρωθούμε
στο σενάριο διαφυγής από το κοντέινερ (βλ. ενότητα 3.1). Επειδή τρέχει ο
\textlatin{Docker daemon} ως \texttt{\textlatin{root}}, πιθανότατα θα
καταφέρουμε να αποκτήσουμε πρόσβαση \texttt{\textlatin{root}} στον κεντρικό
υπολογιστή, εάν ξεφύγουμε από το κοντέινερ. Θα ρίξουμε μια ματιά στα βήματα που
μπορούμε να κάνουμε για να αναγνωρίσουμε το λειτουργικό σύστημα του κοντέινερ, 
την εικόνα του κοντέινερ, το λειτουργικό σύστημα του κεντρικού υπολογιστή καθώς
και τα αδύναμα σημεία του κοντέινερ.

Σε πολλές εικόνες \textlatin{Docker} έχουν αφαιρεθεί τα περιττά εργαλεία,
\textlatin{binary} αρχεία και βιβλιοθήκες, για να γίνει η εικόνα μικρότερη.
Ωστόσο, μπορεί να χρειαστούμε αυτά τα εργαλεία κατά τη διάρκεια μιας δοκιμής
διείσδυσης. Αν είμαστε \texttt{\textlatin{root}} σε ένα κοντέινερ, το πιθανότερο
είναι να  μπορούμε να εγκαταστήσουμε τα απαραίτητα εργαλεία. Αν έχουμε πρόσβαση
μόνο σε μη \texttt{\textlatin{root}} χρήστη, ενδέχεται να μην είναι δυνατή η
εγκατάσταση οποιουδήποτε εργαλείου. Σε αυτή την περίπτωση, θα πρέπει να
δουλέψουμε με ό,τι έχουμε στη διάθεσή μας ή να βρούμε έναν τρόπο να μεταφέρουμε
\textlatin{statically compiled binaries} μέσα στο κοντέινερ.

\subsubsection{Αναγνώριση χρηστών}

Το πρώτο βήμα που πρέπει να κάνουμε είναι να δούμε αν είμαστε προνομιούχος
χρήστης και να αναγωρίσουμε άλλους χρήστες. Μπορούμε να δούμε τον τρέχοντα
χρήστη μας χρησιμοποιώντας το αναγνωριστικό και να δούμε όλους τους χρήστες
κοιτάζοντας το \texttt{\textlatin{/etc/passwd}}. \\

\texttt{\textlatin{(cont)\# id}}

\texttt{\textlatin{uid=0(root) gid=0(root) groups=0(root)}}

\texttt{\textlatin{(cont)\# cat /etc/passwd}}

\texttt{\textlatin{...}}

\texttt{\textlatin{test:x:1000:1000:,,,:/home/test:/bin/bash}} \\

Βλέπουμε ότι ο τρέχων χρήστης μας είναι \texttt{\textlatin{root}} (το
αναγνωριστικό χρήστη είναι 0) και ότι υπάρχουν δύο χρήστες (εκτός από τους
προεπιλεγμένους χρήστες στο \textlatin{Linux}). Από προεπιλογή, τα κοντέινερ
λειτουργούν ως \texttt{\textlatin{root}}. Αυτό είναι υπέροχο από την άποψη των
επιτιθέμενων, γιατί μας επιτρέπει να έχουμε πλήρη πρόσβαση σε οτιδήποτε
βρίσκεται μέσα στο κοντέινερ. Ένα καλά διαμορφωμένο κοντέινερ πιθανότατα δεν
εκτελείται ως \texttt{\textlatin{root}}.

\subsubsection{Αναγνώριση του λειτουργικού συστήματος του κοντέινερ}

Το επόμενο βήμα είναι να προσδιορίσουμε το λειτουργικό σύστημα (και ίσως την
εικόνα \textlatin{Docker}) του κοντέινερ.

Όλες οι σύγχρονες διανομές \textlatin{Linux} έχουν ένα αρχείο
\texttt{\textlatin{/etc/os-release}} \footnote{Παρ'όλο που το αρχείο αυτό
πρωτοεμφανίστηκε από το \texttt{\textlatin{systemd}} τα λειτουργικά συστήματα
που ρητά δεν χρησιμοποιούν το \texttt{\textlatin{systemd}} π.χ
\textlatin{Void Linux} χρησιμοποιούν το \texttt{\textlatin{/etc/os-release}}.}
που περιέχει πληροφορίες σχετικά με το λειτουργικό σύστημα που εκτελείται. \\

\texttt{\textlatin{(host)\$ docker run -it --rm centos:latest cat /etc/os-release}}

\texttt{\textlatin{...}}

\texttt{\textlatin{PRETTY\_NAME="CentOS Linux 8 (Core)"}}

\texttt{\textlatin{...}} \\

Για να έχουμε μια καλύτερη ιδέα για το τι πρέπει να κάνει ένα κοντέινερ,
μπορούμε να δούμε τις διαδικασίες. Επειδή τα κοντέινερ θα πρέπει να έχουν μόνο
μια μοναδική διεργασία (π.χ. που τρέχει μια βάση δεδομένων), που εκτελείται. \\

\texttt{\textlatin{(host)\$ docker run --rm -e MYSQL\_RANDOM\_ROOT\_PASSWORD=true --name=database mariadb:latest}}

\texttt{\textlatin{...}}

\texttt{\textlatin{(host)\$ docker exec database ps -A -o pid,cmd}}

\texttt{\textlatin{PID CMD}}

\texttt{\textlatin{   1 mysqld}}

\texttt{\textlatin{320 ps -A -o pid,cmd}} \\


Σε αυτό το παράδειγμα, βλέπουμε ότι η εικόνα \textlatin{mariadb} έχει μόνο μία
διαδικασία \textlatin{(mysqld)} \footnote{Παρατηρούμε επίσης και τη διαδικασία
που δείχνει όλες τις διαδικασίες με \textlatin{process id 320}}. Με αυτόν τον
τρόπο γνωρίζουμε ότι το κοντέινερ είναι \textlatin{MySQL Server} και είναι
πιθανώς με βάση την προεπιλεγμένη εικόνα \textlatin{MySQL Docker (mariadb)}.

\subsubsection{Αναγνώριση του λειτουργικού συστήματος του κεντρικού υπολογιστή}

Είναι επίσης σημαντικό να αναζητήσουμε πληροφορίες για τον κεντρικό υπολογιστή.
Αυτό μπορεί να είναι χρήσιμο για τον εντοπισμό και τη χρήση σχετικών
εκμεταλλεύσεων.

Επειδή τα κοντέινερ χρησιμοποιούν τον πυρήνα του κεντρικού υπολογιστή, μπορούμε
να χρησιμοποιήσουμε την έκδοση του πυρήνα για αναγνώριση πληροφοριών σχετικά με
τον κεντρικό υπολογιστή. Ας ρίξουμε μια ματιά στο παρακάτω παράδειγμα που
εκτελείται σε έναν κεντρικό υπολογιστή \textlatin{Ubuntu}. \\

\texttt{\textlatin{(host)\$ docker run -it --rm alpine:latest cat /etc/os-release}}

\texttt{\textlatin{...}}

\texttt{\textlatin{PRETTY\_NAME="Alpine Linux v3.10"}}

\texttt{\textlatin{...}}

\texttt{\textlatin{(host)\$ docker run -it --rm alpine:latest uname -rv}}

\texttt{\textlatin{5.0.0-36-generic \#39\~18.04.1-Ubuntu SMP Fri Dec 12 11:09:50 UTC 2019}} \\


Τρέχουμε ένα κοντέινερ \textlatin{Alpine Linux}, το οποίο βλέπουμε όταν κοιτάμε
στο αρχείο \texttt{\textlatin{/etc/os-release}}. Ωστόσο, όταν κοιτάμε την έκδοση
του πυρήνα (χρησιμοποιώντας την εντολή \texttt{\textlatin{uname}}), βλέπουμε ότι
χρησιμοποιούμε έναν πυρήνα \textlatin{Ubuntu}. Αυτό σημαίνει ότι πιθανότατα
τρέχουμε σε έναν κεντρικό υπολογιστή \textlatin{Ubuntu}.

Βλέπουμε επίσης τον αριθμό έκδοσης του πυρήνα (σε αυτήν την περίπτωση
\textlatin{5.0.0-36-generic}). Αυτό μπορεί να χρησιμοποιηθεί για να δούμε εάν το
σύστημα είναι ευάλωτο σε εκμεταλλεύσεις πυρήνα, επειδή ορισμένα
\textlatin{exploits} του πυρήνα μπορούν να χρησιμοποιηθούν για να διαφύγει ένας
επιτιθέμενος από το κοντέινερ.

\subsubsection{Ανάγνωση μεταβλητών περιβάλλοντος}

Οι μεταβλητές περιβάλλοντος είναι ένας τρόπος διαβίβασης πληροφοριών σε
κοντέινερ όταν ξεκινούν. Όταν ξεκινά ένα κοντέινερ, οι μεταβλητές περιβάλλοντος
διαβιβάζονται σε αυτό. Αυτές οι μεταβλητές συχνά περιέχουν κωδικούς πρόσβασης
και άλλες ευαίσθητες πληροφορίες.

Μπορούμε να απαριθμήσουμε όλες τις μεταβλητές περιβάλλοντος που έχουν οριστεί
μέσα σε ένα \textlatin{Docker} κοντέινερ χρησιμοποιώντας την εντολή
\texttt{\textlatin{env}} (ή κοιτάζοντας το αρχείο
\texttt{\textlatin{/proc/pid/environ}} μιας διαδικασίας).\\

\texttt{\textlatin{(host)\$ docker run --rm -e MYSQL\_ROOT\_PASSWORD=supersecret --name=database mariadb:latest}}

\texttt{\textlatin{(host)\$ docker exec -it database bash}}

\texttt{\textlatin{(cont)\# env}}

\texttt{\textlatin{...}}

\texttt{\textlatin{MYSQL\_ROOT\_PASSWORD=supersecret}}

\texttt{\textlatin{...}}\\

Πρέπει να σημειωθεί ότι δεν πρόκειται για εσφαλμένη διαμόρφωση. Η χρήση
μεταβλητών περιβάλλοντος είναι ο επιδιωκόμενος τρόπος για να διαβιβαστούν
ευαίσθητες πληροφορίες σε ένα \textlatin{Docker} κοντέινερ κατά το χρόνο
εκτέλεσης. Ωστόσο, κατά τη διάρκεια μιας δοκιμής διείσδυσης τύπου
\textlatin{blackbox}, οι ευαίσθητες πληροφορίες που είναι αποθηκευμένες στις
μεταβλητές περιβάλλοντος μπορεί να είναι χρήσιμες.

\subsubsection{Έλεγχος δυνατοτήτων}


Μόλις έχουμε μια σαφή εικόνα με τι είδους σύστημα εργαζόμαστε, θα μπορέσουμε να
κάνουμε πιο λεπτομερή αναγνώριση. Ένα από τα πιο σημαντικά πράγματα που πρέπει
να εξετάσουμε είναι οι δυνατότητες του πυρήνα του κοντέινερ. Εμείς μπορούμε να
το ελέγξουμε αυτό, κοιτάζοντας το \texttt{\textlatin{/proc/self/status}}
\footnote{Το \texttt{\textlatin{self}} στο \texttt{\textlatin{/proc/self}}
αναφέρεται στην τρέχουσα διαδικασία.}. Αυτό το αρχείο περιέχει πολλές γραμμές που
περιέχουν πληροφορίες σχετικά με τις χορηγούμενες δυνατότητες. \\

\texttt{\textlatin{(cont)\# grep Cap /proc/self/status}}

\texttt{\textlatin{CapInh: 00000000a80425fb}}

\texttt{\textlatin{CapPrm: 00000000a80425fb}}

\texttt{\textlatin{CapEff: 00000000a80425fb}}

\texttt{\textlatin{CapBnd: 00000000a80425fb}}

\texttt{\textlatin{CapAmb: 0000000000000000}} \\

Βλέπουμε πέντε διαφορετικές τιμές που περιγράφουν τις δυνατότητες της
διαδικασίας:

• \textlatin{\textbf{CapInh}}: Οι κληρονομητές δυνατότητες είναι οι δυνατότητες που
μπορεί να πάρει μία \textlatin{child} διαδικασία.

• \textlatin{\textbf{CapPrm}}: Οι επιτρεπόμενες δυνατότητες είναι οι μέγιστες
δυνατότητες που μια διαδικασία μπορεί να χρησιμοποιήσει.

• \textlatin{\textbf{CapEff}}: Οι δυνατότητες που έχει η διαδικασία.

• \textlatin{\textbf{CapBnd}}: Οι δυνατότητες που επιτρέπονται στο δέντρο
κλήσεων.

• \textlatin{\textbf{CapAmb}}: Δυνατότητες που μπορούν να κληρονομήσουν οι
\textlatin{non-root} διαδικασίες. \\


Μας ενδιαφέρει η τιμή \texttt{\textlatin{CapEff}}, γιατί αυτή η τιμή
αντιπροσωπεύει τις τρέχουσες δυνατότητες. Οι δυνατότητες αντιπροσωπεύονται ως
δεκαεξαδική τιμή. Κάθε δυνατότητα έχει μια τιμή και η τιμή
\texttt{\textlatin{CapEff}} είναι ο συνδυασμός
των τιμών των χορηγηθεισών δυνατοτήτων. Μπορούμε να χρησιμοποιήσουμε το εργαλείο
\textlatin{capsh} για να λάβουμε την λίστα δυνατοτήτων από μια δεκαεξαδική τιμή
(αυτό μπορεί να εκτελεστεί σε διαφορετικό σύστημα). \\

\texttt{\textlatin{(host)\$ capsh --decode=00000000a80425fb}}

\texttt{\textlatin{0x00000000a80425fb=cap\_chown,cap\_dac\_override,cap\_fowner,}}

\texttt{\textlatin{cap\_fsetid,cap\_kill,cap\_setgid,cap\_setuid,cap\_setpcap,}}

\texttt{\textlatin{cap\_net\_bind\_service,cap\_net\_raw,cap\_sys\_chroot,}}

\texttt{\textlatin{cap\_mknod,cap\_audit\_write,cap\_setfcap}} \\


Μπορούμε να το χρησιμοποιήσουμε το πρόγραμμα αυτό για να ελέγξουμε αν υπάρχουν
δυνατότητες που μπορούν να χρησιμοποιηθούν για να διαφύγουμε από το κοντέινερ
\textlatin{Docker} (βλ. ενότητα 4.1.4).


\subsubsection{Έλεγχος για προνομιακή λειτουργία (\textlatin{Privileged Mode})}

Όπως αναφέρθηκε προηγουμένως, εάν το κοντέινερ τρέχει σε προνομιακή
λειτουργία, έχει όλες τις δυνατότητες. Αυτό καθιστά εύκολο να ελέγξουμε αν
εκτελούμε μια διαδικασία σε ένα κοντέινερ με προνομιακή λειτουργία.
Το \texttt{\textlatin{0000003ffffffffff}} είναι η αναπαράσταση όλων των
δυνατοτήτων. \\

\texttt{\textlatin{(host)\$ docker run -it --rm --privileged ubuntu:latest grep CapEff /proc/1/status}}

\texttt{\textlatin{CapEff: 0000003fffffffff}}

\texttt{\textlatin{(host)\$ capsh --decode=0000003fffffffff}}

\texttt{\textlatin{0x0000003fffffffff=cap\_chown,cap\_dac\_override,}}

\texttt{\textlatin{cap\_dac\_read\_search,cap\_fowner,cap\_fsetid,cap\_kill,}}

\texttt{\textlatin{cap\_setgid,cap\_setuid,cap\_setpcap,cap\_linux\_immutable,}}

\texttt{\textlatin{cap\_net\_bind\_service,cap\_net\_broadcast,cap\_net\_admin,}}

\texttt{\textlatin{cap\_net\_raw,cap\_ipc\_lock,cap\_ipc\_owner,cap\_sys\_module,}}

\texttt{\textlatin{cap\_sys\_rawio,cap\_sys\_chroot,cap\_sys\_ptrace,cap\_sys\_pacct,}}

\texttt{\textlatin{cap\_sys\_admin,cap\_sys\_boot,cap\_sys\_nice,cap\_sys\_resource,}}

\texttt{\textlatin{cap\_sys\_time,cap\_sys\_tty\_config,cap\_mknod,cap\_lease,}}

\texttt{\textlatin{cap\_audit\_write,cap\_audit\_control,cap\_setfcap,}}

\texttt{\textlatin{cap\_mac\_override,cap\_mac\_admin,cap\_syslog,cap\_wake\_alarm,}}

\texttt{\textlatin{cap\_block\_suspend,cap\_audit\_read}} \\

Αν βρούμε ένα προνομιακό κοντέινερ, μπορούμε εύκολα να ξεφύγουμε από αυτό (όπως
φαίνεται στην ενότητα 4.1.3).

\subsubsection{Έλεγχος \textlatin{Volumes}}

\textlatin{Volumes}, δηλαδή οι κατάλογοι που είναι προσαρτημένοι από τον
κεντρικό υπολογιστή στο κοντέινερ, είναι τα μόνιμα δεδομένα του κοντέινερ. Αυτά
τα μόνιμα δεδομένα ενδέχεται να περιέχουν ευαίσθητες πληροφορίες, γι' αυτό είναι
σημαντικό να ελέγξουμε ποιοί καταλόγοι τοποθετούνται στο κοντέινερ (βλ.
ενότητα 2.2.3).

Μπορούμε να το κάνουμε αυτό, κοιτάζοντας τις προσαρτημένες θέσεις του συστήματος
αρχείων. \\

\texttt{\textlatin{(host)\$ docker run -it --rm -v /tmp:/host/tmp ubuntu cat /proc/mounts}}

\texttt{\textlatin{overlay / overlay...}}

\texttt{\textlatin{...}}

\texttt{\textlatin{/dev/mapper/ubuntu--vg-root /host/tmp...}}

\texttt{\textlatin{/dev/mapper/ubuntu--vg-root /etc/resolv.conf...}}

\texttt{\textlatin{/dev/mapper/ubuntu--vg-root /etc/hostname ext4...}}

\texttt{\textlatin{/dev/mapper/ubuntu--vg-root /etc/hosts...}}

\texttt{\textlatin{...}}



Κάθε γραμμή περιέχει πληροφορίες για μία προσάρτηση.
Βλέπουμε τη \textlatin{root} προσάρτηση \textlatin{OverlayFS} στην κορυφή και σε
ποια διαδρομή δείχνει στον κεντρικό υπολογιστή. Βλέπουμε επίσης ποιοι κατάλογοι
είναι προσαρτημένοι από το \textlatin{root file system} στον κεντρικό υπολογιστή
(που σε αυτήν την περίπτωση είναι το \textlatin{LVM logical volume root} που
αναπαριστάται στο σύστημα αρχείων ως
\texttt{\textlatin{/dev/mapper/ubuntu--vg-root}}). Στην εντολή μπορούμε να δούμε
ότι το \texttt{\textlatin{/tmp}} του κεντρικού υπολογιστή προσαρτάται ως
\texttt{\textlatin{/host/tmp}} στο κοντέινερ και το 
\texttt{\textlatin{/host/tmp}} έχει προσαρτηθεί στο
\texttt{\textlatin{/proc/mounts}}. Δυστυχώς δεν βλέπουμε σε ποιο μονοπάτι ο
κεντρικός υπολογιστής είναι προσαρτηθεί, βλέπουμε μόνο το μονοπάτι μέσα στο
κοντέινερ.

Τώρα γνωρίζουμε ότι αυτό είναι ένα ενδιαφέρον μονοπάτι, γιατί το περιεχόμενό του
πρέπει να είναι \textlatin{persistent}. Κατά τη διάρκεια μιας δοκιμής
διείσδυσης, αυτός θα ήταν ένας κατάλογος που αξίζει να δώσουμε επιπλέον
προσοχή.


\subsubsection{Έλεγχος για ένα προσαρτημένο \textlatin{Docker Socket}}


Είναι αρκετά σύνηθες το \textlatin{Docker Socket} να προσαρτάται σε κοντέινερ.
Για παράδειγμα, στη περίπτωση που θέλουμε να έχουμε ένα κοντέινερ που να
παρακολουθεί την υγεία όλων των άλλων κοντέινερς. Ωστόσο, αυτό είναι επικίνδυνο
όπως αναλύεται στην ενότητα 4.1.5. Μπορούμε να αναζητήσουμε το
\textlatin{socket} χρησιμοποιώντας δύο τεχνικές. Είτε κοιτάμε τις προσαρτήσεις
ή προσπαθούμε να αναζητήσουμε αρχεία με παρόμοια ονόματα με το
\texttt{\textlatin{docker.sock}}. \\

\texttt{\textlatin{(host)\$ docker run -it --rm -v /var/run/docker.sock:/var/run/}}

\texttt{\textlatin{docker.sock ubuntu grep docker.sock /proc/mounts}}

\texttt{\textlatin{tmpfs /run/docker.sock tmpfs rw,nosuid,noexec,relatime,size=792244k,mode=755 0 0}} \\

Στο ανωτέρω παράδειγμα προσαρτούμε το \texttt{\textlatin{/var/run/docker.sock}}
στο κοντέινερ ως \texttt{\textlatin{/var/run/docker.sock}} και κοιτάμε το
\texttt{\textlatin{/proc/mounts}}. Μπορούμε να δούμε ότι το το
\texttt{\textlatin{docker.sock}} είναι προσαρτημένο στο
\texttt{\textlatin{/run/docker.sock}} (δεν είναι στην πραγματικότητα
προσαρτημένο στο \texttt{\textlatin{/var/run/docker.sock}} γιατί το
\texttt{\textlatin{/var/run/}} είναι ένας συμβολικός σύνδεσμος στο
\texttt{\textlatin{/run/}}). \\

\texttt{\textlatin{(host)\$ docker run -it --rm -v /var/run/docker.sock:/var/run/}}

\texttt{\textlatin{docker.sock ubuntu find . -name "docker.sock" /}} \\

Εδώ προσαρτούμε το \texttt{\textlatin{/var/run/docker.sock}} στο κοντέινερ και
ψάχνουμε αρχεία που ονομάζονται "\texttt{\textlatin{docker.sock}}".

\subsubsection{Έλεγχος ρυθμίσεων δικτύου}

Θα πρέπει επίσης να δούμε το δίκτυο του κοντέινερ. Θα πρέπει να κοιτάξουμε
ποια κοντέινερ βρίσκονται στο ίδιο δίκτυο και τι είναι ικανό το κοντέινερ
να φτάσει από πλευράς δικτύου. Για να γίνει αυτό, πιθανότατα θα χρειαστεί να
εγκαταστήσουμε κάποια εργαλεία. Ακόμη και τα πιο βασικά εργαλεία δικτύωσης
(π.χ. \texttt{\textlatin{ping}}) αφαιρούνται από τα περισσότερα
\textlatin{Docker images}, γιατί λίγα κοντέινερ θα τα χρειαστούν.


Από προεπιλογή, όλα τα κοντέινερ λαμβάνουν μια διεύθυνση \textlatin{IPv4} στο
υποδίκτυο 172.17.0.0/16. Είναι δυνατό να βρούμε τη διεύθυνση (χωρίς να
εγκαταστήσ τίποτα) ενός κοντέινερ στο οποίο έχουμε πρόσβαση κοιτάζοντας το
αρχείο \texttt{\textlatin{/etc/hosts/}}. Το \textlatin{Docker} θα προσθέσει μια
γραμμή που κάνει \textlatin{resolve} το όνομα του κεντρικού υπολογιστή στη
διεύθυνση \textlatin{IPv4} σε \texttt{\textlatin{/etc/hosts}}. \\

\texttt{\textlatin{(host)\$ docker run -it --rm alpine tail -n1 /etc/hosts}}

\texttt{\textlatin{172.17.0.2 e0e6b96367db}} \\

Μπορούμε να δούμε το δίκτυο \textlatin{Docker} χρησιμοποιώντας το
\texttt{\textlatin{nmap}} (το οποίο θα πρέπει να εγκαταστήσουμε μόνοι μας). \\

\texttt{\textlatin{(host)\$ docker run -it --rm ubuntu bash}}

\texttt{\textlatin{(cont)\# apt update}}

\texttt{\textlatin{...}}

\texttt{\textlatin{(cont)\# apt install nmap}}

\texttt{\textlatin{...}}

\texttt{\textlatin{(cont)\# nmap -sn -PE 172.17.0.0/16}}

\texttt{\textlatin{...}}

\texttt{\textlatin{Nmap scan report for 172.17.0.1}}

\texttt{\textlatin{Host is up (0.000044s latency).}}

\texttt{\textlatin{MAC Address: 02:42:5F:92:ED:72 (Unknown)}}

\texttt{\textlatin{Nmap scan report for 172.17.0.3}}
 
\texttt{\textlatin{Host is up (0.000027s latency).}}

\texttt{\textlatin{MAC Address: 02:42:AC:11:00:03 (Unknown)}} \\

Βλέπουμε ότι μπορούμε να φτάσουμε σε δύο κοντέινερ, 172.17.0.1 και 172.17.0.2.
Η πρώτη διεύθυνση είναι ο ίδιος ο κεντρικός υπολογιστής και η δεύτερη ένα άλλο
\textlatin{docker}. Είναι δυνατό να καταγραφεί η κίνηση αυτού του κοντέινερ
χρησιμοποιώντας μια επίθεση \textlatin{ARP man-in-the-middle} (βλ.
ενότητα 4.1.7).

\subsection{Δοκιμές διείσδυσης σε ένα κεντρικό υπολογιστή που τρέχει
\textlatin{Docker}}

Κατά τη δοκιμή διείσδυσης ενός κεντρικού υπολογιστή με εγκατεστημένο το
\textlatin{Docker} σε αυτόν, μας ενδιαφέρει να βρούμε σφάλματα και εσφαλμένες
διαμορφώσεις που μας επιτρέπουν να χρησιμοποιούμε το \textlatin{Docker} για
πρόσβαση σε ευαίσθητα δεδομένα ή να κλιμακώσουμε τα προνόμιά μας εντός του
συστήματος. Σε αυτή την ενότητα θα εξετάσουμε τα διάφορα βήματα που μπορούμε να
κάνουμε για να συγκεντρώσουμε πληροφορίες σχετικά με το σύστημα και τη
διαμόρφωση του \textlatin{Docker}. Αυτό θα μας πει αν είναι δυνατό να
εκτελέσουμε ένα \textlatin{Docker daemon} (βλ. ενότητα 3.2).

\subsubsection{Έκδοση \textlatin{Docker}}

Το πρώτο βήμα που κάνουμε εάν δοκιμάζουμε ένα σύστημα που έχει εγκαταστεστημένο
το \textlatin{Docker}, είναι να ελέγξουμε την έκδοση. Το \textlatin{Docker} δεν
χρειάζεται να τρέχει και εμείς δεν χρειαζόμαστε ειδικές άδειες (δηλαδή άδειες
\textlatin{Docker}) για να ελέγξουμε την έκδοση του \textlatin{Docker}
\footnote{Η έκδοση είναι \textlatin{hardcoded} ως συμβολοσειρά μέσα στο
\textlatin{Docker client binary}}. \\

\texttt{\textlatin{(host)\$ docker -v}}

\texttt{\textlatin{Docker version 20.10.8, build 3967b7d}} \\

Μόλις βρούμε την έκδοση \textlatin{Docker}, θα πρέπει να ελέγξουμε για τυχόν
\textlatin{CVEs} που είναι διαθέσιμα για την έκδοση του \textlatin{Docker} που
είναι εγκατεστημένη στον κεντρικό υπολογιστή.

\subsubsection{Ποιός επιτρέπεται να χρησιμοποιήσει το \textlatin{Docker}}

Επειδή η πρόσβαση στο \textlatin{Docker} ισοδυναμεί με δικαιώματα
\texttt{\textlatin{root}}, οι χρήστες που επιτρέπεται να χρησιμοποιούν το
\textlatin{Docker} είναι ενδιαφέροντες στόχοι. Αν υπάρχει ένας τρόπος να γίνουμε
ένας από αυτούς τους χρήστες, θα έχουμε πρόσβαση στα πάντα στον κεντρικό
υπολογιστή.

Κάθε χρήστης που έχει πρόσβαση ανάγνωσης και εγγραφής στο
\textlatin{Docker socket} (δηλαδή το \texttt{\textlatin{/var/run/docker.sock}})
έχει δικαιώματα χρήσης του \textlatin{Docker}.

Γι' αυτό το πρώτο πράγμα που πρέπει να κάνουμε είναι να δούμε ποιοι χρήστες
έχουν δικαιώματα ανάγνωσης και εγγραφής στο \textlatin{Docker socket}.

Από προεπιλογή, ο χρήστης \texttt{\textlatin{root}} και κάθε χρήστης στην ομάδα
\texttt{\textlatin{docker}} έχει δικαιώματα ανάγνωσης και εγγραφής στο
\textlatin{Docker socket}.

Μπορούμε να δούμε ποιος ανήκει στην ομάδα \texttt{\textlatin{docker}}
κοιτάζοντας στο \textlatin{/etc/group}. \\

\texttt{\textlatin{\$ grep docker /etc/group}}

\texttt{\textlatin{docker:x:999:milouk}} \\

Βλέπουμε ότι μόνο ο \texttt{\textlatin{milouk}} είναι μέλος της ομάδας
\texttt{\textlatin{docker}}. Μπορεί επίσης να έχει ενδιαφέρον να δούμε ποιοι
χρήστες έχουν δικαιώματα \texttt{\textlatin{sudo}} (στο
\texttt{\textlatin{/etc/sudoers}}). Χρήστες χωρίς \texttt{\textlatin{sudo}} αλλά
με δικαιώματα \textlatin{Docker} πρέπει να εκλαμβάνονται ως χρήστες
\texttt{\textlatin{sudo}} (βλ. ενότητα 4.1.1).

Είναι επίσης πιθανό το \texttt{\textlatin{setuid bit}} να έχει οριστεί στο
\textlatin{Docker client}. Σε αυτή τη περίπτωση, μπορούμε επίσης να
χρησιμοποιήσουμε το \textlatin{Docker}. \\

\texttt{\textlatin{(host)\$ ls -l \$(which docker)}}

\texttt{\textlatin{-rwxr-xr-x 1 root root 88965248 dec 13 08:28 /usr/bin/docker}}

\texttt{\textlatin{(host)\# chmod +s \$(which docker)}}

\texttt{\textlatin{(host)\$ ls -l \$(which docker)}}

\texttt{\textlatin{-rwsr-sr-x 1 root root 88965248 dec 13 08:28 /usr/bin/docker}} \\

\subsubsection{Ρυθμίσεις (\textlatin{Configuration})}

Το \textlatin{Docker} για να ρυθμιστεί χρησιμοποιεί πολλά αρχεία. Το πιο
σημαντικό είναι ο τρόπος με τον οποίον ο \textlatin{Docker daemon} ξεκινά. Τα
περισσότερα συστήματα θα έχουν κάποιον \textlatin{service manager} που
διαχειρίζεται \textlatin{daemon processes}. Σε πολλές σύγχρονες διανομές
\textlatin{Linux} αυτό είναι έργο του \texttt{\textlatin{systemd}}. Σε άλλα
συστήματα \textlatin{Linux} χρησιμοποιείται το αρχείο διαμόρφωσης
\texttt{\textlatin{/etc/docker/daemon.json}}
\footnote{\textlatin{https://docs.docker.com/engine/reference/commandline/dockerd/}}
(και οι προεπιλογές ενδέχεται να οριστούν στο
\texttt{\textlatin{/etc/default/docker}}). Αυτά τα αρχεία θα μας πουν επίσης εάν
το \textlatin{Docker API} είναι διαθέσιμο μέσω \textlatin{TCP} το οποίο, εάν δεν
ρυθμιστεί σωστά, μπορεί να είναι επικίνδυνο.

Μπορούμε επίσης να αναζητήσουμε αρχεία ρυθμίσεων χρήστη, που μπορεί να περιέχουν
μυστικά και ευαίσθητα δεδομένα. Δείτε την ενότητα 4.1.2 για περισσότερες
πληροφορίες.

\subsubsection{Διαθέσιμες εικόνες και κοντέινερς}

Πρέπει να ελέγξουμε ποιες εικόνες και κοντέινερ (τόσο σε λειτουργία όσο και
σταματημένα) είναι διαθέσιμα στον κεντρικό υπολογιστή. Αυτό θα μας πει
περισσότερα για το σύστημα στο οποίο κάνουμε δοκιμές.

Η εντολή \texttt{\textlatin{docker images -a}} θα εμφανίσει όλες τις διαθέσιμες
εικόνες (συμπεριλαμβανομένων των ενδιάμεσων εικόνων) και το
\texttt{\textlatin{docker ps -a}} θα εμφανίσει όλα τα κοντέινερ (σε λειτουργία
και μη). \\

\texttt{\textlatin{(host)\$ docker images -a}}

\texttt{\textlatin{REPOSITORY \hspace{8em} TAG \hspace{1em} IMAGE ID \hspace{1em} CREATED \hspace{1em} SIZE}}

\texttt{\textlatin{grnet/dilosi\_nginx\_dev \hspace{2em} latest 982db5a3ca83 5 hours ago 141MB}}

\texttt{\textlatin{grnet/dilosi\_backend\_dev \hspace{1em} latest 7eb1820b06e0 5 hours ago 1.46GB}}

\texttt{\textlatin{grnet/dilosi\_frontend\_dev \hspace{0.5em} latest 406ee25a3dfb 5 hours ago 1.6GB}} 

\texttt{\textlatin{(host)\$ docker ps -a --no-trunc --format="\{\{.Names\}\} \{\{.Command\}\} \{\{.Image\}\}"}} 

\texttt{\textlatin{dilosi\_nginx\_dev "/docker-entrypoint.sh nginx -g 'daemon off;'" grnet/dilosi\_nginx\_dev:latest}} \\ 


Θα πρέπει επίσης να εξετάσουμε τις μεταβλητές περιβάλλοντος που έχουν
διαβιβαστεί στα κοντέινερ, επειδή οι μεταβλητές περιβάλλοντος χρησιμοποιούνται
για τη διαβίβαση πληροφοριών (συμπεριλαμβανομένων κωδικών πρόσβασης και
μυστικών) σε ένα κοντέινερ κατά τη δημιουργία του. Χρησιμοποιώντας το
\texttt{\textlatin{docker inspect}} μπορούμε να δούμε πληροφορίες σχετικά με τα
κοντέινερ, συμπεριλαμβανομένων των καθορισμένων μεταβλητών περιβάλλοντος. \\

\texttt{\textlatin{(host)\$ docker run --rm -e MYSQL\_ROOT\_PASSWORD=supersecret --name=database mariadb:latest}}

\texttt{\textlatin{(host)\$ docker inspect database | jq -r '.[0].Config.Env'}}

\texttt{\textlatin{[}}

\texttt{\textlatin{"MYSQL\_ROOT\_PASSWORD=supersecret",}}

\texttt{\textlatin{...}} \\

Τα κοντέινερς μπορεί να έχουν \textlatin{volumes}. Αυτοί οι τόμοι μας λένε
περισσότερα σχετικά με το που μπορεί να βρίσκονται ευαίσθητα και σημαντικά
δεδομένα. Μπορούμε επίσης να παραθέσουμε τους τόμους χρησιμοποιώντας το
\texttt{\textlatin{docker inspect}}. \\

\texttt{\textlatin{(host)\$ docker inspect database | jq -r '.[0].HostConfig.Binds'}}

\texttt{\textlatin{[}}

\texttt{\textlatin{"/tmp/database/:/var/lib/mysql/"}}

\texttt{\textlatin{]}} \\


\subsubsection{Κανόνες \texttt{\textlatin{iptables}}}

Όπως είδαμε στην ενότητα 4.1.6, το \textlatin{Docker} θα παρακάμψει τους κανόνες
\texttt{\textlatin{iptables}} του κεντρικού υπολογιστή.
Χρησιμοποιώντας τις εντολές \texttt{\textlatin{iptables -vnL}} και
\texttt{\textlatin{iptables -t nat -vnL}}  μπορούμε να δούμε τους κανόνες από
τους προεπιλεγμένους πίνακες \textlatin{filter} και  \textlatin{nat} αντίστοιχα.
Είναι σημαντικό ότι όλοι οι κανόνες του τείχους προστασίας σχετικά με τα
κοντέινερ \textlatin{Docker} ορίζονται στην αλυσίδα \textlatin{DOCKER-USER}
στο φίλτρο, επειδή όλη η κίνηση του \textlatin{Docker} θα περάσει πρώτα την
αλυσίδα \textlatin{DOCKER-USER}.

\section{Εργαλεία αυτοματοποίησης}

Οι περισσότερες αξιολογήσεις ασφαλείας είναι χρονικά περιορισμένες. Χρειάζονται
μεγάλα και πολύπλοκα συστήματα, προκειμένου να αξιολογηθεί ένα σύστημα σε
σύντομο χρονικό διάστημα. Υπάρχουν εργαλεία που αυτοματοποιούν το μεγαλύτερο
κομμάτι της διαδικασίας αξιολόγησης. Αντί να κάνουμε κάθε βήμα χειροκίνητα, αυτά
τα εργαλεία σαρώνουν τα συστήματα αυτόματα και συστηματικά για να βρουν γνωστά
τρωτά σημεία και πιθανά αδύνατα σημεία. Σε αυτή την ενότητα θα συζητήσουμε τα
εργαλεία που μας βοηθούν να αυτοματοποιήσουμε μέρος των χειροκίνητων βημάτων της
ενότητας 5.1 για την αναγνώριση και εκμεταλλευση των τρωτών σημείων που
συζητήσαμε στο κεφάλαιο 4.

Όπως θα δούμε, τα περισσότερα εργαλεία έχουν μια συγκεκριμένη εστίαση (π.χ. μια
μεμονωμένη ευπάθεια ή μέρος ενός συστήματος). Αυτό καθιστά πιο δύσκολο τον
διαχωρισμό τους σε ομάδες που αντιστοιχούν στα μοντέλα επίθεσης του κεφαλαίου 3.
Αντίθετα, τα χωρίζουμε σε ομάδες που ταιριάζουν με το σκοπό τους: σαρωτές
διαμόρφωσης κεντρικού υπολογιστή (\textlatin{host configuration scanners})
(ενότητα 5.2.1), εργαλεία ανάλυσης εικόνας \textlatin{Docker} (ενότητα 5.2.2)
και εργαλεία εκμετάλλευσης (ενότητα 5.2.3).

\subsection{\textlatin{Host configuration Scanners}}

Τα εργαλεία που περιγράφονται σε αυτήν την ενότητα εκτελούνται σε έναν κεντρικό
υπολογιστή που εκτελεί το \textlatin{Docker} (βλ ενότητα 5.1.3). Ελέγχουν για
προβλήματα στη διαμόρφωση του \textlatin{Docker}, στις διαθέσιμες εικόνες και
στα διαθέσιμα κοντέινερ.

\subsubsection{\textlatin{Docker Bench for Security}}

Το ίδιο το \textlatin{Docker} έχει κυκλοφορήσει έναν σαρωτή (που ονομάζεται
\textlatin{Docker Bench for Security}
\footnote{\textlatin{https://github.com/docker/docker-bench-security}}) που
βασίζεται στο \textlatin{CIS Docker Benchmark}. Προορίζεται για να εκτελείται σε
έναν κεντρικό υπολογιστή που τρέχει \textlatin{Docker}. Ο σαρωτής ελέγχει εάν η
διαμόρφωση \textlatin{Docker}, οι εικόνες και τα κοντέινερς στον κεντρικό
υπολογιστή ακολουθούν όλες τις οδηγίες του \textlatin{CIS Docker Benchmark}.
Ορισμένες οδηγίες μπορεί να είναι άσχετες με κάθε κεντρικό υπολογιστή (π.χ.
οδηγίες που σχετίζονται με το \textlatin{Docker Swarm}). Αυτές παραλείπονται από
το \textlatin{Docker Bench for Security}.

\subsubsection{\textlatin{Dockscan}}

Το \textlatin{Dockscan} \footnote{\textlatin{https://github.com/kost/dockscan}}
ελέγχει έναν κεντρικό υπολογιστή και τα κοντέινερ που τρέχουν, για εσφαλμένες
διαμορφώσεις (δεν σχετίζεται κάθε εσφαλμένη ρύθμιση παραμέτρων με την ασφάλεια).
Είναι αρκετά παλιό (η τελευταία αλλαγή έγινε τον Αύγουστο του 2016
\footnote{\textlatin{https://github.com/kost/dockscan/commit/0a21d64}}) και ως
εκ τούτου είναι λιγότερο χρήσιμο κατά τη διάρκεια μιας δοκιμής διείσδυσης. Το
\textlatin{Dockscan} σαρώνει για τις ακόλουθες εσφαλμένες διαμορφώσεις:

• Τον αριθμό των αλλαγμένων αλλά όχι των μόνιμων αρχείων των κοντέινερ που
τρέχουν.

• Τους κενούς κωδικούς πρόσβασης σε κοντέινερ (παρόμοια με την ενότητα 4.2.4).

• Τον αριθμό των διεργασιών που εκτελούνται μέσα σε ένα κοντέινερ.

• Εάν ένας διακομιστής \textlatin{SSH} εκτελείται μέσα σε ένα κοντέινερ.

• Εάν έχει εγκατασταθεί μια μη σταθερή έκδοση του \textlatin{Docker}.

• Τη χρήση μη ασφαλών \textlatin{registries}.

• Εάν έχουν τεθεί όρια μνήμης για κοντέινερ.

• Εάν η προώθηση κίνησης \textlatin{IPv4} είναι ενεργοποιημένη στο
\textlatin{Docker}.

• Εάν χρησιμοποιείται κάποιο \textlatin{mirror registry}.

• Εάν χρησιμοποιείται το \textlatin{AUFS storage driver}.

\subsection{Εργαλεία ανάλυσης εικόνων \textlatin{Docker}}

Τα περισσότερα εργαλεία ανάλυσης ασφάλειας \textlatin{Docker} επικεντρώνονται
στη στατική ανάλυση εικόνων \textlatin{Docker}. Αναζητούν λογισμικό και
βιβλιοθήκες μέσα στις εικόνες και τα ταιριάζουν έναντι γνωστών βάσεων δεδομένων
ευπαθειών. Ορισμένοι ερευνητές αναζητούν επίσης ευαίσθητες πληροφορίες
(π.χ. κωδικούς πρόσβασης) που μπορεί να είναι αποθηκευμένες μέσα στην εικόνα.

Αν και αυτά τα εργαλεία είναι πιο χρήσιμα από αμυντική άποψη (π.χ.
σάρωση εικόνων για προβλήματα προτού αναπτυχθούν), ενδέχεται να αποκαλυφθούν
τρωτά σημεία ή ευαίσθητες πληροφορίες κατά τη διάρκεια μιας δοκιμής διείσδυσης.


\subsection{Εργαλεία εκμετάλλευσης}

Υπάρχουν εργαλεία που εστιάζουν ειδικά στην εκμετάλλευση των τρωτών σημείων.
Αυτά τα εργαλεία επικεντρώνονται στη διαφυγή κοντέινερ ή στην κλιμάκωση των
προνομίων στον κεντρικό υπολογιστή. Μπορούν να είναι χρήσιμα κατά τη διάρκεια
μιας δοκιμής διείσδυσης, επειδή θα αυτοματοποιήσουν την εκμετάλλευση
συγκεκριμένων τρωτών σημείων.

\subsubsection{\textlatin{Break out of the Box}}

Το \textlatin{Break Out of the Box (BOtB)}
\footnote{\textlatin{https://github.com/brompwnie/botb}} είναι ένα εργαλείο που
εντοπίζει και εκμεταλλεύεται κοινά τρωτά σημεία διαφυγής κοντέινερ. Είναι σε
θέση να κάνει τις ακόλουθες αποδράσεις:

• Εάν το \textlatin{BOtB} βρει το \textlatin{Docker socket} τοποθετημένο μέσα
στο κοντέινερ, το \textlatin{BOtB} μπορεί να ξεφύγει από το κοντέινερ
χρησιμοποιώντας την ίδια τεχνική που συζητήσαμε στην ενότητα 4.1.5.

• Το \textlatin{BOtB} μπορεί να διαφύγει από κοντέινερ χρησιμοποιώντας το
\textlatin{CVE-2019-5736} (βλ. ενότητα 4.2.3).

• Το \textlatin{BOtB} είναι σε θέση να αναγνωρίζει ευαίσθητες πληροφορίες σε
μεταβλητές περιβάλλοντος.

• Εάν το κοντέινερ λειτουργεί σε προνομιακή λειτουργία, το BOtB προσπαθεί να
διαφύγει χρησιμοποιώντας την ίδια ευπάθεια \footnote{Να σημειώσουμε ότι η
προνομιακή λειτουργία δεν χρειάζεται για να δουλέψει αυτή η διαφυγή κοντέινερ}.
\cite{TrailOfBits-Docker-Escape}.

\subsubsection{\textlatin{Metasploit}}

Το \textlatin{Metasploit} \footnote{\textlatin{https://www.metasploit.com/}}
είναι ένα \textlatin{framework} εκμετάλλευσης (όχι μόνο για το
\textlatin{Docker}). Έχει όμως κάποια \textlatin{modules} ειδικά για το
 \textlatin{Docker}:

• Το \textlatin{module} "\textlatin{Linux Gather Container Detection}" ελέγχει
εάν τρέχει μέσα σε ένα κοντέινερ (παρόμοια με τους ελέγχους που εξετάσαμε στην
ενότητα 5.1.1) \cite{Metasploit-Linux-Gather-Container-Detection}

• Το \textlatin{module} "\textlatin{Multi Gather Docker Credentials Collection}"
μαζεύει όλα τα αρχεία \texttt{\textlatin{.docker/config.json}} που βρίσκονται
στα \textlatin{home directories} των χρηστών.
\cite{More-Secure-Non-Root-Container}

• Το \textlatin{module} "\textlatin{Unprotected TCP Socket Exploit}" αποκτά
πρόσβαση \texttt{\textlatin{root}} σε κάποιο απομακρυσμένο κεντρικό υπολογιστή
που εκθέτει το \textlatin{Docker API} του μέσω \textlatin{TCP} δημιουργώντας ένα
κοντέινερ με το σύστημα αρχείων του κεντρικού υπολογιστή προσαρτημένο ως τόμο
(βλ. ενότητα 4.1.5) \cite{Metasploit-Unprotected-TCP-Socket}.

\subsubsection{\textlatin{Harpoon}}

Το \textlatin{Harpoon}
\footnote{\textlatin{https://github.com/ProfessionallyEvil/harpoon}} είναι ένα
εργαλείο που μπορεί να αναγνωρίσει εάν τρέχει μέσα σε ένα κοντέινερ
κοιτάζοντας το \texttt{\textlatin{cgroup}} και προσπαθεί να βρει και να ξεφύγει
χρησιμοποιώντας ένα \textlatin{mounted Docker socket} (βλ. ενότητα 4.1.5).
