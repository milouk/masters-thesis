\chapter{Δοκιμές Διείσδυσης \textlatin{Docker} / Βήματα}
\label{dockerPenTesting}

Στο κεφάλαιο 4 μελετήσαμε συγκεκριμένα τρωτά σημεία. Προτού μπορέσουμε να τα
εκμεταλλευτούμε όμως, πρέπει πρώτα να πραγματοποιήσουμε αναγνώριση στο
σύστημα στόχου για τη συλλογή δεδομένων. Αυτά τα δεδομένα μπορούν στη συνέχεια
να χρησιμοποιηθούν για τον εντοπισμό αδύναμων σημείων και
τρωτών σημεία. Αυτό το κεφάλαιο θα επικεντρωθεί στη συλλογή αυτών των
ενδιαφερόντων δεδομένων και τον εντοπισμό αυτών των τρωτών σημείων.
Στην ενότητα 5.1 εστιάζουμε στο πώς να το κάνουμε αυτό χειροκίνητα και για τις
δύο οπτικές του κεφαλαίου 3. Στην ενότητα 5.2 θα ερευνήσουμε τα διαθέσιμα
εργαλεία που θα μας βοηθήσουν να αυτοματοποιήσουμε μέρος των αξιολογήσεων. Στο
Κεφάλαιο 6 θα συνδυάσουμε τις πληροφορίες από το κεφάλαιο 4 και από αυτό το
κεφάλαιο σε μια λίστα ελέγχου.

\section{Χειροκίνητη αναγνώριση ευπαθειών}

Σε αυτήν την ενότητα θα συζητήσουμε πώς μπορούμε να προσδιορίσουμε με μη
αυτόματο τρόπο τα τρωτά σημεία που εξετάσαμε στο κεφάλαιο 4 μόλις έχουμε
πρόσβαση σε ένα σύστημα. Αυτή η ενότητα χωρίζεται σε τρία μέρη, που
αντιστοιχούν στα μοντέλα επιτιθέμενων του κεφαλαίου 3.

Στην ενότητα 5.1.1 εξετάζουμε τεχνικές για να προσδιορίσουμε ποιο μοντέλο
επιτιθέμενου είναι σχετικό κατά τη διάρκεια μιας αξιολόγησης. Αυτό σημαίνει ότι
θα συζητήσουμε τεχνικές για να προσδιορίζουμε αν βρισκόμαστε μέσα σε ένα
κοντέινερ ή σε έναν κεντρικό υπολογιστή.

Το δεύτερο μέρος (ενότητα 5.1.2) αντιστοιχεί αποκλειστικά στις διαφυγές
κοντέινερ (ενότητα 3.1). Παίρνουμε την οπτική μιας διαδικασίας μέσα σε ένα
δοχείο και θα δούμε πώς θα μπορούσαμε να εκτελέσουμε μια επίθεση διαφυγής
κοντέινερ.

Το τρίτο μέρος (ενότητα 5.1.3) αντιστοιχεί σε \textlatin{Docker daemon}
επιθέσεις (ενότητα 3.2). Παίρνουμε την οπτική μιας (μη προνομιούχου) διαδικασίας
σε έναν κεντρικό υπολογιστή με εγκατεστημένο το \textlatin{Docker} και θα δούμε
πώς θα μπορούσαμε 

Θα επικεντρωθούμε κυρίως στις εσφαλμένες διαμορφώσεις (ενότητα 4.1), επειδή
αν και τα σφάλματα που σχετίζονται με την ασφάλεια (ενότητα 4.2) ενδέχεται να
έχουν μεγάλο αντίκτυπο, όλα αυτά μετριάζονται με μια απλή συμβουλή:
«Διατηρήστε τα συστήματά σας ενημερωμένα». Ο έλεγχος εάν ένα σύστημα είναι
ευάλωτο σε ένα γνωστό σφάλμα είναι επίσης πολύ πιο εύκολο από το να ελέγξουμε
εάν ένα σύστημα είναι ευάλωτο λόγω εσφαλμένης διαμόρφωσης, επειδή όλα τα
σφάλματα \textlatin{Docker} εξαρτώνται από την μη ενημερωμένη έκδοση του
\textlatin{Docker}.

\subsection{Αναγνώριση του αν βρισκόμαστε σε περιβάλλον κοντέινερ}

Στις περισσότερες αξιολογήσεις ασφαλείας και δοκιμές διείσδυσης θα είναι
ξεκάθαρο τι είδους σύστημα (δηλαδή αν είμαστε μέσα σε ένα κοντέινερ ή όχι)
επιτιθέμεθα. Σε ορισμένες περιπτώσεις, ωστόσο, μπορεί να μην είναι. Ένα καλό
παράδειγμα αυτού είναι η εύρεση ενός \textlatin{RCE} σε ένα σύστημα κατά τη
δοκιμή διείσδυσης τύπου \textlatin{black box}. Αυτό μας επιτρέπει να εκτελούμε
αυθαίρετες εντολές σε ένα σύστημα για το οποίο δεν γνωρίζουμε τίποτα. Σε μια
τέτοια περίπτωση είναι σημαντικό να ξέρουμε αν είμαστε μέσα σε περιβάλλον
\textlatin{Docker} κοντέινερ ή όχι.

Σε αυτή την ενότητα, θα δούμε βήματα που μας δείχνουν αν βρισκόμαστε σε ένα
κοντέινερ \textlatin{Docker}. Αυτά τα βήματα είναι σε φθίνουσα σειρά ευκολίας
και βεβαιότητας. Εάν γνωρίζουμε ότι βρισκόμαστε μέσα σε ένα κοντέινερ, μπορούμε
να κάνουμε \textlatin{reconnaissance} δηλαδή αναγνώριση μέσα το κοντέινερ (βλ.
ενότητα 5.1.2). Αν ξέρουμε ότι δεν τρέχουμε μέσα σε ένα κοντέινερ, μπορούμε να
εκτελέσουμε αναγνώριση στον κεντρικό υπολογιστή (βλ. ενότητα 5.1.3).

\subsection{\texttt{\textlatin{/.dockerenv}}}

Το \texttt{\textlatin{/.dockerenv}} είναι ένα αρχείο που υπάρχει σε όλα τα
κοντέινερ \textlatin{Docker}. Χρησιμοποιήθηκε στο παρελθόν από την
\textlatin{LXC} για να φορτώσει τις μεταβλητές περιβάλλοντος στο κοντέινερ
\footnote{Η \textlatin{LXC} ήταν το \textlatin{engine} που χρησιμοποιούσε το
\textlatin{Docker} προκειμένου να δημιουργήσει κοντέινερς. Πλέον έχει
αντικατασταθεί με το \texttt{\textlatin{containerd}}}.
Προς το παρόν είναι πάντα άδειο, γιατί το \textlatin{LXC} δεν χρησιμοποιείται
πλέον. Ωστόσο, εξακολουθεί να χρησιμοποιείται (επίσημα) για να προσδιορίσει εάν
μια διεργασία εκτελείται σε ένα \textlatin{Docker} κοντέινερ
\cite{Metasploit-Linux-Gather-Container-Detection}
\cite{Removed-Dockerinit-Reference}.


\subsection{\textlatin{Control Group}}

Για να περιορίσει τους πόρους των κοντέινερ, το \textlatin{Docker} δημιουργεί
ομάδες ελέγχου για καθένα κοντέινερ και μια ομάδα γονικού ελέγχου
(\textlatin{parent control group}) που ονομάζεται \texttt{\textlatin{docker}}.
Εάν ξεκινήσει μια διαδικασία σε ένα κοντέινερ \textlatin{Docker}, αυτή η
διαδικασία θα πρέπει να βρίσκεται στην ομάδα ελέγχου εκείνου του κοντέινερ.
Μπορούμε να το επαληθεύσουμε αυτό κοιτάζοντας την \texttt{\textlatin{cgroup}}
της αρχικής διεργασίας \texttt{\textlatin{(/proc/1/cgroups)}}
\cite{Metasploit-Linux-Gather-Container-Detection}. \\

\texttt{\textlatin{(cont)\# cat /proc/1/cgroup}}

\texttt{\textlatin{12:hugetlb:/docker/0c7a3b8...}}

\texttt{\textlatin{11:blkio:/docker/0c7a3b8...}}

\texttt{\textlatin{...}}

Αν κοιτάξουμε έναν κεντρικό υπολογιστή, δεν βλέπουμε το ίδιο
\texttt{\textlatin{/docker/}} \textlatin{parent control group}. \\

\texttt{\textlatin{(cont)\# cat /proc/1/cgroup}}

\texttt{\textlatin{12:hugetlb:/}}

\texttt{\textlatin{11:blkio:/}}

\texttt{\textlatin{...}}

Σε ορισμένα συστήματα που χρησιμοποιούν \textlatin{Docker} (π.χ. λογισμικό
ενορχήστρωσης), το \textlatin{parent control group} έχει άλλο όνομα (π.χ.
\texttt{\textlatin{kubepod}} για \textlatin{Kubernetes}).

\subsection{\textlatin{Runninb Processes}}



\subsection{Αναγνώριση Ευπαθειών}
\subsection{Εργαλεία}
\subsection{Εύρεση \textlatin{host} / Ευπάθειες \textlatin{Container}}
\subsection{Εκμετάλλευση}
