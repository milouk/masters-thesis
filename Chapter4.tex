\chapter{Μοντέλα Επίθεσης}
\label{attackModes}

Όταν συζητάμε για κοντέινερς, κάνουμε τη διάκριση μεταξύ δύο οπτικών: 
\emph{μέσα} σε ένα κοντέινερ και \emph{έξω} από αυτό.

Όταν είμαστε \emph{μέσα} σε ένα δοχείο, βλέπουμε το κοντέινερ σαν μια
διαδικασία που τρέχει μέσα σε αυτό το κοντέινερ. Αυτή η διαδικασία (και
επομένως η άποψή μας) έχει απομονωθεί από τον κεντρικό υπολογιστή και μπορεί να
δει μόνο αρχεία και πόρους που αφορούν συγκεκριμένα εκείνο το κοντέινερ. Αυτό
σημαίνει ότι μπορούμε να εκτελούμε εντολές, αλλά μόνο μέσα στο κοντέινερ.

Όταν βρίσκεστε \emph{έξω} από ένα κοντέινερ, βλέπουμε τον κεντρικό υπολογιστή
και τα κοντέινερ να τρέχουν στο
host σαν μια διαδικασία που εκτελείται στον κεντρικό υπολογιστή. Μπορούμε να
δούμε τα πάντα σε αυτόν τον κεντρικό υπολογιστή (στον οποίο έχουμε πρόσβαση
επίσης). Για παράδειγμα, μπορούμε να δούμε το διεργασία
\textlatin{Docker daemon} και όλες οι θυγατρικές διεργασίες της. Είμαστε σε
θέση να εκτελέσουμε εντολές απευθείας στον κεντρικό υπολογιστή. Μπορούμε να
χρησιμοποιήσουμε το \textlatin{Docker} (π.χ. αλληλεπίδραση με κοντέινερς) εάν
έχουμε δικαιώματα να χρησιμοποιήσουμε το \textlatin{Docker}. \\

Μπορούμε να σκεφτούμε αυτές τις οπτικές ως μοντέλα επιτιθέμενων. Ένα μοντέλο
επίθεσης είναι μια γενική αναπαράσταση του πώς ένας εισβολέας θα έκανε επίθεση
σε ένα συγκεκριμένο σύστημα. Επειδή έχουμε δύο οπτικές όταν σκεφτόμαστε τα
κοντέινερ, βλέπουμε δύο μοντέλα επιτιθέμενων.

Μπορούμε να σκεφτούμε την πρώτη προοπτική (μέσα σε ένα κοντέινερ) ως ένα μοντέλο
επίθεσης όπου ένας εισβολέας έχει αποκτήσει πρόσβαση σε ένα κοντέινερ.
Ο επιτιθέμενος μπορεί να εκτελεί εντολές μέσα στο κοντέινερ και να έχει
πρόσβαση σε όλα μέσα στο κοντέινερ. Επειδή ο επιτιθέμενος θα επικεντρωθεί
κυρίως στην απόδραση του από την απομόνωση που φέρνει το κοντέινερ, ονομάζουμε
αυτό το είδος επίθεσης κοντέινερ \textlatin{escapes}.

Μπορούμε να σκεφτούμε τη δεύτερη προοπτική (έξω από ένα κοντέινερ) ως ένα
μοντέλο επίθεσης όπου ο επιτιθέμενος έχει μη προνομιούχα πρόσβαση σε έναν
κεντρικό υπολογιστή (\textlatin{host}). Ο επιτιθέμενος είναι σε θέση να
εκτελέσει εντολές στον κεντρικό υπολογιστή, αλλά δεν έχει πρόσβαση
σε όλα. Επειδή ο επιτιθέμενος χρησιμοποιεί το \textlatin{Docker} (συγκεκριμένα
το \textlatin{Docker daemon}) στον κεντρικό υπολογιστή για πρόσβαση,
ονομάζουμε αυτόν τον τύπο επίθεσης "επίθεση στο \textlatin{Docker daemon}".
Αναφέρουμε περαιτέρω τις επιθέσεις \textlatin{Docker daemon} σε επόμενη ενότητα.

Στα ακόλουθα κεφάλαια θα συζητήσουμε θέματα ευπάθειας στο \textlatin{Docker}
(κεφάλαιο 4) και τον τρόπο αναγνώρισής τους (κεφάλαιο 5). Θα το κάνουμε αυτό
χρησιμοποιώντας το μοντέλα επίθεσης αυτού του κεφαλαίου.

Για να διευκρινίσουμε τα μοντέλα επίθεσης, θα ρίξουμε μια ματιά στην εικόνα στο
παρακάτω σχήμα. Με βέλη  απεικονίζουμε τι επιτίθεται σε τι.

\begin{figure}[ht]
    \centering
    \begin{tikzpicture}[x=0.75pt,y=0.75pt,yscale=-1,xscale=1]
        % Host Rectangle
        \draw (0,100) -- (400,100) -- (400,130) -- (0,130) -- cycle ;
        \draw (200, 115) node {Κεντρικός Υπολογιστής};

        %Docker Daemon Rectangle
        \draw (200,75) -- (400,75) -- (400,95) -- (200,95) -- cycle ;
        \draw (300,85) node {\textlatin{Docker Daemon}};

        % Process A Rectangle
        \draw (0,40) -- (85,40) -- (85,70) -- (0,70) -- cycle ;
        \draw (42.5,55) node {Διαδικασία 1};

        % Process B Rectangle
        \draw (105,40) -- (190,40) -- (190,70) -- (105,70) -- cycle ;
        \draw (147.5,50) node {Διαδικασία 2};
        \draw (147.5,62.5) node {{\footnotesize (\textlatin{unprivileged})}};

        %Container Process C Rectangle
        \draw (200,0) -- (295,0) -- (295,70) -- (200,70) -- cycle ;
        \draw (247.5,10) node {Κοντέινερ};

        %% Process C Rectangle
        \draw (205,20) -- (290,20) -- (290,50) -- (205,50) -- cycle ;
        \draw (247.5,35) node {Διαδικασία 3};

        %Container Process D Rectangle
        \draw (305,0) -- (400,0) -- (400,70) -- (305,70) -- cycle ;
        \draw (352.5,10) node {Κοντέινερ};

        %% Process D Rectangle
        \draw (310,20) -- (395,20) -- (395,50) -- (310,50) -- cycle ;
        \draw (352.5,35) node {Διαδικασία 4};

    \end{tikzpicture}
    \caption{}\label{fig:attacker-model-empty}
    \medskip
    \small
    Δύο διαδικασίες που τρέχουν απευθείας στο κεντρικό υπολογιστή και δύο οι οποίες τρέχουν μέσα σε \textlatin{Docker} κοντέινερς.
\end{figure}

Βλέπουμε τις ακόλουθες διαδικασίες που απεικονίζονται στις εικόνες:

\begin{enumerate}
    \item Μια τυπική (προνομιούχα) διαδικασία που εκτελείται απευθείας στον κεντρικό
    υπολογιστή.
    \item Μια τυπική μη προνομιούχα διαδικασία που εκτελείται απευθείας στον κεντρικό
    υπολογιστή.
    \item Μια διεργασία που εκτελείται σε ένα κοντέινερ \textlatin{Docker}.
    \item Όμοια με το 3
\end{enumerate}

\section{Διαφυγές Κοντέινερ \textlatin{Container Escapes}}

Σε μια διαφυγή κοντέινερ, ένας επιτιθέμενος έχει αποκτήσει πρόσβαση σε ένα
δοχείο και προσπαθεί να ξεφύγει από την απομόνωσή του. Όταν ένας επιτιθέμενος
αποκτά πρόσβαση σε ένα κοντέινερ, έχουν κερδίσει μια βάση μέσα στο στόχο τους,
αλλά αυτή η βάση είναι (όπως όλα τα άλλα μέσα στο κοντέινερ) απομονωμένη από
τον κεντρικό υπολογιστή (\textlatin{host}). Οι διαφυγές κοντέινερ εστίαζουν
στην επίθεση και την παράκαμψη των μηχανισμών απομόνωσης και προστασίας που
διαχωρίζουν το κοντέινερ από τον κεντρικό υπολογιστή και άλλα κοντέινερς.

Στο σχήμα που ακολουθεί βλέπουμε δύο παραλλαγές των αποδράσεων κοντέινερς.
Βλέπουμε τη διαδικασία 3 να έχει πρόσβαση στη διαδικασία 2, η οποία είναι μια
διαδικασία που εκτελείται απευθείας στον κεντρικό υπολογιστή.
Βλέπουμε επίσης τη διαδικασία 3 να αποκτά πρόσβαση στη διαδικασία 4, η οποία
βρίσκεται μέσα σε ένα άλλο κοντέινερ. Και στις δύο περιπτώσεις η διαδικασία 3
διαφεύγει από την απομόνωση του κοντέινερ και αποκτά πρόσβαση σε δεδομένα στα
οποία δεν θα πρέπει να έχει πρόσβαση.

\begin{figure}[ht]
    \centering
    \begin{tikzpicture}[x=0.75pt,y=0.75pt,yscale=-1,xscale=1]
        % Host Rectangle
        \draw (0,100) -- (400,100) -- (400,130) -- (0,130) -- cycle ;
        \draw (200, 115) node {Κεντρικός Υπολογιστής};

        %Docker Daemon Rectangle
        \draw (200,75) -- (400,75) -- (400,95) -- (200,95) -- cycle ;
        \draw (300,85) node {\textlatin{Docker Daemon}};

        % Process A Rectangle
        \draw (0,40) -- (85,40) -- (85,70) -- (0,70) -- cycle ;
        \draw (42.5,55) node {Διαδικασία 1};

        % Process B Rectangle
        \draw (105,40) -- (190,40) -- (190,70) -- (105,70) -- cycle ;
        \draw (147.5,50) node {Διαδικασία 2};
        \draw (147.5,62.5) node {{\footnotesize (\textlatin{unprivileged})}};

        %Container Process C Rectangle
        \draw (200,0) -- (295,0) -- (295,70) -- (200,70) -- cycle ;
        \draw (247.5,10) node {Κοντέινερ};

        %% Process C Rectangle
        \draw (205,20) -- (290,20) -- (290,50) -- (205,50) -- cycle ;
        \draw (247.5,35) node {Διαδικασία 3};

        %Container Process D Rectangle
        \draw (305,0) -- (400,0) -- (400,70) -- (305,70) -- cycle ;
        \draw (352.5,10) node {Κοντέινερ};

        %% Process D Rectangle
        \draw (310,20) -- (395,20) -- (395,50) -- (310,50) -- cycle ;
        \draw (352.5,35) node {Διαδικασία 4};

        % Lines
        \draw [latex-,very thick] (190,55) -- (205,35) ;
        \draw [-latex, very thick] (290,35) -- (310,35) ;
    \end{tikzpicture}
    \caption{}\label{fig:container-escape}
    \medskip
    \small
    Η διαδικασία 3 τρέχει μέσα σε ένα κοντέινερ έχει πρόσβαση σε δεδομένα του κεντρικού υπολογιστή (στα οποία δεν θα έπρεπε να έχει πρόσβαση), στη περίπτωση αυτή στη διαδικασία 2
\end{figure}

Στην πρώτη παραλλαγή, η διαδικασία 3 διαφεύγει από το κοντέινερ για να
αποκτήσει πρόσβαση σε δεδομένα που δεν θα πρέπει να έχει πρόσβαση στον
κεντρικό υπολογιστή.

Στη δεύτερη παραλλαγή, η διαδικασία 3 δραπετεύει από το κοντέινερ και αποκτά
προσβάση σε άλλο κοντέινερ. Τα κοντέινερς δεν πρέπει να απομονώνονται μόνο από
τον κεντρικό υπολογιστή, αλλά και από άλλα κοντέινερς. Αυτό επιτρέπει πολλαπλά
κοντέινερς με ευαίσθητα δεδομένα που θα εκτελούνται στον ίδιο κεντρικό
υπολογιστή να μην έχουν πρόσβαση το ένα στα δεδομένα του άλλου.
